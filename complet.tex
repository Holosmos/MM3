\documentclass{livre}
\titre{Cours complet de MM3}

\newcommand{\card}{\sharp}
\renewcommand{\div}{\mid}
\newcommand{\iR}{\barre{\R}}

\begin{document}


\tableofcontents
\printbibliography

\mainmatter

\part{Algèbre}
\parttoc
\chapter{Groupes et groupes symétriques}

\section{Introduction}

\subsection{Groupe abstrait}

\definition{ 
Un groupe est la donnée d'un couple $(G,\cdot)$ où $G$ est un ensemble et $\cdot : G\fois G \vers G$ une loi de composition interne, telle que :
\begin{enumerate}
\item associativité : \[ \forall a,b,c\in G, \; (a\cdot b) \cdot c = a \cdot (b\cdot c) ;\]
\item existence de l'élément neutre $e\in G$ : \[ \forall g \in G, \; g\cdot e = e\cdot g = g ;\]
\item existence de l'inverse : \[\forall x\in G, \exists y \in G, \; x\cdot y  = y\cdot x = e. \]
\end{enumerate}
}{}

\paragraph{Notations}Pour un groupe multiplicatif on note $ab$ l'élément $a\cdot b$, l'élément neutre est noté $1$ et l'inverse de $a$ est noté de $a^{-1}$.

\demonstration{ 
Soient $e,e'$ deux éléments neutres. Alors \[ e' =e\cdot e' = e.\]

Soient $b,c$ inverses de $a$. Alors : \[ b = b\cdot a \cdot c = c.\]
}{Unicité de l'élément neutre et de l'inverse}

\subsection{Groupe commutatif}

\definition{ 
Un groupe $G$ est commutatif si la loi de composition l'est : \[\forall x,y\in G, \; xy = yx. \]
}{Groupe commutatif (ou Abélien)}
\paragraph{Notations}
En général la loi de composition d'un tel groupe est notée comme un groupe additif $(G,+)$. Le neutre est alors $0$ et l'inverse de $x$ est $-x$.

\subsection{Exemples}

\begin{itemize}
\item Le couple $(\Z,+)$ est un groupe abélien où $+$ est l'addition usuelle des entiers.
\item $(\R,+)$ et $(\Q,+)$ sont également des groupes abéliens.
\item $(\R\prive{0},\fois)$ et $(\Q\prive{0},\fois)$ sont des groupes abéliens.
\item $\GL(n,\R)$ est un groupe pour la composition de matrices en tant que loi de composition. Ce n'est pas un groupe commutatif.
\end{itemize}


\section{Sous-groupe}

\subsection{Sous-groupe}
\definition{ 
Soit $G$ un groupe (multiplicatif) et $H\dans G$ un sous-ensemble de $G$. $H$ est un sous-groupe de $G$ si c'est un groupe avec la loi de composition et d'inverse astreintes à $H$\footnotemark.
}{Sous-groupe}
\footnotetext{C'est-à-dire si $H$ est stable par l'application $(x,y) \donne xy^{-1}$.}

\proposition{ 
Soit $G$ un groupe.

 Si $(H_i)_{i\in I}$ est une famille de sous-groupes de $G$ alors $\Inter_{i\in I}^{}H_i$ est un sous-groupe de $G$.
}{}
\definition{ 
Pour tout $i\in I$, $H_i$ vérifie la propriété de sous-groupe et donc l'intersection aussi.
}{}

\paragraph{Remarque}Généralement la réunion de sous-groupes n'est pas un sous-groupe. En effet si $x\in H_1$ et $y\in H_2$ alors il n'y a aucune raison que $xy\in \Union H_i$.

Pour une équivalence il faut rajouter une hypothèse. Si $H,K$ sont deux sous-groupes de $G$ alors $H\union K$ est un sous-groupe si, et seulement si, $H\dans K$ ou $K \dans H$.

En effet supposons $H\not\dans K$ et que $H\union K$ est un sous-groupe. Si $K\not\dans H$ alors on peut choisir $x\in K-K\inter H$ et $y\in H-K\inter H$. On a $x,y\in K\union H$ et donc par hypothèse $xy\in H \union K$ et donc il existe des inverses respectifs $x^{-1},y^{-1}$. Supposons $xy\in H$ : $H\ni (xy)y^{-1}=  xe=x\in H$ absurde.


\definition{ 
Si $G$ est un groupe et $X$ une partie de $G$ alors on appelle sous-groupe de $G$ engendré par $X$ le plus petit sous-groupe de $G$ contenant $X$. On le notera ici $\langle X \rangle$.

On a de plus si on note $\mathbb{G}$ l'ensemble des sous-groupes de $G$ : \[ \langle X \rangle  = \Inter_{H \in \mathbb{G} \et H \contient X} H.\]
}{Groupe engendré}

\paragraph{Exemple}Soit $G$ un groupe et $x \in G$. Alors : \[ \langle x \rangle = \enstq{x^{k}}{k\in \Z}.\]
En effet c'est un sous-groupe de $\langle x \rangle$ et le plus petit.

\subsection{Ordre d'un groupe et d'un élément}

\definition{ 
Si $G$ est un groupe fini, on appelle \textit{ordre de} $G$ son cardinal, on le note généralement $\abs{G}$ ou $\sharp G$.

Si $G$ est un groupe et $x\in G$ alors on appelle \textit{ordre de} $x$ le cardinal de son sous-groupe engendré (s'il est fini).

Dans le cas où le groupe en question ne serait pas fini, on dit que l'ordre est infini.
}{Ordre d'un groupe}

\paragraph{Exemples}
\begin{itemize}
\item Dans $\Z$, tous les éléments non nuls sont d'ordre infini.
\item Dans $\Z/n\Z$ pour $n\in \N^{*}$, $\Z/n\Z$ est d'ordre $n$ puisque toute classe admet un représentant dans $\ens{0,\ldots, n-1}$.
\item Ordre des éléments de $\Z/4\Z$ : 
\[ \begin{matrix}
x & \vline & \barre{0} & \vline & \barre{1} & \vline & \barre{2} & \vline & \barre{3} \\
\abs{x} & \vline & 1 & \vline & 4 & \vline & 2 & \vline & 4
\end{matrix}\]
\end{itemize}


\theoreme{ 
Pour tout groupe $G$ et tout sous-groupe $H$ de $G$, l'ordre (i.e. le cardinal) de $H$ divise l'ordre de $G$ : \[\card H \div \card G. \]
}{Théorème de \textsc{Lagrange}}
\demonstration{
Le cardinal de l'ensemble $G/H$ est appelé \textit{indice} de $H$ dans $G$ et est noté $[G:H]$. 
De plus, ses classes forment une partition de $G$ et chacune d'entre elles a le même cardinal que $H$. On a alors : \[\card G = \card H \times [G:H]. \]
}{Théorème de \textsc{Lagrange}}

\section{Morphisme de groupes}

\subsection{Morphisme de groupes}

\definition{ 
Soient $G,H$ deux groupes. Une application $f : G \vers H$ est un morphisme de groupes si : \[\forall x,y\in G, \; f(x\cdot y) = f(x)\cdot f(y). \]
}{}

\proposition{ 
Soient $f:G \vers H$ un morphisme de groupes. Alors : 
\begin{enumerate}
\item $f(e_G) = e_H$ ;
\item $\forall x\in G, \; f(x^{-1}) = f(x)^{-1}$
\end{enumerate}
}{}

\subsection{Image et noyau}
\definition{ 
Soit $f : G \vers H$ un morphisme de groupes. On définit :
\begin{enumerate}
\item $\Ker(f) = \enstq{x\in G}{f(x) = e}$ ;
\item $\im(f) = \enstq{f(x)}{x\in G}$.
\end{enumerate}
}{}

\proposition{ 
Soit $f: G\vers H$ un morphisme de groupes.
\begin{enumerate}
\item $\Ker(f)$ et $\im(f)$ sont des sous-groupes de $G$ et $H$ respectivement ;
\item $f$ est injective si, et seulement si, $\Ker(f) = \ens{e}$ ;
\item $f$ est surjective si, et seulement si, $\im(f) = H$.
\end{enumerate}
}{}
\demonstration{ Point par point :

\begin{enumerate}
\item On a bien entendu $f(e) = e$ et $f(x)^{-1} = f(x^{-1})$ pour tout $x\in G$. Ainsi $\im(f) = f(G)$ est un sous-groupe de $H$.

Soient $x,y \in G$, alors $f(xy^{-1}) = f(x)f(y^{-1}) = ee^{-1} = e$ donc $xy^{-1}\in G$. De plus $f(e) = e$ donc $\Ker(f)$ est un sous-groupe de $G$.
\item Soient $x,y\in G$ : \[\left(f(x) = f(y) \ssi x = y \right)\ssi \left( f(xy^{-1}) = e \ssi xy^{-1} = e \right). \]
\item Par définition, si $\im(f) = H$ alors $f$ est surjective et réciproquement.
\end{enumerate}
}{}

\section{Groupe symétrique}
\subsection{Groupe de permutations}
\definition{ 
Soit $E$ un ensemble. On définit : \[S_E = \ens{\text{bijections}\ E\vers E}. \]
La loi étant la composition des applications. Elle est associative, admet un élément neutre (application identité) et toute application admet une application inverse par définition.
}{}

\proposition{ 
Si $\card E = n$ alors $S_E$ est isomorphe (au sens de groupes) à $S_{\ens{1,2,\ldots,n}} := S_n$.
}{}
\demonstration{ 
Puisque $\card E = n$ il existe une bijection $\phi = E \vers \ens{1,2,\ldots,n}$. On considère alors l'application de $\theta : S_E \vers S_n$ définie par : $\omega \donne \phi\rond\omega\rond \phi^{-1}$. Comme $\omega,\phi$ sont des bijections, l'application $\phi\rond\omega\rond \phi^{-1}$ est une bijection. L'application  $\theta$ est bien définie.

On a :
\begin{align*}
\theta(\omega' \rond \omega) &= \phi\rond(\omega'\rond \omega)\rond \phi^{-1} \\
\theta(\omega' \rond \omega) &= \phi \rond \omega' \rond \id \rond \ \omega \rond \phi^{-1} \\
 \theta(\omega' \rond \omega) &= \theta(\omega')\rond \theta(\omega).
\end{align*}
$\theta$ est bien un morphisme de groupes.
On a $\theta^{-1}(\omega) = \phi^{-1} \rond \omega \rond \phi$ qui fait de $\theta$ une bijection.
}{}

\definition{ 
On appelle $S_n$ le \textit{groupe symétrique}.
}{Groupe symétrique}
\paragraph{Remarque}On omet la notation $\rond$. Si $\omega \in S_n$ on décrit son action sur $\ens{1,2,\ldots,n}$ par : \[ \matrice{1 & 2 & \ldots & n \\ \omega(1) & \omega(2) & \ldots & \omega(n)}.\]

\paragraph{Exemple de composition}Dans $S_4$ : \[ \matrice{1 & 2 & 3 & 4 \\ 1 & 4 & 3 & 2} \matrice{1 & 2 & 3 & 4 \\ 2 & 1 & 4 & 3} = \matrice{1 & 2 & 3 & 4 \\ 4 & 1 & 2 & 3}.\]
\subsection{Transpositions et cycles}
\definition{ 
Une \textit{transposition} de $S_n$ est une permutation qui échange deux éléments et laisse invariants les $n-2$ autres.
}{Transposition}
\paragraph{Notation}Pour tous $i,j \in \ens{1,2,\ldots,n}$ avec $i\neq j$ on note $(ij)$ la transposition : \[ (ij) : \systeme{i &\donne j \\ j &\donne i \\ k &\donne k, \; \forall k \neq i,j }. \]
\paragraph{Remarque}Une transposition est une involution. C'est à dire que l'ordre d'une transposition est $2$.

\proposition{ 
$\card S_n = n!$.
}{}
\definition{ 
On appelle \textit{cycle} de longueur $r>1$ (noté $r$-cycle) (dans $S_n$) une permutation $\omega$ telle qu'il existe $x_1,x_2,\ldots,x_r \in \ens{1,2,\ldots,n}$ vérifiant : 
\begin{enumerate}
\item $\omega(x_1) = x_2, \omega^{n}(x_1) = x_{1+n}$ avec $n < r$ ;
\item $\omega(x_r) = x_1$ ;
\item $\omega(x) = x$ si $x\not\in\ens{x_1,x_2,\ldots,x_r}$.
\end{enumerate}
}{Cycle}
\paragraph{Notation}On  note un tel cycle : $\matrice{x_1 & x_2 & \ldots & x_r}$.
\paragraph{Remarque}Les $2$-cycles sont exactement les transpositions.
\paragraph{Exemple}Dans $S_3$ : \[ \matrice{1 & 2 & 3 \\ 2 & 3 & 1} = \matrice{1 & 2 & 3} = \systeme{1 \donne 2 \\ 2 \donne 3 \\ 3 \donne 1}.\]

\subsection{Décomposition des cycles}

\definition{ 
On appelle \textit{support} du cycle $\omega$ le sous-ensemble : \[ \ens{x_1,x_2,\ldots,x_r} \dans \ens{1, 2 ,\ldots, n}. \]
}{Support}

\lemme{ 
Deux cycles de supports disjoints commutent.
}{}
\demonstration{ 
Soient : \[ \systeme{v = (x_1,x_2,\ldots,x_r) \\  w = (y_1,y_2,\ldots,y_s)}\] avec $\ens{x_1,x_2,\ldots,x_r}\inter \ens{y_1,y_2,\ldots,y_s} = \vide$.

Sur un élément extérieur du support la permutation agit comme l'identité donc deux supports disjoints impliquent que les permutations associées permutent (puisque que l'identité permute).
}{}

\lemme{ 
Un $r$-cycle est d'ordre $r$.
}{}
\demonstration{ 
Soit $w = \matrice{x_1 & x_2 & \ldots & x_r}$ un $r$-cycle. Il est clair qu'un élément du support est d'ordre $r$. Les autres restent fixés par $w$ et donc $w$ est d'ordre $r$.
}{}

\proposition{ 
Toute permutation de $S_n$ est décomposable en produit de cycles de supports disjoints. Cette décomposition est unique à l'ordre des facteurs près.
}{}

\paragraph{Exemples} Soit : \[S_5 \ni \matrice{1 & 2 & 3 & 4 & 5 \\ 3&  2 & 5 & 4 & 1} = w. \] On peut décomposer $w$ : \[ \matrice{1 & 3 & 5}\matrice{2}\matrice{4} = \matrice{1 & 3 & 5}.\] \[ S_8 \ni w = \matrice{1 & 2 & 3 & 4 & 5 & 6 & 7 & 8 \\ 5 & 6& 1 & 7 & 3 & 8 & 4 & 2} = \matrice{1 & 5 & 3} \matrice{2 & 6 & 8} \matrice{4 & 7}.\]

\theoreme{ 
Le groupe symétrique est engendré par les transpositions.
}{}
\demonstration{
On procède par récurrence sur $n$.
\begin{enumerate}
\item $S_2 = \ens{1, \matrice{1 & 2}}$ est engendré par $\matrice{1 & 2}$.
\item Soit $n>2$, supposons que $S_{n-1}$ est engendré par les transpositions de $S_{n-1}$. Soit $w \in S_n$ :
\begin{enumerate}
\item Soit $w(n) = n$ et alors on décompose $w$ en cycles de tailles inférieures ou égales à $S_{n-1}$ et c'est démontré.
\item Soit $w(n) \neq n$. On pose $m = w(n)$ et soit $t = \matrice{n & m}$. On pose $v = tw$ et alors $v(n) = n$ et on lui applique le cas précédent. On a alors par unicité de la décomposition que $w$ est elle-même engendrée par des transpositions et c'est démontré.
\end{enumerate}
\end{enumerate}
}{}

\theoreme{ On a les propositions suivantes :
\begin{enumerate}
\item Si $w \in S_n$ est une permutation qui s'écrit de deux façons différentes comme produit de transpositions  : \[ w = \tau_1 \tau_2 \ldots \tau_r = \tau_1' \tau_2' \ldots \tau_{r'}',\] alors $(-1)^{r} = (-1)^{r'}$.

On appelle $(-1)^{r}$ la \textit{signature} de $w$.
\item La signature est un morphisme de groupes de $S_n \vers \ens{1,-1} \cong \Z/2\Z$.
\end{enumerate}
}{}
\demonstration{ 
Soit $w\in S_n$. On pose : 
\begin{align*}
\eps(w)  &= \prod_{1 \leq i < j \leq n} \frac{w(i) - w(j)}{i-j} \\
\eps(w) &= \frac{\prod_{1 \leq i < j \leq n} (w(i) - w(j))}{\prod_{1 \leq i < j \leq n}(i-j)}\\
\eps(w) &= \frac{N}{D}.
\end{align*}
Avec
\begin{align*}
N = \prod_{1 \leq i,j \leq n \; ; \; w^{-1}(i) < w^{-1}(j)} (i-j) = \pm D.
\end{align*}
D'où : \[ \eps(w) = \pm 1.\]
}{}
\paragraph{Exemple}$w = \matrice{1 & 2 & 3}$. On a : \[\eps(w) = \frac{ (w(1) - w(2))(w(1) - w(3))(w(2)-w(3))}{(1-2)(1-3)(2-3)} =  \frac{(2-3)(2-1)(3-1)}{(1-2)(1-3)(2-3)} = 1.\]

\lemme{ 
On a :
\begin{enumerate}
\item $\eps : S_n \vers \ens{\pm 1}$ est un morphisme de groupes ;
\item $\eps(ij) = -1$ pour tout $i\neq j$.
\end{enumerate}
}{}
\demonstration{ 
Si \[ w = \tau_1 \tau_2 \ldots \tau_r = \tau_1' \tau_2' \ldots \tau_{r'}'\] alors par le lemme : \[\eps(w) = (-1)^{r} = (-1)^{r'}. \]
}{Théorème}

\demonstration{ 
Soit $E = \enstq{(ij)}{1 \leq i < j \leq n}$. On pose : \[ f_w : \systeme{
E &\vers E \\ 
\matrice{i & j} &\donne \matrice{w(i) & w(j)} \text{ si } w(i) <w(j) \\
\matrice{i & j}& \donne \matrice{w(j) & w(i)} \text{ si } w(i) > w(j)
}.\]
$f$ est une bijection car elle est injective et l'ensemble de départ et d'arrivée ont le même cardinal qui est fini.

Donc on a : 
\begin{align*}
\eps(w) &= \frac{\prod_{1 \leq i < j \leq n}(w(i) - w(j))}{\prod_{(i,j)\in E} (w(i) - w(j))} \\
\eps(w) &= \pm 1.
\end{align*}
Pour vérifier que $\eps$ est un morphisme, on calcul $\eps(wv)$ : 
\begin{align*}
\eps(wv) &=  \prod_{(i,j) \in E} \frac{wv(i) - wv(j)}{i-j} \\
\eps(wv) &= \prod_{(i,j) \in E} \frac{wv(i) - wv(j)}{v(i) - v(j)} \prod_{(i,j) \in E} \frac{v(i) - v(j) }{i - j} \\
\eps(wv) &=  \prod_{(i,j) \in E} \frac{wv(i) - wv(j)}{v(i) - v(j)} \eps(v).
\end{align*}
On calcule :
\begin{align*}
\eps(w) &\overset{?}{=} \prod_{(i,j) \in E} \frac{wv(i) - wv(j)}{v(i) - v(j)} \\
\eps(w) &= \prod_{(i,j) \in E_1} \frac{wv(i) - wv(j)}{v(i) - v(j)}\prod_{(i,j) \in E_2} \frac{wv(i) - wv(j)}{v(i) - v(j)}
\end{align*}
Où $E_1 = \enstq{(i,j) \in E}{v(i) < v(j)}$ et $E_2 = \enstq{(i,j) \in E}{v(j) < v(i)}$ ; $E = E_1 \coprod E_2$.
\begin{align*}
\eps(w) &= \prod_{(i,j) \in E_2} \frac{wv(j) - wv(i)}{v(j) - v(i)}\prod_{(i,j) \in E_1} \frac{wv(i) - wv(j)}{v(i) - v(j)} \\
\eps(w) &= \prod_{i < j  \; ; \; v^{-1}(j) < v^{-1}(i)} \frac{w(i) - w(j)}{i - j} \prod_{i < j \; ; \; v^{-1}(i) < v^{-1}(j)}\frac{w(i)- w(j)}{i-j} \\
\eps(w) &= \prod_{i < j} \frac{w(i) - w(j)}{i - j}
\end{align*}
}{Lemme}
\newpage
\chapter{Déterminants et réduction}
\section{Déterminants}

\subsection{Différentes définitions}

Soit $A \in M_n(\R)$ avec $A = (a_{i,j})_{1\leq i,j\leq n}$

\definition{
On définit en premier lieu : \[ \det A = \sum_{w \in S_n} \eps(w) a_{w(i),1}\cdot a_{w(2),2}\cdot \ldots \cdot a_{w(n),n}. \]
C'est la formule de \textsc{Cramer}.
}{Déterminant}

\definition{ 
Une seconde définition possible :

Pour tous $i,j \in \ens{1,\ldots,n}$, soit $A_{i,j}\in M_{n-1}(\R)$ la matrice (extraite) obtenue en enlevant la $i$-ième ligne et la $j$-ième colonne de $A$.

On a alors : \[ \det' A  =  a_{1,1}\cdot \det'(A_{1,1}) -a_{1,2}\cdot \det'(A_{1,2}) + \ldots + (-1)^{n-1}a_{1,n}\cdot \det'(A_{1,n}) = \sum_{i=1}^{n}(-1)^{i+1}a_{1,i}\cdot \det'(A_{1,i})\]
}{}

\paragraph{Exemple}Prenons : \[A = \matrice{2 & 1 & -1 \\ 0 & 2 & 1 \\ 4 & -1 & 0}. \] On a : \[ A_{1,1} = \matrice{2 & 1 \\ -1 & 0} \; ; \; A_{1,2} = \matrice{0 & 1 \\ 4 & 0}.\]Ce qui donne avec la seconde définition : \[ \det A = 2 \det \matrice{2 & 1 \\ -1 & 0} - \det \matrice{0 & 1  \\ 4 & 0} - \det \matrice{0 & 2 \\ 4 & -1}.\]
\paragraph{Exemple 2}On vérifie que les deux définitions coïncident : \[ \det \matrice{a_{1,1} & a_{1,2} \\ a_{2,1} & a_{2,2}} = a_{1,1}a_{2,2} - a_{2,1}a_{1,2}.\] \[\det \matrice{a_{1,1} & a_{1,2} \\ a_{2,1} & a_{2,2}} = a_{1,1}\det(a_{2,2}) - a_{1,2}\det(a_{2,1}) = a_{1,1}a_{2,2} - a_{2,1}a_{1,2}.\]

\paragraph{Remarque}Soient $E$ un $\R$-espace vectoriel de dimension $n$ et $B = (e_1,\ldots,e_n)$  une base de $E$. Soit $(u_1,u_2,\ldots,u_n) \in E^{n}$ un $n$-uplet  de vecteurs de $E$. Pour tout $j$, on pose : \[ u_j = \sum_{i=1}^{n}a_{i,j}\cdot e_{i}  \; \; a_{i,j}\in \R.\]On appelle \textit{déterminant} dans la base $B$ de $(u_1,\ldots,u_n)$ le réel : \[ \det_B(u_1,u_2,\ldots,u_n) = \det(a_{i,j}).\]

\paragraph{Exemple}Pour $n=2$. On prend :
\begin{align*}
u_1 &= 2e_1 + 3e_2, \\
u_2 &= -e_1 + 6e_2.
\end{align*}
On a alors : \[ \det_B(u_1,u_2) = \det \matrice{2 & -1 \\ 3 & 6} = 15.\]

\paragraph{Remarque}Si $u_j = e_j$ pour tout $j \in \ens{1,\ldots,n}$ alors $\det_B(e_1,\ldots,e_n) = \det(I_d) = 1$. 

\proposition{ On a les énoncés :
\begin{enumerate}
\item pour tout $w \in S_n$ : \[ \det_B(u_{w(1)},u_{w(2)},\ldots,u_{w(n)}) = \eps(w) \det_B(u_1,u_2,\ldots,u_n) ;\]
\item on en déduit que le déterminant change de signe si on échange deux colonnes ;
\item si pour $i\neq j$ on a $u_i = u_j$ alors le déterminant est nul (puisque négatif et positif simultanément).
\end{enumerate}
}{}
\demonstration{ 
Il suffit de montrer le premier point.

On sait que $S_n$ est engendré par les transpositions. On suppose donc que $w\in S_n$ est une transposition. 

En fait, $S_n$ est engendré par les transpositions simples, i.e. les transpositions de la forme $(k,k+1)$ avec $1\leq k < n$.\footnotemark

On suppose donc que $w$ est de la forme $(k,k+1)$. Soit $A$ la matrice $(u_1,u_2,\ldots,u_n)$ de ces $n$ vecteurs dans les coordonnées de la base $B$. Soit $A'$ la matrice obtenue en permutant les colonnes $k$ et $k+1$ de $A$. Il faut donc vérifier que : \[ \det A' = \eps(w) \det A = -\det A.\]On calcule à gauche et à droite :
\begin{align*}
\det A &= \sum_{j=1}^{n}(-1)^{j+1}a_{1,j}\det(A_{1,j}), \\
\det A'&= \sum_{j=1}^{n}(-1)^{j+1}a_{1,j}'\det(A_{1,j}').
\end{align*}
\begin{itemize}
\item Pour $j\neq k,k+1$ on a $a_{1,j}' = a_{1,j}$ et $A_{1,j}'$ est obtenue en échangeant les colonnes $k$ et $k+1$ de $A_{1,j}$
\item Pour $j=k$ on a $a_{1,k}' = a_{1,k+1}$ et donc $A_{1,k}' = A_{1,k+1}$.
\item Pour $j=k+1$ on a $a_{1,k+1}' = a_{1,k}$ et donc $A_{1,k+1}' = A_{1,k}$.
\end{itemize}
On en déduit :
\begin{align*}
\det A' &= \sum_{j\neq k,k+1}(-1)^{j+1}\det(A_{i,j}')\footnotemark + (-1)^{k+1}a_{1,k}'\det(A_{1,k}') + (-1)^{k}a_{1,k+1}'\det(A_{1,k+1}'),\\
\det A' &= \sum_{j\neq k,k+1}(-1)^{j+1}(-\det(A_{i,j}))+ (-1)^{k+1}a_{1,k+1}(-\det(A_{1,k+1}))+ (-1)^{k}a_{1,k}(-\det(A_{1,k})),\\
\det A' &= -\det A.
\end{align*}
}{}
\footnotetext[1]{
En effet, toute transposition est un produit de transpositions simples par une conjugaison adaptée : on \og renomme\fg{} les éléments.}
\footnotetext[2]{Par récurrence sur $n$ on a $\det(A_{i,j}') = - \det(A_{i,j})$.}

\subsection{Formes $n$-linéaires alternées}

\definition{ 
Soit $E$ un $\R$-espace vectoriel de dimension $n\geq 1$. Une forme $n$-linéaire sur $E$ est une application $\varphi : E^{n}\vers \R$ qui est linéaire sur chaque composante.
}{Forme $n$-linéaire}

\proposition{ 
Soit $B$ une base de $E$ avec $\dim E = n$.\[\fonc{\det_B}{E^{n}}{\R}{(u_1,\ldots,u_n)}{\det_B(u_1,\ldots,u_n)}\] est une forme $n$-linéaire.
}{}
\demonstration{ 
On pose : 
\[A = \matrice{ 
a_{1,1} & \ldots & a_{1,k-1} & aa_{1,k}' + ba_{1,k}'' & a_{1,k+1} & \ldots & a_{1,n} \\
a_{2,1} & \ldots & a_{2,k-1} & aa_{2,k}' + ba_{2,k}'' & a_{2,k+1} & \ldots & a_{2,n} \\
\vdots & & \vdots & \vdots & \vdots & & \vdots 
}\]
\[ A' =\matrice{ 
a_{1,1} & \ldots & a_{1,k-1} & a_{1,k}' & a_{1,k+1} & \ldots & a_{1,n} \\
a_{2,1} & \ldots & a_{2,k-1} & a_{2,k}' & a_{2,k+1} & \ldots & a_{2,n}\\
\vdots & & \vdots & \vdots & \vdots & &\vdots
}\]
\[ A'' =\matrice{ 
a_{1,1} & \ldots & a_{1,k-1} & a_{1,k}'' & a_{1,k+1} & \ldots & a_{1,n} \\
a_{2,1} & \ldots & a_{2,k-1} & a_{2,k}'' & a_{2,k+1} & \ldots & a_{2,n}\\
\vdots & & \vdots & \vdots & \vdots & &\vdots
}\]
On veut montrer : \[\det A  = a \det A' + b\det A''. \]
On calcule :
\begin{align*}
\det A &= \sum_{j\neq k}(-1)^{j+1}a_{1,j}\det(A_{i,j}) + (-1)^{k+1}(aa_{1,k}'+ba_{1,k}'')\det(A_{1,k}), \\
\det A' &= \sum_{j\neq k}(-1)^{j+1}a_{1,j}\det(A_{i,j}') + (-1)^{k+1}a_{1,k}'\det(A_{1,k}), \\
\det A'' &= \sum_{j\neq k}(-1)^{j+1}a_{1,j}\det(A_{i,j}'') + (-1)^{k+1}a_{1,k}''\det(A_{1,k}) 
\end{align*}
On doit alors montrer : \[\forall j \neq k, \; \det A_{i,j} = a \det (A_{i,j}') + b\det(A_{i,j}'') \]ce qui est démontré par hypothèse de récurrence.
}{}

\definition{ 
Soit $\varphi : E^{n}\vers \R$ une forme $n$-linéaire alternée avec $E$ un $\R$-espace vectoriel.

$\varphi$ est une forme $n$-linéaire alternée si on a : \[ \varphi(u_1,u_2,\ldots,u_n) = 0 \] dès que deux composantes $u_i,u_j$ avec $i\neq j$ coïncident.
}{Forme $n$-linéaire alternée}
\paragraph{Remarque}On en déduit que le déterminant dans une base donnée est une forme $n$-linéaire alternée.

\proposition{ 
Soit $\varphi$ une forme $n$-linéaire alternée. Alors pour tout $w\in S_n$, $\varphi(u_{w(1)},\ldots,u_{w(n)}) = \eps(w) \varphi(u_1,\ldots,u_n)$.
}{}
\demonstration{ 
On peut supposer que $w$ est une transposition simple : $w = (k,k+1)$ avec $1\leq k < n$.

On veut montrer : \[ \varphi(u_1,\ldots,u_{k-1},u_{k+1},u_k,u_{k+2},\ldots,u_n) = - \varphi(u_1,\ldots,u_n).\]Pour simplifier les notations, on oublie les indices $u_i$ avec $i\neq k,k+1$. On a : \[ \varphi(u_k + u_{k+1}, u_k + u_{k+1}) = 0\] et donc par linéarité : \[\varphi(u_k,u_k) + \varphi(u_k,u_{k+1}) + \varphi(u_{k+1},u_k) + \varphi(u_{k+1},u_{k+1}) = 0 \ssi \varphi(u_k,u_{k+1}) = - \varphi(u_{k+1},u_k).\]
}{}

\proposition{ 
Soient $E$ un $\R$-espace vectoriel de dimension $n$ et $B=(e_1,\ldots,e_n)$ une base de $E$.

Soit $\varphi : E^{n}\vers \R$ une forme $n$-linéaire alternée. Alors : \[ \varphi(u_1,\ldots,u_n) = \det_B(u_1,\ldots,u_n)\varphi(e_1,\ldots,e_n)\]où les $u_i$ sont exprimés dans la base $B$.
}{}
\paragraph{Remarque}Toutes les formes $n$-linéaires alternées sont proportionnelles au déterminant.
\demonstration{ 
Soit $u_j = \sum_{i=1}^{n}a_{i,j}e_i$, les $a_{i,j}$ sont les coordonnées des $u_j$ dans la base $B$.

On a : \[ \varphi(u_1,\ldots,u_n) = \varphi\left( \sum_{i=1}^{n}a_{i,1}e_i, \ldots, \sum_{i=1}^{n}a_{i,n}e_i \right).\]Comme $\varphi$ est $n$-linéaire alternée :
\begin{align*}
\varphi(u_1,\ldots,u_n) &= \sum_{w \in S_n}a_{w(1),1}a_{w(2),2}\ldots a_{w(n),n}\varphi(e_{w(1)},\ldots,e_{w(n)}) \\
\varphi(u_1,\ldots,u_n) &= \sum_{w \in S_n}a_{w(1),1}a_{w(2),2}\ldots a_{w(n),n}\eps(w)\varphi(e_1,\ldots,e_n) \\
\varphi(u_1,\ldots,u_n) &=\det_B(u_1,\ldots,u_n)\varphi(e_1,\ldots,e_n)
\end{align*}
}{}
\paragraph{Remarques}On a démontré :
\begin{enumerate}
\item Pour une base $B$ choisie, le déterminant $\det_B$ est une forme $n$-linéaire alternée ;
\item pour toute forme $n$-linéaire alternée, $\varphi$, on a : $\varphi (\cdot) = \det_B(\cdot)\varphi(B)$ ;
\item en particulier, les deux déterminants coïncident.
\end{enumerate}

\proposition{ 
Pour tout $A \in M_n(\R)$ on a : \[ \det(A) = \det(A^{t}).\]
}{}
\demonstration{ 
On a :
\begin{align*}
A &= (a_{i,j}) \\
A^{t} &= (b_{i,j}), \; b_{i,j} = a_{j,i}
\end{align*}
On calcule par la formule de \textsc{Cramer} : 
\begin{align*}
\det(A^{t}) &= \sum_{w\in S_n}\eps(w)\prod_{i=1}^n b_{w(i),i}, \\
\det(A^{t}) &= \sum_{w\in S_n}\eps(w)\prod_{i=1}^n a_{i,w(i)}.
\end{align*}
Pour $w$ fixé, dans $i$ décrit $1$ à $n$ alors $w(i)$ décrit également $1$ à $n$. On effectue un changement de variable $j = w(i)$ et alors $i = w^{-1}(j)$ et on a :
\begin{align*}
\det(A^{t}) &= \sum_{w\in S_n}\eps(w)\prod_{j=1}^na_{w^{-1}(j),j},\\
\det(A^{t}) &= \sum_{w\in S_n}\eps(w^{-1})\prod_{j=1}^na_{w(j),j}, \\
\det(A^{t}) &= \sum_{w\in S_n}\eps(w)\prod_{j=1}^na_{w(j),j}, \\
\det(A^{t}) &= \det(A).
\end{align*}
}{}
\paragraph{Remarque}On peut calculer $\det(A)$ en développant par rapport à la première ligne ou la première colonne (au choix). On a alors : 
\[\det(A) = \sum_{i=1}^{n}(-1)^{n}a_{i,1}\det(A_{i,1}).\]

\proposition{ 
Si $A\in M_n(\R)$ est triangulaire alors : \[ \det A = \prod_{i=1}^{n} a_{i,i}.\]
}{}
\demonstration{ 
Supposons $A$ triangulaire supérieure, c'est-à-dire $a_{i,j}=0$ si $i>j$. 

Par la formule de \textsc{Cramer} :\[
\det(A) = \sum_{w\in S_n}\eps(w)\prod_{i=1}^{n}a_{i,w(i)}.\]
Or les seuls $w$ qui contribuent à cette somme sont ceux tels que : \[\forall i \in \ens{1,\ldots,n}, i\leq w(i),\] c'est-à-dire : $w = \id$\footnotemark.

En développant par rapport à une ligne (ou une colonne quelconque) : 
\[\det A = \sum_{i=1}^{n}(-1)^{i+j}a_{j,i}\det(A_{j,i}).\]
Si $A'$ désigne la matrice obtenue en permutant les lignes de $A$ par $w= \matrice{1 & 2 & \ldots & j}$ : \[\det (A') = \eps(w)\det(A) = (-1)^{j+1}\det (A) .\]On note $A' = (a_{k,l}')_{k,l \in \ens{1,\ldots,n}}$.

En choisissant $j>1$ :
\begin{align*}
\det(A') &\overset{\footnotemark}{=} \sum_{i=1}^{n}(-1)^{i+1}a_{1,i}'\det(A_{1,i}'), \\
\det(A') &= \sum_{i=1}^{n}(-1)^{i+1}a_{j,i}\det(A_{j,i}) ;\\
\det(A) &= (-1)^{j+1}\det(A'), \\
\det(A) &= \sum_{i=1}^{n}(-1)^{j+i}a_{j,i}\det(A_{j,i}).
\end{align*}
}{}
\footnotetext[3]{Soit $w\in S_n$, $w : \ens{1,2,\ldots,n} \overset{\sim}{\vers}\ens{1,2,\ldots,n}$.

Si $i\leq w(i)$ pour tout $i$ alors $w(k) = k$ pour tout $k$ par récurrence descendante sur $k$ : 
\begin{itemize}
\item $n\leq w(n)$ et donc $w(n)=n$ ;
\item $k-1\leq w(k-1)$ et donc $w(k-1) = w(k)$. 
\end{itemize}
}
\footnotetext{En développant par rapport à la première ligne.}


\section{Déterminant d'un endomorphisme}

\subsection{Invariance par changement de base}

\proposition{ 
Soient $E$ un $\R$-espace vectoriel de dimension $n$, $B = (e_1,e_2,\ldots,e_n)$ une base de $E$ et $C = (u_1,\ldots,u_n)$ un système de $n$ vecteurs de $E$. Alors $C$ est une base de $E$ si, et seulement si : \[ \det_B(C) \neq 0.\]
}{}
\demonstration{ 
Supposons que $C$ est une base de $E$.

On a vu que si $\varphi : E^{n}\vers \K$ est une forme $n$-linéaire alternée alors : \[\forall (u_1,u_2,\ldots,u_n)\in E^{n}, \; \varphi(u_1,u_2,\ldots,u_n) = \det_B(u_1,u_2,\ldots,u_n)\cdot \varphi(e_1,e_2,\ldots,e_n).\]

On applique cette formule avec $ \varphi = \det_C$ et on a : 
\begin{align*}
\det_C(u_1,u_2,\ldots,u_n)&= \det_B(C) \det_C(B),\\
1 = \det_C(C) &= \det_B(C)\det_C(B),
\end{align*}
et donc $\det_B(C)\neq 0$.

Supposons maintenant que $C$ est liée. Il existe alors $i$ tel que $u_i$ est combinaison linéaire des $u_j$ avec $j\neq i$.
Par exemple :
\begin{align*}
u_i &= \sum_{j\neq i}^{}a_j\cdot u_j, \; (a_j\in \R) \\
\det_B(C) &= \det_B(u_1,u_2,\ldots,u_{i-1},\sum_{j\neq i}^{}a_j\cdot u_j,u_{i+1},\ldots,u_n), \\
\det_B(C) &= \sum_{j\neq i}^{}a_j\det_B(u_1,u_2,\ldots,u_{i-1},\underset{j\neq i}{u_j},u_{i+1},\ldots,u_n),
\end{align*}
or $\det_B$ est alternée et comme $u_j$ apparaît deux fois dans la dernière expression, on a $$\det_B(C) = 0.$$ 
}{}

\proposition{ 
Soient $E$ un $\R$-espace vectoriel de dimension $n$, $B = (e_1,\ldots,e_n)$, $C = (u_1,\ldots,u_n)$ deux bases de $E$ et $f$ un endomorphisme de $E$. Alors : \[\det_B(f(e_1),\ldots,f(e_n)) = \det_C(f(u_1),\ldots,f(u_n)). \]
}{}
\paragraph{Remarque}En d'autres termes, $\det_B(f(B))$ ne dépend pas du choix de la base $B$. On l'appelle $\det(f)$.
\demonstration{ 
On utilise la formule : \[\forall (u_1,u_2,\ldots,u_n)\in E^{n}, \; \varphi(u_1,u_2,\ldots,u_n) = \det_B(u_1,u_2,\ldots,u_n)\cdot \varphi(e_1,e_2,\ldots,e_n),\] où $\varphi$ est une forme $n$-linéaire alternée.

On pose : \[\varphi(u_1,\ldots,u_n) = \det_B(f(u_1),f(u_2),\ldots,f(u_n)) \]et on a alors : 
\[\varphi(u_1,\ldots,u_n) = \det_B(f(u_1),\ldots,f(u_n)) = \det_C(f(u_1),\ldots,f(u_n)) \det_B(C).\]
De même : 
\[\det_B(f(u_1),\ldots,f(u_n)) = \det_B(C)\det_B(f(e_1),\ldots,f(e_n)).\]
Et donc :
\[\det_B(f(u_1),\ldots,f(u_n)) \det_B(C) = \det_B(f(e_1),\ldots,f(e_n))\det_B(C) \]
et $\det_B(C)\neq 0$. Donc l'égalité voulue est obtenue. 
}{}

\proposition{ 
Soient $E$ un $\R$-espace vectoriel de dimension $n$, $f$ et $g$ deux endomorphismes de $E$. Alors :
\[ \det(fg)=\det(f)\det(g).\]
}{}
\demonstration{ 
Soit $B=(e_1,\ldots,e_n)$ une base de $E$, \[ \det(fg) = \det_B(fg(e_1),\ldots,fg(e_n)).\]

Considérons la forme $n$-linéaire alternée $\varphi$ telle que : \[\varphi(u_1,\ldots,u_n) = \det_B(g(u_1),\ldots,g(u_n)), \] alors on a : 
\begin{align*}
\varphi(f(u_1),\ldots,f(u_n)) &= \det_B(f(u_1),\ldots,f(u_n))\varphi(e_1,\ldots,e_n), \\
\det_B(gf(e_1),\ldots,gf(e_n)) &= \det_B(f(e_1),\ldots,f(e_n))\det_B(g(e_1),\ldots,g(e_n)), \\
\det(gf)& = \det(g)\det(f).
\end{align*}
}{}
\paragraph{Remarque}Si $A,B\in M_n(\R)$ alors \[ \det(AB) = \det(A) \det(B).\]


\section{Diagonalisation}

\definition{ 
Une matrice $A$ est \textit{diagonalisable} si elle est conjugué par un isomorphisme à une matrice diagonale. 
}{}

\subsection{Valeur propre et vecteur propre}
Soit $E$ un $\R$-espace vectoriel de dimension $n$. Soit $f$ un endomorphisme de $E$.

\definition{ 
On appelle \textit{valeur propre} de $f$ un réel $\lambda$ tel qu'il existe un $v\in E-\ens{0}$ tel que $f(v) = \lambda \cdot v$.

On dit que $v$ est un \textit{vecteur propre} de valeur propre $\lambda$.
}{}

Quitte à prendre la matrice $A$ de $f$ dans une base $(e_1,\ldots,e_n)$ fixée de $E$, $\lambda$ est une valeur de $f$ (ou de $A$) si, et seulement si \[\det (A-\lambda I_d) = 0. \]

\paragraph{Remarque}Soient $A\in M_n(\R)$, $B$ la base canonique et $C = AB$. $\det(A)$ est non nul si, et seulement si, $A$ est inversible. D'autre part s'il existe un vecteur propre $v$ de valeur propre $\lambda$ alors \[ \ker(f-\lambda I_d) \neq \ens{0}.\] Or $f-\lambda I_d$ est un endomorphisme de $E$ et $E$ est de dimension finie. Donc il y a équivalence : \[ \ker(f-\lambda I_d) \neq\ens{0} \ssi \det(A-\lambda I_d) = 0.\]

\definition{ 
On appelle polynôme caractéristique de $f$ (ou de $A$) le polynôme : \[ \chi_f(t) = \chi_A(t) = \det(A-t I_d).\]
}{}
\paragraph{Exemple}En dimension $2$ : $A = \matrice{a & b \\ c & d}$ on a \[\chi_A(t) = t^2 -(a+d)t +ad-bc = t^2 - \tr(A)t + \det(A). \]

\paragraph{Remarque}$\chi_A(t)$ est un polynôme de degré $n$ de coefficient dominant $(-1)^{n}$ et de terme constant $\chi_A(0) = \det(A)$.
\subsection{Sous-espaces propres}
\definition{ 
Soit $f$ un endomorphisme de $E$ et de matrice $A$. Soit $\lambda\in \R$. On appelle \textit{sous-espace propre} de $f$ (ou de $A$) de valeur propre $\lambda$ le sous-espace vectoriel $\ker(f-\lambda I_d)$.
}{}

\proposition{ 
Soient $\lambda,\mu \in \R$. Alors si $\lambda\neq\mu$ on a \[\ker(f-\lambda I_d)\ker(f-\mu I_d) = \ens{0}. \]
Plus généralement si, $\lambda_1,\ldots,\lambda_k\in \R$ distincts alors on a : \[ \sum_{i=1}^{k}\ker(f-\lambda_i I_d) = \bigoplus_{i=1}^{k}\ker(f-\lambda_i I_d)\]
}{}
\demonstration{ 
Il s'agit de vérifier que pour tout $i\neq j$ on a : \[ \ker(f-\lambda_i I_d)\inter \ker(f-\lambda_j I_d) = \ens{0}.\]

Si $v \in \ker(f-\lambda_i I_d)\inter \ker(f-\lambda_j I_d) $ alors : \[ f(v) = \lambda_i v = \lambda_j v\implique v = 0.\]
}{}

\corollaire{ 
Soient $\dim E=n$, $f$ est un endomorphisme de $E$, $\lambda_1,\lambda_2,\ldots,\lambda_k$ valeurs propres de $f$ et $E_i$ le sous-espace associé à la valeur propre $\lambda_i$. Alors si \[ E = \bigoplus_{i=1}^{k}E_i,\]l'endomorphisme $f$ est diagonalisable.
}{}
\demonstration{ 
Si on fait la réunion : \[ B = \Union_{i=1}^{k}B_i,\]où $B_i$ est une base de $E_i$ on obtient une base de $E$. Dans cette base la matrice de $f$ est diagonale où l'élément diagonal $\lambda_i$ est la valeur propre correspondante. La matrice de passage de la base canonique à la base $B$ donne la diagonalisablisation.
}{}

Donc pour diagonaliser $A$ il faut vérifier si $E = \bigoplus_{i=1}^{k}E_i$ où les $E_i$ sous les sous-espaces propres.

\paragraph{Exemple}Soit :
\begin{align*}
A &= \matrice{1 & 2 & -3 \\ 1 & 4 & -5 \\ 0 & 2 & -2}.\\
\chi_A(\lambda) &= \det\matrice{1-\lambda & 2 & -3 \\ 1 & 4-\lambda & -5 \\ 0 & 2 & -2-\lambda},\\
\chi_A(\lambda) &= (1-\lambda)((4-\lambda)(-2-\lambda)+10) - (2(-2-\lambda)+6),\\
\chi_A(t) &= -\lambda(\lambda-1)(\lambda-2).
\end{align*}
Les trois valeurs propres sont $0,1,2$ et sont de multiplicité $1$.

\begin{align*}
E_0 = \ker (A) &= \enstq{x\in \R^{3}}{Ax = 0} = \left\langle \matrice{1 \\ 1 \\ 1}\right\rangle,\\
E_1 = \ker (A-I_d) &= \enstq{x\in \R^{3}}{\matrice{0 & 2 & -3 \\ 1 & 3 & -5 \\ 0 & 2 & -1}x = 0} =\left\langle\matrice{1 \\ 3 \\ 2} \right\rangle,\\
E_2 = \ker(A-2I_d) &= \enstq{x\in \R^{3}}{\matrice{-1 & 2 & -3 \\ 1 & 2 & -5 \\ 0 & 2 & -4}x = 0} = \left\langle \matrice{1 \\ 2 \\ 1}\right\rangle.
\end{align*}
On a l'égalité : \[ E_0 \oplus E_1 \oplus E_2 = \R^{3}.\]
On en déduit les matrices de passage : 
\begin{align*}
P &= \matrice{1 & 1 & 1 \\ 1 & 3 & 2 \\ 1 & 2 & 1}, \\
P^{-1}AP &= \matrice{0 & 0& 0 \\ 0 & 1 & 0 \\ 0 & 0 & 2}.
\end{align*}
\subsection{Conditions de diagonalisabilité}
\proposition{ 
Soient $E$ un $\R$-espace vectoriel de dimension $n$, $f$ un endomorphisme de $E$, $\chi_f(t) \in \R[t]$, $\deg \chi_f = n$.

Si $\chi_f$ admet $n$ racines distinctes alors $f$ est diagonalisable.
}{}
\demonstration{ 
Si : \[ \chi_f(t)  = \prod_{i=1}^{n}(\lambda_i - t)\]avec $\lambda_1,\lambda_2,\ldots,\lambda_n$ racines distinctes.
On a alors que pour tout $i$ : \[ E_i = \ker(f-\lambda_i \id) \neq \ens{0}\] et donc $\dim E_i\geq 1$. On a alors que \[ \sum_{i=1}^{n}E_i = \bigoplus_{i=1}^{n}E_i\] est de dimension supérieure à $n$ ce qui implique $\bigoplus E_i = \R^{n}$.
}{}
\paragraph{Remarque}La condition donnée est nécessaire mais non suffisante. On cherche donc une condition nécessaire et suffisante.

\proposition{ 
Soient $E$ un $\R$-espace vectoriel de dimension $n$, $f$ un endomorphisme de $E$, $\lambda$ une valeur propre de $f$, $m_\lambda$ la multiplicité de $\lambda$ en tant que racine de $\chi_f(t)$ et $E_\lambda$ le sous-espace propre associé à $\lambda$.

Alors $\dim E_\lambda \leq m_\lambda$.
}{}
\demonstration{ 
Soit $k = \dim E_\lambda$ et $(e_1,e_2,\ldots,e_k)$ une base de $E_\lambda$. On peut compléter $(e_1,\ldots,e_k)$ en une base $(e_1,\ldots,e_n)=B$ de $E$. \[ {\rm Mat}_B(f) = \matrice{\lambda I_d& \vline &X \\ \hline 0 & \vline&  A}.\]

Or le déterminant d'une matrice triangulaire par blocs est le produit des déterminants des matrices diagonales.\footnotemark

Ainsi : \[ \chi_f(t) = \det\matrice{(\lambda - t)I_d & \vline & X \\ \hline 0 & \vline & A-t I_d} = (\lambda -t)^{k}\chi_A(t)\] et donc $m_\lambda \geq k$.
}{}
\footnotetext{En effet, en utilisant la règle de \textsc{Cramer} la preuve est assez aisée.}

\corollaire{ 
On a les propositions suivantes :
\begin{enumerate}
\item Si $\chi_f(t)$ n'est pas scindé sur $\R$ alors $f$ n'est pas diagonalisable.
\item S'il existe une valeur propre $\lambda$ de $f$ telle que $\dim E_\lambda < m_\lambda$ alors $f$ n'est pas diagonalisable.
\end{enumerate}
}{}
\demonstration{ 
On démontre :
\begin{enumerate}
\item[2.] Soient $\lambda_1,\ldots,\lambda_k$ les valeurs propres de $\chi_f(t)$, $m_i$ la multiplicité de $\lambda_i$ et $E_i$ l'espace propre associé à $\lambda_i$. Alors la proposition nous dit que $\dim E_i \leq m_i$.

Or $\deg \chi_f(t) = $n et donc 
\begin{align*}
\sum_{i=1}^{k} m_i &\leq n \\
\sum_{i=1}^{k}\dim E_i &\leq \sum_{i=1}^{k}m_i \leq n
\end{align*}
S'il existe $i_0$ tel que $\dim E_{i_0} < m_{i_0}$ alors cela implique \[\sum_{i=1}^{k}\dim E_i < \sum_{i=1}^{k}m_i \leq n.\] Et donc \[ \bigoplus_{i=1}^{n}E_i < n.\]
\item[1.] Idem.
\end{enumerate}
}{}

\theoreme{ 
Soient $E$ un $\R$-espace vectoriel de dimension $n$, $f$ un endomorphisme de $E$.

$f$ est diagonalisable si, et seulement si, on a les conditions suivantes :
\begin{enumerate}
\item $\chi_f(t)$ est scindé sur $\R$ ;
\item pour tout $\lambda \in \chi_f^{-1}(0)$, la dimension du $\ker (f-\lambda \id)$ est égal à la multiplicité de $\lambda$ dans $\chi_f(t)$.
\end{enumerate}
}{}
\demonstration{ 
Le corollaire nous dit que ces conditions sont nécessaires.

Remarquons que : \[ \sum_{i=1}^{r}E_i = \bigoplus_{i=1}^{r} E_i\] où $E_i$ est le sous-espace propre de $\lambda_i$ et $r$ le nombre de racines deux à deux distinctes. Or la dimension de la somme est la somme des dimensions, c'est-à-dire la somme des multiplicité qui est égale à $n$. Donc $f$ est diagonalisable.
}{}

\paragraph{Exemple}On prend
\begin{align*}
A &= \matrice{0 & 1 & -1 \\ -1 & 2 & -1 \\ -1 & 1 & 0}.\\
\chi_A(t) &= \det\matrice{-t & 1 & -1 \\ -1 & 2-t & -1 \\ -1 & 1 & -t },\\
 \chi_A(t) &= -t(t-1)^{2}.
\end{align*}
Les racines sont $0,1$ de multiplicités respectives $1$ et $2$.
On a : 
\begin{align*}
E_0 = \ker A &= \left\langle \matrice{1 \\ 1 \\ 1}\right\rangle,\\
E_1 &= \ker A - \id &= \left\langle\matrice{1\\1\\0},\matrice{0\\1\\1}\right\rangle.
\end{align*}
On a \[ \dim E_0 = 1 \et \dim E_1 = 2\] et donc $f$ est diagonalisable.

\paragraph{Contre-exemple de minimalité}On a que \[A = \matrice{0 & 1 & 0 \\ 0 & 0 & 1 \\ 0 & 0 & 0} \]n'a comme valeurs propres que $0$, elle n'est pas diagonalisable parce que si elle est nulle dans une base elle l'est dans toutes.

\section{Polynômes en un endomorphisme de $E$}

\subsection{Polynômes évalué en un endomorphisme}

\definition{ 
Soit $P\in \R[t]$ un polynôme : \[ P(t) = \sum_{k=0}^{d}a_kt^{k}.\]On note pour $f$ un endomorphisme de $E$ : \[ P(f) = \sum_{k=0}^{d}a_kf^{k}\in {\rm End}_\R(E).\]

Avec la convention $f^{0} = \id$ et la notation $f^{k+1} = f\rond f^{k}$.
}{}
\definition{ 
On dit qu'un polynôme $P\in \R[t]$ annule $f$ si $P(f) = 0_{{\rm End}_\R}$.
}{}

\proposition{
On a que : \[ \fonc{\phi}{\R[t]}{{\rm End}_{\R}(E)}{P(t)}{P(f)}\] est un morphisme d'anneaux.

C'est-à-dire : \[\forall P,Q \in \R[t], \; \phi(P+Q) = \phi(P) + \phi(Q) \; ; \; \phi(PQ) = \phi(P)\phi(Q). \]
}{}
\paragraph{Remarque}Ainsi l'ensemble des polynômes annulateurs de $f$ est un idéal de $\R[t]$. Or $\R[t]$ est un anneau principal donc l'ensemble des polynômes annulateurs de $f$ est principal. Il existe donc un polynôme $Q\in \R[t]$ tel que tout polynôme annulateur de $f$ s'écrit $RQ$ avec $R\in \R[t]$.

\definition{ 
On appelle polynôme minimal de $f$ le polynôme unitaire de plus petit degré, $m_f$ annulant $f$.
}{}
On a évidemment que tout polynôme annulateur de $f$ est de la forme $P\cdot m_f, P \in \R[t]$.

\paragraph{Exemple}Avec \[A = \matrice{0 & 1 & 0 \\ 0 & 0 & 1 \\ 0 & 0 & 0} \] qui est une matrice nilpotente, c'est-à-dire $A^{3} = 0$. On a \[m_A(t) \mid t^{3} \implique m_A = 1,t,t^{2} \ou t^{3}.\] Or $(t\donne 1)(A) = \id \neq 0$, $(t\donne t)(A) = A \neq 0$ et $(t\donne t^{2})(A) = A^{2} = \matrice{0 & 0 & 1 \\ 0 & 0 & 0 \\ 0 & 0 & 0 }\neq 0$ et donc $m_A(t) = t^{3}$. 

\proposition{ 
Soit $f\in {\rm End}_\R(E)$. Alors :
\begin{enumerate}
\item si $f$ est diagonalisable, alors il existe un polynôme scindé $P\in \R[t]$ annulant $f$ ayant que des racines simples ;
\item si $P \in \R[t]$ annule $f$ alors toute valeur propre de $f$ est racine de $P$.
\end{enumerate}
}{}
\demonstration{ 
Dans l'ordre :
\begin{enumerate}
\item Soit $B = (e_1,\ldots,e_n)$ une base de vecteurs propres. Soient $\mu_1,\ldots,\mu_r$ des scalaires deux à deux distinctes tels que \[ \ens{\mu_1,\mu_2,\ldots,\mu_r} = \ens{\lambda_1,\lambda_2,\ldots,\lambda_n}\]avec $r\leq n$.

On pose : \[ P(t) = \prod_{i=1}^{r}(t-\mu_i).\] On cherche à savoir si $P(f) = 0$.
\begin{align*}
P(f) = 0 &\ssi P(f)(e_j) = 0, \; \forall j, \\
P(f)(e_j) &= \left( \prod_{i=1}^{r}(f-\mu_i \id) \right)(e_j), \\
f(e_j) = \lambda_j e_j & \implique \exists i, \mu_i = \lambda_i.
\end{align*}
Or pour tous $k,l$ : \[ (f-\mu_k \id)(f - \mu_l \id) =(f - \mu_l \id)(f - \mu_ k\id)    \]et donc :
\begin{align*}
P(f)(e_j) &= \left( \prod_{k\neq i} (f-\mu_k \id)\right)(f-\mu_i \id)(e_j), \\
P(f)(e_j) &= \left( \prod_{k\neq i} (f-\mu_k \id)\right)(f(e_j) - \mu_i e_j) = 0.
\end{align*} 
\item On suppose que $P(f) = 0$ et $\chi_f(\lambda) = 0$ avec $P \in \R[t]$ et $\lambda\in \R$. 

Soit $v\in \ker(f-\lambda \id), v\neq 0$, alors :
\begin{align*}
P(f)(v) &= \sum_{k=1}^{d}a_{k}f^{k}(v), \\
P(f)(v) &= \sum_{k=1}^{d}a_k\lambda^{k}v.
\end{align*}
Donc $P(\lambda)\cdot v = 0$ et comme $v\neq 0$ : $P(\lambda) = 0$.
\end{enumerate}
}{}

\subsection{Lemme des noyaux}

\proposition{ 
Soit $f\in {\rm End}_\R(E)$.
\begin{enumerate}
\item Soit $P \in \R[t]$ de la forme $P = ST$ avec $S,T \in \R[t]$ avec $S$ et $T$ premiers entre eux.

Alors si $P(f) = 0$ alors $$E = \ker(S(f))\oplus \ker(T(f)).$$
\item Soit $P\in \R[t]$, $P = P_1P_2\ldots P_k$ avec $P_i \in \R[t]$ premiers entre eux deux à deux.

Alors si $P(f) = 0$ alors $$E = \bigoplus_{i=1}^{k}\ker P_i(f).$$
\end{enumerate}
}{Théorème des noyaux}

\theoreme{ 
$f \in {\rm End}_\R(E)$ avec $\dim E = n$.

Supposons qu'il existe $P\in \R[X]$ est un polynôme scindé avec des racines simples. Alors $P(f) = 0$ implique que $f$ est diagonalisable.
}{}
\paragraph{Remarque}C'est équivalent à $m_f(t)$ scindé avec des racines simples. En effet si $P$ est scindé avec des racines simples et qui annulent $f$ alors $m_f$ divise $P$ et donc $m_f$ est scindé avec des racines simples.

\demonstration{ 
Soit : \[ P(X) = (X-\lambda_1)\ldots(X - \lambda_k)\]
avec $\lambda_1,\lambda_2,\ldots,\lambda_k$ réels distincts. Ainsi $X-\lambda_i$ et $X-\lambda_j$ sont premiers entre eux pour tous $i\neq j$.

Ainsi d'après le théorème des noyaux : \[ E = \bigoplus_{i=1}^{k}\ker(f-\lambda_i \id). \] Donc $f$ est diagonalisable.
}{}

\corollaire{ 
Soit $f\in {\rm End}(E)$. $f$ est diagonalisable si, et seulement si, son polynôme minimal $m_f$ est scindé avec des racines simples.
}{}
\demonstration{ 
Le sens d'implication a déjà été fait, l'autre sens est donné par le théorème précédent.
}{}

\subsection{Trigonalisation}
\definition{ 
On dit que $f\in {\rm End}(E)$ est \textit{trigonalisable} s'il existe une base $B$ de $E$ telle que la matrice en base $B$ de $f$ est triangulaire supérieure.

De même, une matrice $A\in M_n(\R)$ est trigonalisable si elle est conjuguée à une matrice triangulaire supérieure, i.e. s'il existe $P\in \GL_n(\R)$ telle que $P^{-1}AP$ est trigonalisable.
}{}

\proposition{ 
Soit $f\in {\rm End}(E)$.

 $\chi_f$ est scindé dans $\R[X]$ si, et seulement si, $f$ trigonalisable.
}{}
\paragraph{Remarque}On peut remplacer partout $\R$ par $\K = \C,\R,\Q$ et $E$ par un $\K$-espace vectoriel. Si $E$ est un $\C$-espace vectoriel de dimension finie et si $f\in {\rm End}_\C(E)$ alors la proposition assure la trigonalisation de $f$ (et ainsi de tout endomorphisme).

\demonstration{ 
Si $f$ est trigonalisable, alors il existe une base $B$ telle que la matrice, $(a_{i,j})$ de  $f$ soit trigonale supérieure dans cette base. Alors le polynôme caractéristique (qui est indépendant de la base) est exactement : $\chi_f(t) = \prod_{i=1}^{n} (a_{ii}-t)$. Ce polynôme est bien scindé. 

Pour la réciproque on effectue une récurrence sur $n=\dim E$. On suppose que c'est vrai pour tout espace vectoriel de dimension strictement inférieure à $n$ : \[ \chi_f(t) = (\lambda_1 - t)(\lambda_2 - t)\ldots(\lambda_n-t)\]avec $\lambda_1,\ldots,\lambda_n \in \R$.

$\lambda_1$ est une valeur propre.
Il existe par hypothèse $v_1\in E$ un vecteur propre tel que $v_1\neq 0$ et $f(v_1) = \lambda_1 v_1$. Par le théorème de la base incomplète, il existe une base $B$ de la forme $B = (v_1,e_2,e_3,\ldots,e_n)$. Soit $A$ la matrice de $f$ dans la base $B$. On a : \[ A =\matrice{ 
\lambda_1 & \vline & \star & \vline & \star & \vline & \ldots \\
\hline
0 &\vline \\
0 & \vline &  & & B\\
0 & \vline
}\]
Avec $B\in M_{n-1}(\R)$ qui peut être la matrice d'un endomorphisme de $\R^{n-1}$.
\begin{align*}
\chi_A(t) &= \det \matrice{\lambda_1 - t &\vline & \star \\ \hline 0 & \vline & B - t I_d}, \\
\chi_A(t) &= (\lambda_1 - t)\chi_B(t), \\
\chi_A(t) &= (\lambda_1 - t)(\lambda_2 - t)\ldots (\lambda_n - t).
\end{align*}
et donc $\chi_B(t) = (\lambda_2 - t)\ldots(\lambda_n -t)$ est scindé.

Par récurrence, il existe $Q\in \GL_{n-1}(\R)$ tel que $Q^{-1}BQ$ soit triangulaire supérieure. Posons : 
\[ P = \matrice{ 1 & \vline & 0 \\ \hline 0 & \vline & Q}, \;  P^{-1} = \matrice{ 1 & \vline & 0 \\ \hline 0 & \vline & Q^{-1}}.\]
On a alors que $P^{-1}AP$ est triangulaire.
}{}

\subsection{Comment calculer $m_f$ ? (\textsc{Cayley-Hamilton})}
\theoreme{ 
Soit $f \in {\rm End}_\R(E)$. On a que $m_f$ divise $\chi_f$, c'est-à-dire : $\chi_f(f) = 0$.
}{\textsc{Cayley-Hamilton}}
\demonstration{ 
On veut montrer que $\chi_A(A) = 0$ où $A\in M_n(\R)$. Puisque $M_n(\R)\dans M_n(\C)$ on peut se placer dans se dernier.

On sait alors que $A$ est trigonalisable dans $M_n(\C)$, c'est-à-dire qu'il existe $P \in \GL_n(\C)$ tel que $P^{-1}AP$ est triangulaire supérieure.

Or pour tout $k$ : $(P^{-1}AP)^{k} = P^{-1}A^{k}P$. Donc : \[\chi_A(P^{-1}AP) = P^{-1}\chi_A(A) P.\]Comme $P$ est inversible, $\chi_A(0)$ si, et seulement si, $\chi_A(P^{-1}AP) = 0$. Posons $A' = P^{-1}AP$. On a $\chi_{A'} = \chi_A$.
\begin{align*}
T &= (\lambda_n I_d - A')(\lambda_{n-1} I_d - A')\ldots(\lambda_1 I_d - A') \\
T(v_1) &= \left( \prod_{i=2}^{n} (\lambda_i I_d - A') \right)(\lambda_1 I_d - A')(v_1) = 0 \\
T(v_2) &= \left( \prod_{i=3}^{n} (\lambda_i I_d - A') \right)(\lambda_2 I_d - A')(\lambda_1 I_d - A')(v_2) \\
(\lambda_2 I_d - A')(\lambda_1 I_d - A')(v_2) &= (\lambda_1 I_d -A')(\lambda_2 I_d - A')(v_2)\\
(\lambda_1 I_d -A')(\lambda_2 I_d - A')(v_2) &= (\lambda_1 I_d - A')(-a_{1,2}'v_1) \\
(\lambda_1 I_d -A')(\lambda_2 I_d - A')(v_2) &= -a_{1,2}'(\lambda_1 I_d - A')(v_1) \\
(\lambda_1 I_d -A')(\lambda_2 I_d - A')(v_2) &= -a_{1,2}'(\lambda_1 v_1 - \lambda_1 v_1) = 0
\end{align*}
Par récurrence on trouve $T(v_i) = 0$ pour tout $i$.
}{}
\paragraph{Exercice}Calculer $T(v_3)$.

\paragraph{Remarque}À noter :
\begin{enumerate}
\item \'Etant donné $f\in {\rm End}(E)$, pour calculer $m_f$ on cherche le plus petit diviseur de $\chi_f$ qui annule $f$.
\item Soit $f\in {\rm End}(E)$. Supposons que $f$ est inversible, alors $\det(f)\neq 0$, i.e. $\chi_f(0) \neq 0$. Soit $\chi_f(t) = (-1)^{n}t^{n} + a_{n-1}t^{n-1}+\ldots + a_1t + a_0$, $a_0$ est donc non nul. On a :
\[ 0 = a_0^{-1}\chi_f(f) = (-1)^{n}a_0^{-1}f^{n} + \ldots +a_1a_0^{-1}f + I_d\]ce qui donne : \[I_d = f\left( (-1)^{n+1}a_0^{-1}f^{n-1} + \ldots + (-1)a_1a_0^{-1}I_d  \right).\]
\end{enumerate}

\paragraph{Exemple}Soit : \[ A = \matrice{0 & 1 & -1 \\ -1 & 2 & -1 \\ -1 & 1 & 0}.\]On a : \[ \chi_A(t) = -t(t-1)^{2}\] on en déduit : \[ t(t-1) \mid m_A(t) \mid t(t-1)^{2}.\] Donc soit $m_A(t) = t(t-1)$ soit $m_A(t) = t(t-1)^{2}$. Dans le premier cas si $m_A(A) = 0$ alors $A$ est diagonalisable. Dans le second, $A$ est non diagonalisable.
\[ A(A-I_d) = \matrice{0 & 1 & -1 \\ -1 & 2 & -1 \\ -1 & 1&0}\matrice{-1 & 1 & -1 \\ -1 & 1 & -1 \\ -1 & 1 & -1} = \matrice{0 & 0 & 0\\0 & 0 & 0\\0 & 0 & 0}.\]

\section{Applications}
\subsection{Calculs de puissances}
Soit $A\in M_n(\R)$, si $A$ est diagonalisable alors : \[ A = P A'P^{-1}\]où $P$ est inversible et $A'$ diagonale. Et donc pour tout $k$ : \[ A^{k} = P \matrice{\lambda_1^{k} \\ & \lambda_2^{k} \\
 & &\ddots \\ && & \lambda_n^{k}}P^{-1}.\]

De même, si \[ \exp(A) = \sum_{k=0}^{\infty} \frac{1}{k!}A^{k}\] alors \[ \exp(A) = P \matrice{e^{\lambda_1} \\ & e{\lambda_2} \\ && \ddots \\ &&& e{\lambda_n}}P^{-1}.\]

\subsection{Systèmes différentiels}
Soient $x_1,x_2,x_3 : \R \vers \R$ et le système différentiel : \[ \systeme{ 
x_1' &= x_1 + 2x_2 - 3x_3 \\
x_2' &= x_1 + 4x_2 - 5x_3 \\
x_3' &= 2x_2 - 2x_3
}.\] 
On pose $X = \matrice{x_1\\x_2\\x_3} : \R \vers \R^{3}$ et $A = \matrice{ 1  & 2 & -3 \\ 1 & 4 & -5 \\ 0 & 2 & -2}$. On a : \[ X' = AX. \]
$A$ a pour vecteurs propres : \[ v_1 = \matrice{1 \\ 1 \\ 1}, \; v_2 = \matrice{1 \\ 3 \\ 2}, \; v_3 = \matrice{1 \\ 2 \\1}\] de valeurs propres respectives : \[\lambda_1 = 0, \; \lambda_2 = 1 , \; \lambda_3 = 2. \]De matrice de passage : \[ P = \matrice{1 & 1 & 1 \\ 1 & 3 & 2 \\ 1 & 2 & 1}. \] \[A' = P^{-1}AP = \matrice{0 & 0 & 0 \\ 0 & 1 & 0 \\ 0 & 0 & 2}. \] On pose $Y = P^{-1}X = \matrice{y_1 \\ y_2 \\ y_3}.$ Ainsi :\[ X' = AX \ssi P^{-1}X' = P^{-1}APP^{-1}X \ssi Y' = BY\] \[ Y' = BY \ssi \systeme{y_1' &= 0 \\ y_2' &= y_2 \\ y_3' &= 2y_3} \ssi \systeme{y_1 &= c_1 \\ y_2 &= c_2 e^{t} \\y_3 &= c_3 e^{2t} }, \; c_1,c_2,c_3 \in \R.\]On a alors : 
\begin{align*}
X &= PY, \\
\matrice{x_1 \\ x_2 \\ x_3} &= X = PY = \matrice{1 & 1 & 1 \\ 1 & 3 & 2 \\ 1 & 2 & 1}\matrice{c_1 \\ c_2 e^{t} \\ c_3 e^{2t}}, \\
\matrice{x_1 \\ x_2 \\ x_3} &= \matrice{c_1 + c_2e^{t} + c_3e^{2t} \\ c_1 + 3c_2e^{t} + 2c_3e^{2t} \\ c_1 + 2c_2e^{t} + c_3 e^{2t}}.
\end{align*}

\subsection{Application aux suites récurrentes}Soit $(u_n)_{n\in \N}$ une suite de réels telle que \[ \forall n\in \N, \; u_{n+2} = u_{n+1} + u_n. \]
On introduit une seconde suite $v_n$ telle que $v_n = u_{n+1}$ pour tout $n$.
La relation de récurrence s'écrit alors : 
\[ \systeme{ u_{n+1} &= v_n \\ v_{n+1} = u_n + v_n }\]si on pose $X_n = \matrice{u_n \\ v_n}$ et $A = \matrice{0 & 1 \\ 1 & 1}$ on a alors que la relation de récurrence est \[X_{n+1} = A X_n.\]
On diagonalise $A$ : \[ \chi_A(t) = t^2 - t -1 \ssi  t \in \ens{r_1 = \frac{1+\sqrt{5}}{2},r_2=\frac{1-\sqrt{5}}{2}}.\]On a : \[ A' = P^{-1}AP = \matrice{r_1 & 0 \\ 0 & r_2}\] avec $P = \matrice{1 & 1 \\ r_1 & r_2}$.
On pose $Y_n = P_1X_n = \matrice{a_n \\b_n}$ : \[ X_{n+1} = AX_n \ssi Y_{n+1} = A' Y_n.\]
On en déduit : \[ Y_n = \matrice{c_1 r_1^{n} \\ c_2 r_2^{n}} \implique \matrice{u_n \\ v_n} = P\matrice{c_1 r_1^{n} \\ c_2 r_2^{n}} = \matrice{1 & 1 \\ r_1 & r_2}\matrice{c_1 r_1^{n} \\ c_2 r_2^{n}}  = \matrice{c_1r_1^{n} + c_2r_2^{n} \\ c_1r_1^{n+1} + c_2r_2^{n+1}}.\]Ainsi $u_n$ est de la forme : \[ u_n = c_1r_1^{n} + c_2r_2^{n}, \; c_1,c_2 \in \R.\]

\newpage
\part{Analyse}
\parttoc
\chapter{Développements limités}
\section{Fonctions négligeables et équivalentes}

On considère des fonctions $f,g$ de $V$ dans $\R$ où $V$ est un voisinage épointé dans $\barre{\R} = \R \union \ens{\infty}$. C'est-à-dire que $V$ est de la forme $U - \ens{a}$ où $U$ est un voisinage de $a$ dans $\barre{\R}$ et $a\in \iR$.
\begin{itemize}
\item si $a = \infty$ alors $V \contient \ens{k,\infty}$ ;
\item si $a\in \R$ alors $V \contient ]k,a[ \union ]a,l[$ avec $k < a < l$ et $k,l\in \R$.
\end{itemize}

$f,g$ sont définies au voisinage de $a\in \iR$.

\subsection{Négligeable}

\definition{ 
On dit que $f$ est \textit{négligeable} devant $g$ au voisinage de $a$ s'il existe un voisinage $V$ tel qu'il existe une fonction $\eps : V \vers \R$ telle que : 
\begin{itemize}
\item $f = \eps \cdot g$  ;
\item $\lim_a \eps = 0$.
\end{itemize}
On note $f \underset{(a)}{=}\oo(g)$.
}{}
\paragraph{Remarque}On note :\[\fonc{\eps f}{V}{\R}{t}{\eps(t)f(t)}. \]
\paragraph{Exemples}Par exemple :
\begin{enumerate}
\item Si $g=1$ alors $f = \oo(1)$ si, et seulement si, $\lim_a f =0$.
\item Si $f=0$ au voisinage de $a$ alors pour toute fonction $g$ : $f = \oo(g)$.
\item Si $f$ est bornée et $\lim_a(g) = \infty$ alors $f = \oo(g)$ (on prend alors $\eps = f/g$).
\item On a $x^m \underset{(\infty)}{=}\oo(x^n)$ si, et seulement si, $m < n$.
\item Pour tous $\alpha,\beta >0$ : \[\systeme{ x^{\alpha} &\underset{\infty}{=} \oo(e^{\beta x}) \\ (\ln x)^{\alpha} &\underset{(\infty)}{=} \oo(x^{\beta})},\]car $\lim_\infty x^{\alpha}e^{-\beta x} = 0$.
\end{enumerate}

\proposition{ 
Si $f/g$ est définie dans un voisinage de $a$, alors : \[ f\underset{(a)}{=}\oo(g) \ssi \lim_a (f/g) = 0.\]
}{}
\demonstration{ 
On prend $\eps = f/g$.
}{}
\paragraph{Remarque}Il peut arriver que $f/g$ n'est pas défini dans aucun voisinage de $a$.
\paragraph{Exemples}Contre-exemples :
\begin{enumerate}
\item Avec $g(t) = \sin(1/[t-a])$, pour tout voisinage de $V$ de $a$, $g(t)$ s'annule en un point de $V$.
\item Même si le quotient n'est pas définit : $t \underset{(0)}{=} \oo(\sin(1/t))$.
\end{enumerate}

\proposition{ 
On a au voisinage de $a$ :
\begin{enumerate}
\item la propriété $\oo$ est transitive ;
\item la propriété $\oo$ est compatible avec la multiplication, i.e. : si $f =\oo(g)$ alors $fh = \oo(gh)$ ;
\item si $f = \oo(g)$ et si $h = \oo(k)$ alors $fh = \oo(gk)$.
\end{enumerate}
}{}
\demonstration{ 
Dans l'ordre :
\begin{enumerate}
\item Pour $f = \eps_1 g$ et $g = \eps_2 h$ avec $\lim_a \eps_i = 0$ alors : $f = \eps_1\eps_2 h$ et $\lim_a \eps_1\eps_2 = 0$.
\item Si $f = \eps g$, $\lim_a \eps =0$, alors $fh = \eps gh$.
\item De même.
\end{enumerate}
}{}
\paragraph{Contre-exemple}$\oo$ n'est pas compatible avec l'addition. Par exemple : $x \underset{(\infty)}{=} \oo(x^{3})$ et $x^2 \underset{(\infty)}{=} \oo(-x^3)$ n'entraine pas $x+x^2 \underset{(\infty)}{=} \oo(0)$.

\subsection{\'Equivalence}

\definition{ 
On dit que $f$ est \textit{équivalence} à $g$ au voisinage de $a$ si : $f-g \underset{(a)}{=} \oo(g)$. On note $f\underset{(a)}{\sim} g$.
}{}

\proposition{ 
Si $f/g$ est définie dans un voisinage de $a$ alors : \[f \underset{(a)}{\sim} g \ssi \lim_a f/g = 1.\]
}{}
\proposition{ 
$\underset{(a)}{\sim}$ est une relation d'équivalence.
}{}
\demonstration{ 
Par définition :
\begin{enumerate}
\item elle est réflexive : $f \underset{(a)}{\sim} f$ puisque $0 \underset{(a)}{=}\oo(f)$ ;
\item elle est symétrique si $f\underset{(a)}{\sim}g$ alors il existe $\eps$ telle que $\lim_a \eps = 0$ et $f = (1+\eps)g$, or $1/(1+\eps)$ est aussi définie au voisinage de $a$ et puisque $g = (1/[1+\eps])f$ on a \[g = (1+(1/[1+\eps] -1))f \] or en posant $\eps' = [1+\eps] -1$ on a $\lim_a \eps' = 0$ ;
\item elle est transitive : $f \underset{(a)}{\sim} g \et g \underset{(a)}{\sim} h $ implique qu'il existe $\eps_1,\eps_2$ telles que $f = (1+\eps_1)g$, $g = (1+\eps_2)h$ et donc $f = (1+\eps)h$ avec $\eps = \eps_1 + \eps_2 + \eps_1\eps_2$ et $\lim_a \eps = 0$.
\end{enumerate}
}{}

\proposition{
Si $f \underset{(a)}{\sim} g$ et si $\lim_a f$ existe alors $\lim_a g$ existe et $\lim_a g = \lim_a f$.
}{}
\demonstration{ 
Soit $\eps$ telle que $\lim_a \eps = 0$ alors puisque $f = (1+\eps)g$ on a \[\lim_a f = \lim_a(1+\eps)g = \lim_a g. \]
}{}

\proposition{ 
Le produit et le quotient (quand il est défini) d'équivalences est une équivalence.

Une puissance entière d'équivalences est une équivalence.
}{}
\demonstration{ 
Si $f =(1+\eps_1)g $et $h = (1+\eps_2)k$ alors $fh = (1+\eps)gk$ avec $\eps = \eps_1+\eps_2 +\eps_1\eps_2$.
}{}

\proposition{ 
Si $f\underset{(a)}{\sim}g$ et si $\varphi : I \vers \R$ telle que $\lim_b \varphi = a$, $b\in I$. Alors \[ f\rond \varphi \underset{(a)}{\sim} g \rond \varphi.\]
}{}
\demonstration{ 
Si $f = (1+\eps)g$ avec $\lim_a \eps = 0$. Alors \[ f\rond \varphi =(1+\eps') \cdot g\rond \varphi\]avec $\eps' = \eps\rond\varphi$ et $\lim_a \eps' = 0$.
}{}

\proposition{ 
On a :
\begin{enumerate}
\item Si $f$ est dérivable en $a$ alors si $f'(a) \neq 0$ on a $f(x) -f(a) \sim f'(a)(x-a)$.
\item Si $g$ est continue dans un voisinage épointé de $a$, alors si $f\underset{(a)}{\sim}g>0$ alors \[\int_{a}^{x}f(t)\dt \underset{(a)}{\sim} \int_{a}^{x}g(t)\dt. \]
\end{enumerate}
}{}
\demonstration{ 
Dans l'ordre :
\begin{enumerate}
\item Si $f$ est dérivable en $a$ alors : \[ \frac{f(x) - f(a)}{x-a} \underset{(a)}{\sim} f'(a)\]puisque si $\lim_a g =b\in \R^{*}$ alors $g \underset{(a)}{\sim}b$.
\item On sait que $f-g \underset{(a)}{=} \oo(g)$ et on veut : \[ \int_{x}^{a}(f-g)(t)\dt \underset{(a)}{=} \oo \left( \int_{x}^{a}g(t)\dt \right).\]

En posant $h = f-g$ on se ramène au problème : \[ h = \oo(g) \implique \int_{a}^{x}h = \oo \int_{a}^{x}g.\]
Si $h= \eps g$ et $\lim_a\eps = 0$ alors 
\begin{align*}
\int_{a}^{x}g &= \int_{a}^{x}\eps g 
\end{align*}
Or \[ \frac{\abs{\int_{x}^{a}\eps g}}{\int_{a}^{x}g} \leq \max_{[a,x]}\abs{\eps}\frac{\int_{a}^{x}g}{\int_{a}^{x}g} \underset{x\to a}{\longrightarrow} 0.\]
Donc \[\frac{\abs{\int_{a}^{x}\eps g =h}}{\abs{\int_{a}^{x}g}} \underset{x\to a}{\longrightarrow} 0.\]
\end{enumerate}
}{}

\section{Dérivées successives et formules de \textsc{Taylor}}

Soit $p\geq 0$ un entier.
\definition{ 
Soit $I$ un intervalle de $\R$ et $f:I\vers \R$.
\begin{enumerate}
\item $f\in C^{0}$ si $f$ est continue ;
\item $f \in C^{p}$ ($p\geq 1$) si $f$ est dérivable et $f' \in C^{p-1}$.
\end{enumerate}
}{}
\paragraph{Remarque}Si $f\in C^{p}$ alors les $p$-ièmes dérivées successives et $f$ sont toutes continues sur $I$. $f\in C^{\infty}$ si $f^{(p)}$ existe et est continue pour tout $p\geq 1$.

\proposition{ 
Si $f,g \in C^{p}$ alors $f+g$, $fg$, $f/g$ et $f\rond g$ (si définie) sont $C^{p}$.
}{}
\demonstration{ 
Dans l'ordre :
\begin{enumerate}
\item $(f+g)^{(p)} = f^{(p)} + g^{(p)}$ par récurrence sur $p$ ;
\item $(fg)^{(p)} = \sum_{k=0}^{p}\kpn{k}{p}f^{(k)}g^{(p-k)}$ ;
\item par récurrence sur $p$ pour $(f\rond g)^{(p)}$ en utilisant : $(f\rond g)' = (f'\rond g)g'$.
\end{enumerate}
}{}

\paragraph{Rappels sur les primitives}Si $f:I\vers \R$ est de classe $C^1$ avec $I\dans \R$ un intervalle ouvert. Alors si $f'$ est continue $f(x) -f(a) = \int_{a}^{x}f'(t)\dt$.

\subsection{Formules de \textsc{Taylor}}

Soit $I\dans \R$ un intervalle ouvert.

\theoreme{ 
Soit $f:I\vers \R$ de classe $C^k$. Alors pour tous $a,b\in I$ on a : \[ f(b) = \sum_{i=0}^{n-1}\frac{(b-a)^{i}}{i!}f^{(i)}(a) + \int_{a}^{b}\frac{(b-t)^{n-1}}{(n-1)!}f^{(n)}(t)\dt.\]
}{Formule de \textsc{Taylor} avec reste intégral}
\demonstration{ 
Par récurrence sur $n$, on note \[ (T_n) : f(b) = \sum_{i=0}^{n-1}\frac{(b-a)^{i}}{i!}f^{(i)}(a) + \int_{a}^{b}\frac{(b-t)^{n-1}}{(n-1)!}f^{(n)}(t)\dt.\]Supposons que $(T_k)$ soit vraie pour tout $k<n$. Alors par intégration par parties : 
\begin{align*}
u(t) &= -\frac{(b-t)^{k}}{k!}, \\
v(t) &= f^{(k)}(t),\\
R_k &= \int_{a}^{b}\frac{(b-s)^{k-1}}{(k-1)!}f^{(k)}(s)\dd s,
\end{align*}
on a :
\begin{align*}
R_k &= \int_{a}^{b}u'(s)v(s)\dd s\\
R_k &= [u(s)v(s)]^{b}_{a} - \int_{a}^{b}u(s)v'(s)\dd s\\
R_k &= u(b)v(b) - u(a)v(a)  + \int_{a}^{b}\frac{(b-s)^{k}}{k!}f^{(k+1)}(s)\dd s\\
R_k &= \frac{(b-a)^{k}}{k!}f^{(k)}(a) + \int_{a}^{b}\frac{(b-s)^{k}}{k!}f^{(k+1)}(s)\dd s\\
\end{align*}
On applique $(T_{n-1})$ : 
\begin{align*} 
f(b) &=f(a) + \sum_{i=0}^{n-2}\frac{(b-a)^{i}}{i!}f^{(i)}(a) + R_{n-1}  \\
f(b) &= f(a) + \sum_{i=1}^{n-2}\frac{(b-a)^{i}}{i!} + \frac{(b-a)^{n-1}}{(n-1)!}f^{(n-1)}(a) + R_n
\end{align*}
donc $(T_n)$ vraie.
}{}

\theoreme{ 
Soit $n>0$, $f:I\vers \R$ de classe $C^{n+1}$. Pour tous $a,b\in I$ avec $a\neq b$, il existe $\theta$ strictement compris en $a$ et $b$ tel que : 
\[ f(b) = \sum_{i=0}^{n}\frac{(b-a)^{i}}{i!}f^{(i)}(a) + \frac{(b-a)^{n+1}}{(n+1)!}f^{(n+1)}(\theta). \]
}{Formule de \textsc{Taylor} avec reste en $f^{(n+1)}(\theta)$}
\demonstration{
On pose $A$ telle que \[ \frac{(b-a)^{n+1}}{(n+1)!}\cdot A = \int_{a}^{b}\frac{(b-s)^{n+1}}{(n+1)!}f^{(n)}(s)\dd s - \frac{(b-a)^{n}}{n!}f^{(n)}(a).\]
Soit $F:I\vers \R$ telle que :
\[ F(x) = \int_{x}^{b}\frac{(b-t)^{n-1}}{(n-1)!}f^{(n)}(t)\dt - \frac{(b-x)^{n}}{n!}f^{(n)}(x) - \frac{(b-x)^{n+1}}{(n+1)!}A. \]
On calcule $F'(x)$ :
\begin{align*}
F'(x) &= -\frac{(b-x)^{n-1}}{(n-1)!}f^{(n)}(x) -\frac{(b-x)^{n}}{n!}f^{(n+1)}(x) +\frac{(b-x)^{n-1}}{(n-1)!}f^{(n)}(x) + \frac{(b-x)^{n}}{n!}A \\
F'(x) &= \frac{(b-x)^{n}}{n!}\left(A-f^{(n+1)}(x)\right).
\end{align*}
$F$ est dérivable donc continue sur $I$ :
\begin{align*}
F(a) &= \int_{a}^{b}\frac{(b-t)^{n-1}}{(n-1)!}f^{(n)}(t)\dt - \frac{(b-a)^{n}}{n!}f^{(n)}(a) - \frac{(b-a)^{n+1}}{(n+1)!}A =0,\\
F(b) &= 0.
\end{align*}
Par le théorème de \textsc{Rolle}, il existe $\theta$ strictement entre $a$ et $b$ tel que $F'(\theta) = 0$. C'est-à-dire :
\begin{align*}
\frac{(b-\theta)^{n}}{n!}\left(A-f^{(n+1)}(\theta)\right) = 0 \\
A &= f^{(n+1)}(\theta).
\end{align*}
On en déduit :
\[ \frac{(b-a)^{n+1}}{(n+1)!}f^{(n+1)}(\theta) = \int_{a}^{b}\frac{(b-s)^{n-1}}{(n-1)!}f^{(n)}(s)\dd s - \frac{(b-a)^{n}}{n!}f^{(n)}(a). \]
On a alors le résultat en remplaçant dans $(T_n)$.
}{}
\paragraph{Remarque}Si $\abs{f^{(n+1)}(s)}\leq M$ pour tout $s\in I$ alors \[\abs{f(b) - \sum_{i=0}^{n}\frac{(b-a)^{i}}{i!}f^{(i)}(a) }\leq M \frac{\abs{b-a}^{n+1}}{(n+1)!}. \]

\subsection{Fonctions usuelles}
\proposition{ 
Soit $n\in \N$, on regarde le développement de \textsc{Taylor} en $0$ à l'ordre $n+1$, $\forall i, \; \exp^{(i)}(0) = 1$. On prend $b=x,a=0$ :
\begin{align*}
\exp(x) &= \sum_{i=0}^{n}\frac{x^{n}}{n!}+\frac{x^{n+1}}{(n+1)!}\exp(\theta)\\
\theta &\in ]0,x[.
\end{align*}
}{Exponentielle}
\proposition{ 
La dérivée $n$-ième de $\cos(t)$ est $\cos(t+n\pi/2)$.
\[\abs{\cos(x) -\sum_{i=0}^{n}(-1)^{i+1}\frac{x^{2i}}{(2i)!}}\leq \frac{x^{2n+2}}{(2n+2)!}\]
car $\abs{\cos\theta}\leq 1$.
}{Cosinus, sinus}

\section{Développement limité à l'ordre $n$ d'une fonction de classe $C^n$}
\subsection{Développements limités}
\definition{ 
Soit $I\dans \R$ un intervalle ouvert tel que $0\in I, n\in \N$. On dit qu'une fonction $f : I\vers \R$ admet un \textit{développement limité} à l'ordre $n$ en $0$ si, et seulement s'il existe un polynôme $P$ de degré $n$ à coefficients réels tel que 
\[\lim_{x\to 0} \frac{f(x)-P(x)}{x^{n}} = 0. \]
Notons $$\eps(x) = \frac{f(x) - P(x)}{x^{n}}$$ alors \[\systeme{ f(x) &= P(x) + x^{n}\eps(x)\footnotemark, \\ \lim_{x\to 0}\eps(x) &= 0. } \]
}{}
\footnotetext{C'est-à-dire, $f(x) - P(x) = \oo(x^{n})$.}

\definition{ 
Soit $I\dans \R$ un intervalle ouvert et soit $n\in \N$. On dit qu'une fonction $f : I\vers \R$ admet un \textit{développement limité} à l'ordre $n$ en $a$ si, et seulement si, la fonction $t\donne f(t+a)$ admet un développement limité à l'ordre $n$ en $0$. C'est-à-dire si, et seulement s'il existe un polynôme de degré $n$, $P$ à coefficients réels tel que :
\[f(x) = P(x-a)+ \oo((x-a)^{n}) \]au voisinage de $a$.
}{}

\theoreme{ 
Si $f$ admet  un développement limité à l'ordre $n$ en un point $a$, alors ce développement limité est unique.
}{}
\demonstration{ On peut supposer $a=0$.
Supposons que $$f(x) = P_1(x) + x^{n}\eps_1(x) = P_2(x) + x^{n}\eps_2(x)$$ où $\lim_0 \eps_i = 0$ pour $i\in \ens{1,2}$. On a que \[ (P_1-P_2)(x) = x^{n}(\eps_1 - \eps_2)(x)\]et $(P_1-P_2)(x)$ est de la forme $r_0 + r_1x+\ldots + r_{n}x^{n}$ avec $r_0,r_1,\ldots,r_n\in \R$.

On montre par récurrence que les $r_k$ sont tous nuls.
Quand $x\to 0$ on trouve : \[r_0= 0\] et donc \[r_1x + \ldots + r_nx^{n} = x^{n}(\eps_1-\eps_2)(x). \]

Supposons que $r_0 =r_1 = r_{k-1}=0$, $k>0$. Alors 
\begin{align*}
r_k x^{k} + \ldots + r_nx^{n} &=x^{n}(\eps_1-\eps_2)(x),\\
r_k + r_{k+1}x + \ldots + r_nx^{n-k} &= x^{n-k}(\eps_1 - \eps_2)(x),
\end{align*}
$n-k\geq 0$ et donc $r_k = 0$ en passant à la limite.
}{}
\corollaire{ 
Soit $f(x) = P(x) + x^{n}\eps(x)$ le développement limité d'une fonction $f$ à l'ordre $n$ en $0$. Alors :
\begin{enumerate}
\item si $f$ est paire alors $P$ est pair ;
\item si $f$ est impaire alors $P$ est impaire.
\end{enumerate}
}{}
\demonstration{ 
\begin{align*}
f(x) &= P(x) + x^{n}\eps(x), \\
f(-x) &= P(-x) + x^{n}(-1)^{n}\eps(-x) = P(-x) + x^{n}\eps_1(x),
\end{align*}
Or comme $\eps(x) \to 0$ quand $x\to 0$ alors $\eps_1\to 0$ aussi.
\begin{enumerate}
\item si $f$ est impaire alors on a : \[f(x) = -P(-x) - x^{n}\eps_1(x) \] et comme la première et cette expression sont des développements limits de $f$ à l'ordre $n$ en $0$, par unicité on a $-P(-x) = P(x)$, c'est-à-dire $P$ impaire ;
\item si $f$ est paire, on a : \[ f(x) = P(-x) + x^{n}\eps_1(x)\] alors de même, l'unicité nous dit que $P$ est alors paire.
\end{enumerate}
}{}

\proposition{ 
Soit $f:I\vers \R$ une fonction continue en $a\in I$.
\begin{enumerate}
\item le développement limité de $f$ en $a$ à l'ordre $0$ est \[ f(x) = f(a) + \eps(x), \; \lim_{x\to a}\eps(x) = 0 \ ; \]
\item la fonction $f$ est dérivable en $a$ si, et seulement si, elle possède un développement limité à l'ordre $1$ en $a$, alors dans ce cas le développement limité est donné par : \[ f(x) = f(a) + f'(a)(x-a) + \eps(x)(x-a), \; \lim_{x\to a}\eps(x) = 0.\]
\end{enumerate}
}{}
\demonstration{ 
Dans l'ordre :
\begin{enumerate}
\item On pose $\eps(x) = f(x) - f(a)$. Comme $f$ est continue en $0$, $\eps(x)$ aussi et $\lim_{x\to a}\eps(x) = 0$.
\item Supposons que $f$ soit dérivable en $a$, c'est-à-dire : \[ \lim_{x\to a}\frac{f(x) - f(a)}{x-a} = f'(a).\]On pose \[\eps(x) = \frac{f(x)-f(a)}{x-a}-f'(a).\]On a bien $\lim_{x\to a}\eps(x) = 0$ et \[ f(x)  = f(a) + (x-a)f'(a) + (x-a)\eps(x).\]

Réciproquement, supposons que $f$ admette un développement limité : \[ f(x) = a_0 + (x-a)a_1 + (x-a)\eps(x),\]avec $\lim_{x\to a}\eps(x) = 0$. Alors, par continuité $a_0 = f(a)$ et \[\lim_{x\to a}\frac{f(x) - f(a)}{x-a} = \lim_{x\to a}a_1 + \eps(x) = a_1 =  f'(a). \]
\end{enumerate}
}{}


\subsection{Développements limités et primitives}
\theoreme{ 
Soit $f: I \vers \R$ une application continue. Soit $F$ une primitive de $f$. Soit $a\in I$ et supposons que $f$ admette un développement limité en $a$ à l'ordre $n$ : \[ f(x) = a_0 + a_1(x-a) + \frac{a_2}{2}(x-a)^{2} + \ldots + \frac{a_n}{n!}(x-a)^{n} + (x-a)^{n}\eps(x), \; \lim_{x\to a}\eps(x) = 0. \]
Alors $F$ admet le développement limité suivant à l'ordre $n+1$ en $a$ : \[ F(x) = F(a) + a_0(x-a) + \frac{a_1}{2}(x-a)^{2} + \ldots + \frac{a_n}{(n+1)!}x^{n+1} + (x-a)^{n+1}\eps_1(x), \; \lim_{x\to a}\eps_1(x) = 0.\]
}{}
\demonstration{ 
Soit \[ P(t) = \sum_{k=0}^{n}\frac{a_k}{k!}(t-a)^{k}.\]
Pour tout $x\neq a$ : \[ \eps(x) = \frac{f(x) -P(x)}{(x-a)^{n}}.\]Par hypothèse, $\lim_{x\to a}\eps(x)= 0$. En posant $\eps(a) = 0$, on obtient que $\eps$ est continue sur $I$. Donc $\eps$ admet une primitive et dans l'identité \[ f(x) = a_0 + a_1(x-a) + \frac{a_2}{2}(x-a)^{2} + \ldots + \frac{a_n}{n!}(x-a)^{n} + (x-a)^{n}\eps(x), \; \lim_{x\to a}\eps(x) = 0 \] tous les termes admettent des primitives. Donc
\begin{align*}
F(x) - F(a) &= \int_{a}^{x}f(t)\dt \\
 F(x) - F(a) &= \int_{a}^{x}\left( \sum_{k=0}^{n}\frac{a_k}{k!}(t-a)^{k} + (t-a)^{n}\eps(t)\right)  \dt \\
 F(x) - F(a)  &= \sum_{k=0}^{n}\frac{a_k}{(k+1)!}(x-a)^{k+1} + u(x),\\
 u(x) &= \int_{a}^{x} (t-a)^{n}\eps(t)\dt.
\end{align*}
Par le théorème de \textsc{Rolle} : \[ u(x) = (x-a)(\theta - a)^{n}\eps(\theta)\]pour un $\theta$ compris entre $a$ et $x$. Donc \[ \abs{u(x)} = \abs{x-a}\abs{\theta-a}^{n}\abs{\eps(\theta)} \leq \abs{x-a}^{n+1}\abs{\eps(\theta)}\]
et $\eps(\theta)$ tend vers $0$ quand $x$ tend vers $a$ puisque $\theta$ est compris entre $a$ et $x$.
Donc : \[F(x) = \sum_{k=0}^{n}\frac{a_k}{(k+1)!}(x-a)^{k+1}+(x-a)^{n+1}\eps_1(x) \]où \[\eps_1(x) = \frac{u(x)}{(x-a)^{n+1}} \to 0. \]
}{}

\theoreme{ 
Soit $f:I\vers \R$ de classe $C^n$, $a\in I$. Alors $f$ admet pour développement limité à  l'ordre $n$ en $a$ : \[ f(x) + \sum_{k=0}^{n}\frac{f^{(k)}(a)}{k!}(x-a)^{k} + (x-a)^{n}\eps(x), \; \lim_{x\to a}\eps(x) = 0.\]
}{}
\demonstration{ 
Pour $n=0,1$ ça a été déjà vu. Supposons alors $n\geq 2$. Soit $f\in C^n$, posons $g=f'$ avec $g\in C^{n-1}(I)$.

Par récurrence : \[ g(x) = \sum_{k=0}^{n-1}\frac{g^{(k)}(a)}{k!} (x-a)^{k} + (x-a)^{n-1}\eps(x), \; \lim_{x\to a}\eps(x) = 0.\]
$f$ est une primitive de $g$ : 
\begin{align*}
f(x) &= f(a) + \sum_{k=0}^{n-1}\frac{g^{(k)}(a)}{(k+1)!}(x-a)^{k+1} + (x-a)^{n}\eps_1(x), \; \lim_{x\to a}\eps_1(x) = 0 \\
f(x) &= f(a) + \sum_{k=0}^{n-1}\frac{f^{(k+1)}(a)}{(k+1)!}(x-a)^{k+1} + (x-a)^{n}\eps_1(x) \\
f(x) &= f(a) + \sum_{k=1}^{n}\frac{f^{(k)}(a)}{k!}(x-a)^{k} + (x-a)^{n}\eps_1(x).
\end{align*}
}{}
\paragraph{Exemple}Soit : \[ f(x) = \systeme{\exp(-1/x^2), &\; \text{si} \ x>0 \\ 0,, &\; \text{si}\ x\leq 0 }\]son développement limité en $0$ d'ordre $n$ est : \[ f(x) =x^{n}\eps(x),\; \lim_{x\to 0}\eps(x) = 0.\]

\subsection{Développement limités usuels}
Développements limités en $0$ :
\begin{align*}
\exp(x) = &\sum_{i=0}^{n}\frac{x^{i}}{i!} + x^{n}\eps(x) \\
\ch(x) = &\sum_{i=0}^{n}\frac{x^{2i}}{(2i)!}+ x^{2n+1}\eps(x) \\
\sh(x) = &\sum_{i=0}^{n}\frac{x^{2i+1}}{(2i+1)!} + x^{2n+2}\eps(x) \\
\cos(x) =& \sum_{i=0}^{n}(-1)^{i}\frac{x^{2i}}{(2i)!} + x^{2n+1}\eps(x) \\
\sin(x) =& \sum_{i=0}^{n}(-1)^{i} \frac{x^{2i+1}}{(2i+1)!} + x^{2n+2}\eps(x) \\
\alpha\in \R : \;(1+x)^{\alpha} =& 1 +\sum_{i=0}^{n}\frac{\alpha(\alpha-1)\ldots(\alpha-i)}{(i+1)!}x^{i+1} + x^{n+1}\eps(x) \\
\frac{1}{1-x} =& \sum_{i=0}^{n}x^{i} + x^{n+1}\eps(x) \\
\frac{1}{1+x} =& \sum_{i=0}^{n}(-1)^{i}x^{i} + x^{n+1}\eps(x) \\
\log(1-x) =& -\sum_{i=1}^{n}\frac{x^{i}}{i!}+x^{n}\eps(x) \\
\log(1+x) =& \sum_{i=1}^{n}(-1)^{i+1}\frac{x^{i}}{i}x^{n}\eps(x)\\
\Arctan(x) =& \sum_{i=1}^{n}(-1)^{i+1}\frac{x^{2i-1}}{2i-1} +x^{2n}\eps(x)
\end{align*}

\demonstration{ 
\begin{align*}
\ch(x) &= \frac{e^{x} + e^{-x}}{2} \\
\ch'(x) &= \frac{e^{x} - e^{-x}}{2} (=\sh(x))\\
\ch''(x) &= \ch(x)\\
\ch^{(2i)}(0) &= 1 \\
\sh^{(2i)}(0) &= 0
\end{align*}
}{$\ch$}

\demonstration{ 
\begin{align*}
\cos^{(k)}(x) &= \cos(x+k\pi/2) \\
\cos^{(k)}(0) &= \cos(k\pi/2) \\
\cos^{(2k)}(0) &= (-1)^{k} \\
\cos^{(2k+1)}(0) &= 0
\end{align*}
}{$\cos$}

\demonstration{ 
\begin{align*}
\sin^{(k)}(x) &= \sin(x+k\pi/2) \\
\sin^{(2k)}(0) &= 0 \\
\sin^{(2k+1)}(0) &= (-1)^{k}
\end{align*}
}{$\sin$}

\demonstration{ Par récurrence :
\begin{align*}
f^{(k)}(x) &= \alpha(\alpha-1)\ldots(\alpha-k+1)(1+x)^{\alpha-k} \\
f^{(k)}(0) &= \alpha(\alpha-1)\ldots(\alpha-k+1)
\end{align*}
}{$(1+x)^{\alpha}= f(x)$}

\demonstration{ 
\begin{align*}
\frac{1-x^{n}}{1-x} &= 1+x+x^{2}+\ldots +x^n \\
\frac{1}{1-x} &= 1+x+\ldots+x^{n} + x^{n}\cdot\frac{x}{1-x}
\end{align*}
}{$1/1-x$}

\demonstration{ 
Utiliser le théorème sur le développement limité d'une primitive avec le développement limité de $1/1-x$.
}{$\log(1-x)$}

\demonstration{ 
\begin{align*}
\Arctan'(x) &= \frac{1}{1+x^{2}} \\
\frac{1}{1+x^{2}} &= \sum_{i=1}^{n}(-1)^{i}x^{2i}+x^{2n}\eps(x)
\end{align*}
et on conclut avec le théorème du développement limité d'une primitive.
}{$\Arctan(x)$}

\paragraph{Remarque}On a vu que si \[ f(x)  = \systeme{\exp(-1/x^{2}),\; &\text{si}\ x >0 \\ 0, \; &\text{si}\ x\leq 0}\] alors le développement limité de $f(x)$ en $0$ à l'ordre $n$ est \[f(x) = x^{n}\eps(x). \]Or le développement limité de $0$ en $0$ à l'ordre $n$ est identique.
\paragraph{Exemple}Soit : \[\fonc{f}{\R}{\R}{x}{\systeme{0 &\ \text{si} \ x = 0 \\ x^{3}\sin(1/x) & \ \text{si} \ x\neq 0}}. \]
La fonction $f$ est continue en $0$.

 On regarde le développement limité à l'ordre $2$ en $0$ : \[ f(x) = x^{2}\eps(x), \; \eps(x) = \systeme{0 &\ \text{si} \ x = 0 \\ x\sin(1/x) &\ \text{sinon}}, \lim_{x\to 0}\eps(x)  0.\]Donc le développement limité de $f(x)$ en $0$ à l'ordre $2$ est : \[ f(x) = x^{2}\eps(x).\]Dérivabilité de $f$ en $0$ (puisqu'elle est lisse sur $\R^*$) : \[  \frac{f(x) - f(0)}{x-0} = x^{2}\sin(1/x) \underset{x\to 0}{\longrightarrow} 0\]donc $f$ est dérivable et $f'(0) = 0$. \[ \frac{f'(x) - f'(0)}{x-0} = \frac{3x^{2}\sin(1/x) -x\cos(1/x)}{x} = 3x\sin(1/x) - \cos(1/x) \]donc $f$ n'est pas dérivable à l'ordre $2$ en $0$ (même si elle a un développement limité à l'ordre $2$).
 \section{Calculs avec les développements limités}
\subsection{Règles de calcul des développements limités}
 
\proposition{ 
Soit $f,g$ ayant des développements limités à l'ordre $n$ en $0$ : 
\[f(x) = P(x) + x^{n}\eps(x), \; g(x) = Q(x) + x^{n}\eps(x)\]avec $P,Q$ des polynômes de degré au plus $n$ et $\lim_{x\to 0}\eps(x) = 0$ (non forcément identiques). Alors
\begin{enumerate}
\item le développement limité à l'ordre $n$ en $0$ de $f+g$ est \[ (f+g)(x) = (P+Q)(x) + x^{n}\eps(x) ;\]
\item pour tout $\lambda\in \R$, le développement $\lambda f$ à l'ordre $n$ en $0$ est : \[ (\lambda f)(x) = \lambda P(x) + x^{n}\eps(x). \]
\end{enumerate}
}{} 
\demonstration{ 
\'Ecrivons $f(x) =P(x) + x^{n}\eps_f(x)$ et $g(x) = Q(x) + x^{n}\eps_g(x)$.
\begin{enumerate}
\item $(f+g)(x) = P(x)+Q(x) + x^{n}(\eps_f + \eps_g)(x)$ et on note $\eps = \eps_f + \eps_g$ qui tend bien en $0$.
\item De même.
\end{enumerate}
}{}
 
\proposition{ 
Soit $f$ qui admet le développement limité en $0$ à l'ordre $n$ : \[ f(x) = P(x) + x^{n}\eps(x), \; \lim_{x\to 0}\eps(x) = 0.\]Alors pour tout $p\in \ens{0,\ldots,n}$, $f$ admet le développement limité en $0$ à l'ordre $p$ : \[ f(x) = T_p(P)(x) + x^{p}\eps(x)\]avec $T_p(P)$ le polynôme tronqué de $P$ : \[ T_p(P) = \sum_{k=0}^{p}a_kx^{k}, \; P = \sum_{k=0}^{n}a_kx^{k}.\]
}{} 
\demonstration{ 
On a \[ f(x) = T_p(P)(x) + x^{p}\left(\sum_{k=p+1}^{n}a_kx^{k-p} + x^{n-p}\eps(x)\right).\]Et on pose \[ \eps_1(x) = \sum_{k=p+1}^{n}a_kx^{k-p}+x^{n-p}\eps(x).\]On a bien$ \eps_1(x)\to 0$ quand $x\to 0$.
}{}

\proposition{ 
Soient $f,g$ admettant les développements limités : \[ f(x) = P(x) + x^{n}\eps_1(x), \; g(x) = Q(x) + x^{n}\eps_2(x).\]Alors $fg$ admet le développement limité à l'ordre $n$ en $0$ suivant : \[ (fg)(x) = T_n(PQ)(x) + x^{n}\eps(x).\]
}{}
\paragraph{Remarque}Si $f,g$ admettent les développements limités à l'ordre $n$ en $a$ : \[ f(x) = P(x-a) + (x-a)^{n}\eps_1(x), \; g(x) = Q(x-a) + (x-a)^{n} \eps_2(x)\]alors le développement limité : \[ (fg)(x)  = T_n(PQ)(x-a)\note{On tronque avant d'évaluer en $x-a$.} + (x-a)^{n}\eps(x). \]
\demonstration{ 
\begin{align*}
(fg)(x) &= (PQ)(x) + x^{n}(Q\eps_1(x) + P\eps_2(x))\\
PQ(x) &= T_n(PQ)(x) + x^{n+1}R(x), \; R\in \R[x]\\
(fg)(x) &= T_n(PQ)(x) + x^{n}(xR(x) + Q\eps_1(x) + P\eps_2(x))
\end{align*}
On pose : 
\begin{align*}
\eps(x) &= xR(x) + Q\eps_1(x) + P\eps_2(x) \\
\lim_{x\to 0}&\ xR(x) = 0 \\
\lim_{x\to 0}&\ Q\eps_1(x) = 0\\
\lim_{x\to 0}&\ P\eps_2(x) =0 \\
\lim_{x\to 0}&\ \eps(x) = 0
\end{align*}
}{}

\paragraph{Exemple}On veut le développement limité de : \[ \Arctan(x-1)\exp(x)\]en $1$ d'ordre $3$.
\begin{align*}
\Arctan(y) &= y - \frac{y^{3}}{3} +y^{3}\eps(y) \\ 
\Arctan(x-1) &= (x-1) - \frac{(x-1)^{3}}{3}+(x-1)^{3}\eps(x) \\
\exp(x) &= \exp(x-1+1) = e\exp(x-1) \\
\exp(x) &= e\left(  1 + (x-1) + \frac{(x-1)^{2}}{2} + \frac{(x-1)^{3}}{6} + (x-1)^{3}\eps(x)\right)
\end{align*}
Et donc 
\begin{align*}
f(x) &= e\left( (x-1) - \frac{(x-1)^{3}}{3} + (x-1)^{3}\eps(x) \right)\fois \left( 1 + (x-1) + \frac{(x-1)^{2}}{2} + \frac{(x-1)^{3}}{6} + (x-1)^{3}\eps(x) \right)\\
f(x) &= e\left( (x-1) + (x-1)^{2} + \frac{(x-1)^{3}}{2} - \frac{(x-1)^{3}}{3} \right) + (x-1)^{3}\eps(x)
\end{align*}

\subsection{Développement limité d'une fonction composée}
Puisque la composition de deux fonctions polynômiales est encore un polynôme :

\proposition{ 
Soient $f,g$ admettant un développement limité en $0$ à l'ordre $n$ : \[f(x) = P(x) + x^{n}\eps(x), \; g(x) = Q(x) + x^{n}\eps(x) \]avec $P,Q$ deux polynômes de degré inférieur à $n$.

Supposons que $g(0) = 0$ alors $f\rond g$ admet le développement limité suivant à l'ordre $n$ en $0$ : \[ (f\rond g)(x) = T_n(P\rond Q)(x) + x^{n}\eps(x).\]
}{}
\demonstration{ 
Supposons $n=0$, alors $P$ et $Q$ sont deux polynômes constants donc $f(x) = P(0) + \eps(x)$ et $g(x) = Q(0) + \eps(x)$. Comme $Q(0) = 0$ on a bien $f(g(x)) = (P\rond Q)(x) + \eps(x)$ par continuité.

Supposons que $n\geq 1$. On note $f(x) = P(x) + x^{n}\eps_1(x)$ et $g(x) = Q(x) + x^{n}\eps_2(x)$. Posons $P(x) = a_0 + a_1 x + \ldots + a_nx^{n}$.
\begin{align*}
(f\rond g)(x) &= P(g(x)) + g(x)^{n}\eps_1(g(x)) \\
P(g(x)) &= \sum_{i=0}^{n}a_ig(x)^{i} \\
P(g(x)) &=\footnotemark T_n\left(\sum_{i=0}^{n}a_iQ(x)^{i}\right) + x^{n}\eps_3(x)
\end{align*}
Puisque $Q(0) = 0$, on a $Q(x) = b_1x+\ldots + b_nx^{n}$ et donc : 
\begin{align*}
g(x) &= b_1x + \ldots+ b_nx^{n} + x^{n}\eps_2(x) \\
g(x) &= x(b_1+\ldots+ b_nx^{n-1} + x^{n-1} \eps_2(x))\\
g(x) &= xh(x)\\
(f\rond g)(x) &= P(xh(x)) + x^{n}h(x)^{n}\eps_1(xh(x))\\
(f\rond g)(x) &= T_n(P\rond Q)(x)  + x^{n}(h(x)^{n}\eps_1(xh(x)) + \eps_3(x))
\end{align*}
On pose $\eps_4(x) = h(x)^{n}\eps_1(xh(x)) + \eps_3(x)$ et : 
\begin{align*}
\lim_{x\to 0} xh(x) &= 0\\
\lim_{x\to 0}\eps_3(x) &= 0 \\
\lim_{x\to 0} h(x)^{n} &= b_1^{n} \\
\lim_{x\to 0} \eps_4(x) &= 0.
\end{align*}
}{}
\footnotetext{D'après les formules de développements limités d'une somme et d'un produit.}

\paragraph{Exemple}Développement limité de $\cos(\sin(x))$ à  l'ordre $5$ en $0$ :
\begin{align*}
\sin(x) &= x - \frac{x^{3}}{3!} + \frac{x^{5}}{5!}+x^{6}\eps(x) \\
\cos(x) &= 1-\frac{x^{2}}{2!} + \frac{x^{4}}{4!} + x^{5}\eps(x)\\
\cos(\sin(x)) &= T_5\left(1 - \frac{\left(x - \frac{x^{3}}{3!} + \frac{x^{5}}{5!} \right)^{2}}{2!} + \frac{\left(x - \frac{x^{3}}{3!} + \frac{x^{5}}{5!}\right)^{4}}{4!} \right) + x^{5}\eps(x)\\
\cos(\sin(x)) &= 1 - \frac{x^{2}}{2} + \frac{x^{4}}{3!}+ \frac{x^{4}}{4!}  + x^{5}\eps(x) \\
\cos(\sin(x)) &= 1 - \frac{x^{2}}{2} + \frac{5x^{4}}{24} + x^{5}\eps(x)
\end{align*}

\proposition{ 
Soient $f,g$ admettant des développements limités à l'ordre $n$ en $0$. Alors si $g(0) \neq 0$ alors la fonction $f/g$ admet un développement limité à l'ordre $n$ en $0$.
}{}
\demonstration{ 
Puisque $g(0)\neq 0$, $f/g$ est définie et continue en $0$.
Comme $f/g = f \times 1/g$, il suffit de vérifier que $1/g$ admet un développement limité en $0$ (puis on applique la règle de produit).

Posons $a = g(0) \neq 0$. On a : \[\frac{1}{g(x)} = \frac{1}{a +(g(x) - a)} = \frac{1}{a} \cdot\frac{1}{1 + \left(\frac{g(x)}{a} - 1 \right)} \]
Il suffit de vérifier que : \[ \frac{1}{1 + \left(\frac{g(x)}{a} - 1 \right)} \]admet un développement limité à l'ordre $n$ en $0$.
Posons \[h(x) = \frac{1}{1+x} \]on a alors \[ \frac{1}{1 + \left(\frac{g(x)}{a} - 1 \right)}  = h\left( \frac{g(x)}{a}-1\right) = (h\rond k)(x)\]où $k(x) = g(x)/a - 1$. Or $k(x)$ admet un développement limité à l'ordre $n$ en $0$ et $h(x)$ admet également un développement limité à l'ordre $\infty$ en $0$. Enfin, $k(0) = 0$ et donc on conclut avec le résultat précédent.
}{}
\paragraph{Exemple}Développement limité de $f : f(x) = 1/(a-x)$ en $0$ à l'ordre $n$.
\begin{align*}
f(x) &= \frac{1}{a}\frac{1}{1-x/a}\\
\frac{1}{1-t} &= 1 + t + t^{2} + \ldots + t^{n} + t^{n}\eps(t) \\
f(x) &= \frac{1}{a}\left( 1 + \frac{x}{a} + \frac{x^{2}}{a^{2}} + \ldots + \frac{x^{n}}{a^{n}} \right) + x^{n}\eps(x) \\
\frac{1}{a-x} &= \frac{1}{a}+\frac{x}{a^{2}} + \ldots + \frac{x^{n}}{a^{n+1}}+ x^{n}\eps(x).
\end{align*}

La méthode précédente ne donne pas de formule générale pour le développement limité de $f/g$.
\paragraph{Rappel}Si $P,Q\in \R[x]$, $n\in \N$ et si $Q(0)\neq 0$. Alors la division de $P$ par $Q$ suivant les puissances croissantes à l'ordre $n$ est l'unique polynôme $A$ tel que :
\begin{itemize}
\item $P-AQ$ est divisible par $X^{n+1}$ ;
\item soit $A=0$, soit $\deg A \leq n$.
\end{itemize}
\proposition{Soient $f,g$ avec les développements limités suivants à l'ordre $n$ en $0$ :
\begin{align*}
f(x) &= A(x) + x^{n}\eps_1(x), \\
g(x) &= B(x) + x^{n}\eps_2(x).
\end{align*}
Supposons que $g(0)=B(0)\neq 0$. Le développement limité à l'ordre $n$ de $f/g$ en $0$ est : 
\[ \frac{f}{g}(x) = Q(x) + x^{n}\eps(x)\]où $Q$ est la division de $A$ par $B$ à l'ordre $n$ suivant les puissances croissantes.
}{}
\demonstration{ 
On a $A(x) = Q(x)B(x) + x^{n+1}R(x)$ où $R$ est un polynôme et $Q=0$ ou $\deg Q \leq n$. Ainsi 
\begin{align*}
 f(x) &= Q(x)B(x) + x^{n+1}R(x) + x^{n}\eps_1(x)\\
f(x) - Q(x)g(x) &= x^{n+1}R(x) + x^{n}\eps_1(x) - Q(x)x^{n}\eps_2(x) \\
f(x) - Q(x)g(x) &= x^{n}(\eps_1(x) - Q(x) \eps_2(x) + xR(x))\\
\frac{f}{g}(x) &= Q(x) + x^{n}\eps_3(x)\\
\eps_3(x) &= \frac{1}{g(x)}(\eps_1(x) - Q(x)\cdot \eps_2(x) + xR(x)) \underset{x\to 0}{\longrightarrow}0
\end{align*}
}{}
\paragraph{Exemple}Développement limité de $\tan(x)$ à l'ordre $5$ en $0$.
\begin{align*}
\tan(x) &= \frac{\sin(x)}{\cos(x)}\\
\sin(x) &= x - \frac{x^{3}}{3!} + \frac{x^{5}}{5!} + x^{5}\eps(x) \\
\cos(x) &= 1 - \frac{x^{2}}{2!} + \frac{x^{4}}{4!} + x^{5}\eps(x) \\
x - \frac{x^{3}}{3!} + \frac{x^{5}}{5!}  &= \left(1 - \frac{x^{2}}{2!} + \frac{x^{4}}{4!} \right)\left( x + \frac{x^{3}}{3} + \frac{2x^{5}}{15}\right) + x^{6}R(x) \\
\frac{f(x)}{g(x)} &= x + \frac{x^{3}}{3} + \frac{2}{15}x^{5} + x^{5}\eps(x)
\end{align*}

\section{Applications}

\paragraph{Applications}Les développements limités peuvent être utiles pour : 
\begin{enumerate}
\item les calculs de limites (pour des \og formes indéterminées \fg{}) ;
\item études de fonctions ou courbes paramétrées.
\end{enumerate}

\subsection{Calculs de limites}
\paragraph{Exemple}On veut calculer : \[ \lim_{x\to 0} \frac{x\log\ch x}{1+x\sqrt{1+x} - \exp(\sin x)}.\]
\begin{align*}
\ch(x) &= 1 +\frac{x^{2}}{2!} + x^{2}\eps(x) \\
\log(1+x) &= x - \frac{x^{2}}{2} +x^{2}\eps(x), \\
\log \ch x &= \log(1+(\ch x - 1)) \\
\log \ch x &= T_2\left( \frac{x^{2}}{2}  - \frac{\left(\frac{x^{2}}{2} \right)^{2}}{2}\right) + x^{2}\eps(x) \\
\log \ch x &= \frac{x^{2}}{2} + x^{2}\eps(x) \\
x\log\ch x &= \frac{x^{3}}{2} + x^{3}\eps(x) \ ;
\end{align*}
\begin{align*}
\sqrt{1+x} &= 1 + \frac{x}{2} + \frac{\frac{1}{2}\left(\frac{1}{2}-1\right)x^{2}}{2!} + x^{2}\eps(x) \\
\sqrt{1+x} &= 1 + \frac{x}{2} - \frac{x^{2}}{8} + x^{2}\eps(x)\\
x\sqrt{1+x} &= x + \frac{x^{2}}{2} - \frac{x^{3}}{8} + x^{3}\eps(x) \\
\sin(x) &= x - \frac{x^{3}}{6} + x^{3}\eps(x) \\
\exp(x) &= 1 + x + \frac{x^{2}}{2} + \frac{x^{3}}{6} +x^{3}\eps(x), \\
\exp(\sin x)) &= T_3\left( \left(x\donne 1 + x + \frac{x^{2}}{2} + \frac{x^{3}}{6} \right) \left( x - \frac{x^{3}}{6}\right)\right) + x^{3}\eps(x) \\
\exp(\sin x)) &= 1 + x - \frac{x^{3}}{6} + \frac{x^{2}}{2} + \frac{x^{3}}{6} + x^{3}\eps(x) \\
\exp(\sin x)) &= 1 + x + \frac{x^{2}}{2} + x^{3}\eps(x) \ ;
\end{align*}
Ainsi 
\begin{align*}
\frac{x\log\ch x}{1+x\sqrt{1+x} - \exp(\sin x)} &= \frac{ \frac{x^{3}}{2} + x^{3}\eps(x)}{1 +x + \frac{x^{2}}{2} - \frac{x^{3}}{8}  -1 - x - \frac{x^{2}}{2} + x^{3}\eps(x)}\\
\frac{x\log\ch x}{1+x\sqrt{1+x} - \exp(\sin x)} &= \frac{\frac{x^{3}}{2} + x^{3}\eps(x)}{-\frac{x^{3}}{8} + x^{3}\eps(x)} \\
\lim_{x\to 0}\frac{x\log\ch x}{1+x\sqrt{1+x} - \exp(\sin x)} &=\lim_{x\to 0} \frac{1/2 + \eps(x)}{-1/8 + \eps(x)} = -4.
\end{align*}

\paragraph{Remarque}Un calcul de dérivée s'obtient par un calcul de limite et donc parfois par développements limités.
\paragraph{Exemple}On prend \[ f(x) = \frac{\cos x}{1+x+x^{2}}\]et on cherche $f^{(i)}(0)$ pour $i \in \ens{0,\ldots,4}$, c'est-à-dire que l'on cherche le développement limité de $f$ en $0$ à l'ordre $4$.
\begin{align*}
\cos x &= 1 - \frac{x^{2}}{2} + \frac{x^{4}}{4!} + x^{4}\eps(x).
\end{align*}
On cherche le développement limité de \[ g(x) = \frac{1}{1+x+x^{2}}\]que l'on peut voir comme \[ g(x) = (a\rond b)(x)\; ; \; a(x) = \frac{1}{1+x} \; ; \; b(x) = x+x^{2}.\]

\begin{align*}
a(x) &= 1 - x + x^{2} - x^{3} + x^{4} + x^{4}\eps(x) \\
g(x) &= T_4((x\donne  1 - x + x^{2} - x^{3} + x^{4} )(x+x^{2})) + x^{4}\eps(x) \\
g(x) &= 1 - x - x^{2} + x^{2} + x^{4} + 2x^{3} +x^{4} - x^{3} - 3x^{4} + x^{4 }+ x^{4}\eps(x)\\
g(x) &= 1 - x + x^{3} - x^{4} + x^{4}\eps(x) \\
f(x) &= T_4\left( (1-x+x^{3}-x^{4})\left(1 - \frac{x^{2}}{2} + \frac{x^{4}}{24}\right)\right) + x^{4}\eps(x) \\
f(x) &= 1 - x -\frac{x^{2}}{2} + \frac{3x^{3}}{2} - \frac{23x^{4}}{24} + x^{4}\eps(x)
\end{align*}
Comme $f$ admet un développement limité à l'ordre $4$ en $0$, elle est dérivable quatre fois. De plus 
\begin{align*}
f(0) &= 1 \\
f'(0) &= -1 \\
f^{(2)}(0) &= -1 \\
f^{(3)}(0) &= 9 \\
f^{(4)}(0) &= -23.
\end{align*}

\subsection{Courbes paramétrées}

\paragraph{Rappels sur les fonctions classiques}Quelques rappels :
\begin{itemize}
\item on définit le logarithme népérien par : \[ \log(x) = \int_{1}^{x}\frac{\dt}{t}.\]Ainsi $\log : \R^{*}_{+} \to \R$ est croissante, $C^{\infty}$, 
\begin{align*}
\frac{\dd}{\dx}\log x &= \frac{1}{x}\\
\lim_{x\to 0, x>0} \log x &= -\infty \\
\lim_{x\to \infty} \log x &= +\infty \\
\log(ab) = \log a + \log b.
\end{align*}
\item on définit l'exponentielle, $\exp : \R \vers \R$, qui est croissante, lisse et stable par dérivation.
\begin{align*}
\lim_{x\to -\infty}\exp(x) &= 0 \\
\lim_{x\to + \infty} \exp(x) &= \infty\\
\exp(a+b) &= \exp(a)\exp(b).
\end{align*}
\item soient $a\in \R_+^{*}, b\in \R$ alors on définit : \[ a^{b} = \exp(b\log a).\]
\begin{align*}
a^{b+b'} &= a^{b}a^{b'} \\
(aa')^{b} &= a^{b} (a')^{b} \\
\left(a^{b}\right)^{c} &= a^{bc}\\
a^{0} &= 1 = 1^{b}\\
\frac{\dd }{\dx}x^{b} &= bx^{b-1} \\
\frac{\dd }{\dx}a^{x} &= \log (a)a^{x} \\
\lim_{x\to 0, x >0} x^{a}(\log x)^{n} &= 0\; \; ,\; a >0 \et n\in \Z \\
\lim_{x\to+\infty} x^{a} e^{x} &= +\infty \\
\lim_{x\to -\infty} x^{a}e^{x} &= 0.
\end{align*}
\item trigonométrie :
\begin{align*}
\lim_{x\to 0}\frac{\sin x}{x} &= 1 \\
\sin(x+t) &= \cos(t)\sin(x) + \cos(x)\sin(t) \\
\cos(x+t) &= \cos(x)\cos(t) - \sin(x)\sin(t)\\
\tan(x+t) &= \frac{\tan(t) + \tan(x)}{1 - \tan(x)\tan(t)}.
\end{align*}
\item $\Arcsin : [-1,1] \vers [-\pi/2,\pi/2]$ est lisse sur $]-1,1[$ et : \[ \Arcsin' (x) = \frac{1}{\sqrt{1-x^{2}}}.\]$\Arccos : [-1,1] \vers [0,\pi]$ est la réciproque de $\cos$ et on a la relation : \[ \Arccos(x) + \Arcsin(x) = \frac{\pi}{2}.\]$\Arctan : \R\vers ]-\pi/2,\pi/2[$ est lisse et : \[ \Arctan'(x) = \frac{1}{x^{2}+1}.\]
\item trigonométrie hyperbolique : 
\begin{align*}
\sh(x) &= \frac{e^{x} - e^{-x}}{2} \\
\ch(x) &= \frac{e^{x} + e^{-x}}{2} \\
\th(x) &= \frac{\sh(x)}{\ch(x)}
\end{align*}leurs réciproques $\Argsh : \R \vers \R$, $\Argch : [-1,\infty]\vers \R_+$ et $\Argth:]-1,+1[ \vers \R$ sont lisses sur l'intérieur de leur domaine de définition.
\begin{align*}
\Argsh'(t) &= \frac{1}{\sqrt{1+t^{2}}} \\
\Argch'(t) &= \frac{1}{\sqrt{t^{2}-1}} \\
\Argsh(t) &= \log(t + \sqrt{t^{2}+1}) \\
\Argch(t) &= \log(t + \sqrt{t^{2}-1}).
\end{align*}
\end{itemize}

\definition{ 
Soit $f : I \vers \R^{2}$ avec $I$ un intervalle ou une union finie d'intervalles dans $\R$. Soient $u,v$ telles que \[ \forall t, \; f(t) = (u(t),v(t)).\]
\begin{enumerate}
\item On dit que $\lim_{t\to t_0}f(t) = l$ où $l = (l_1,l_2)$ si $\lim_{t\to t_0}u(t) = l_1$ et $\lim_{t\to t_0}v(t) = l_2$.
\item On dit que $f$ est continue en $t_0$ si les fonctions $u$ et $v$ sont continues en $0$. $f$ est continue sur $I$ si elle est continue en tout point de $I$.
\item On dit que $f$ est dérivable en $t_0$ si $u$ et $v$ le sont et on note $f'(t_0) = (u'(t_0),v'(t_0))$.
\end{enumerate}
}{}

\proposition{ 
Si $f,g :  I \vers \R^{2}$ et si $t_0 \in I$ alors :
\begin{enumerate}
\item si $\lim_{t\to t_0} f(t) = l$ et $\lim_{t\to t_0}g(t) = m$ alors $\lim_{t\to t_0}(f+g)(t) = l+m$ ;
\item si $f,g$ sont dérivables en $t_0$ alors $f+g$ aussi et on a $(f+\lambda g)'(t_0) = f'(t_0) + \lambda g'(t_0)$.
\end{enumerate}
}{}
\proposition{ 
Soit $(r,s)$ une base de $\R^{2}$ et soit $f : I \vers \R^{2}$ telle que $f(t) = (u(t),v(t))$. Soit $(a(t),b(t))$ les coordonnées de $f(t)$ dans la base $(r,s)$.
\begin{enumerate}
\item On a : \[\lim_{t\to t_0}f(t) = l \ssi \systeme{ \lim_{t\to t_0} a(t) = \alpha \\ \lim_{t\to t_0}b(t) = \beta} \]où $(\alpha,\beta)$ sont les coordonnées de $l$ dans la base $(r,s)$.
\item Idem pour la dérivée.
\end{enumerate}
}{}

\demonstration{ 
Soient $\alpha,\beta\in \R$ et $r,s\in \R^{2}$. On a $l = \alpha \cdot r + \beta\cdot s$, \[f(t) = (u(t),v(t)) = a(t)\cdot r + b(t) \cdot s \]avec $a(t),b(t) \in \R$.
\begin{enumerate}
\item On a que $\lim_{t\to t_0}f(t) = l$ c'est par définition : \[ \systeme{\lim_{t \to t_0}u(t) =l_1 \\ \lim_{t\to t_0}v(t) = l_2}.\]
\begin{align*}
\systeme{\lim_{t\to t_0}a(t) = \alpha \\ \lim_{t\to t_0}b(t) = \beta} &\ssi \systeme{\lim_{t\to t_0}(a(t)r_1 + b(t)s_1) = \alpha r_1 + \beta s_1 \\ \lim_{t\to t_0}(a(t)r_2 + b(t)s_2) = \alpha r_2 + \beta s_2} \\
& \ssi \systeme{\lim_{t\to t_0} a(t) r_1 + b(t) s_1 = l_1 \\ \lim_{t \to t_0}a(t) r_2 + b(t) s_2 = l_2 }
\end{align*}
\item De même ... 
\end{enumerate}
}{}

\definition{ 
On dit que $f : I \vers \R^{2}, f(t) = (u(t),v(t))$ admet un développement limité à l'ordre $n$ en $t_0$ si $u(t)$ et $v(t)$ admettent un développement limité à l'ordre $n$ en $t_0$.

Si $u(t) = u_0 + u_1(t-t_0) + \ldots + u_n(t-t_0)^{n} + (t-t_0)^{n}\eps_1(t)$ et $v(t) = v_0 + v_1(t-t_0) + \ldots + v_n(t-t_0)^{n} + (t-t_0)^{n}\eps_2(t)$ alors on appelle \[ f(t) = (u_0,v_0) + (t-t_0) (u_1,v_1) + \ldots + (t-t_0)^{n}(u_n,v_n) + (t-t_0)^{n}\eps(t)\]le développement limité de $f$ à l'ordre $n$ en $t_0$ avec $\lim_{t\to t_0} \eps(t) = (0,0)$.
}{}
\paragraph{Exemple}Le développement limité de $f : t\donne (2t^{3}-t\sin t, t^{3}+\cos t)$ à l'ordre $4$ en $0$ :
\begin{align*}
2t^{3} -t\sin t &= -t^{2} + 2t^{3} +\frac{t^{4}}{6} + t^{4}\eps(t) \\
t^{3}+\cos t &= 1 - \frac{t^{2}}{2} + t^{3} + \frac{t^{4}}{24} + t^{4}\eps(t)\\
f(t) &= (0,1) - t^{2}(1,1/2) + t^{3}(2,1) + t^{4}(1/6,1/24) + t^{4}\eps(t).
\end{align*}

\definition{ 
On appelle \textit{courbe paramétrée} de $\R^{2}$ une fonction $f:I\vers \R^{2}$.
}{}
\paragraph{Exemple}$f:R\vers \R^{2}, t \donne (\cos t,\sin t)$.
\begin{center}
\sageplot[width = 8cm]{parametric_plot((cos(x),sin(x)),(x,0,2*pi))}
\end{center}

\paragraph{Remarque}
Supposons que $f$ soit dérivable en $t\in I$. Alors $u(t),v(t)$ admettent des développements limités à l'ordre $1$ en $t_0$ et donc $f$ admet aussi un développement limité à l'ordre $1$ en $t_0$. Or si \[ f(t) = f(t_0) + (t-t_0) f'(t_0) + (t-t_0)\eps(t)\] alors \[ \lim_{t\to t_0}\frac{1}{t-t_0}(f(t) - f(t_0)) = f'(t_0).\]

\definition{
On appelle $f'(t_0)$ \textit{vecteur tangent} de $f$ en $t_0$. La droite affine passant par $f(t_0)$ et de vecteur directeur $f'(t_0)$ s'appelle la \textit{tangente} à $f$ en $t_0$.
}{}

\paragraph{Remarque}Le vecteur tangent dépend du paramétrage de la courbe et non seulement de sa représentation.

\paragraph{Exemple}Soit $f: \R \vers \R^{2}, t \donne (\cos(t),\sin(t))$ et soit $g : \R \vers \R^{2},t\donne (\cos(2t),\sin(2t))$. Remarquons que $f$ et $g$ on même représentation graphique. Cependant les vecteurs tangents en $0$ à $f$ et $g$ sont :  
\begin{align*}
f'(0) &= (0,1) \\
g'(0) &= (0,2).
\end{align*}
La tangente à $f$ en $t_0$ est la droite d'équation : \[\det \matrice{y - v(t_0) & v'(t_0) \\ x - u(t_0) & u'(t_0)} = 0 \]c'est-à-dire : \[(y-v(t_0))u'(t_0) - (x-u(t_0))v'(t_0) = 0. \]

\newpage
\chapter{Courbes et surfaces paramétrées}

\section{Définitions}
Soit $I\dans \R$ un intervalle et $f : I\vers \R^{2}$ telle que $f(t) = (u(t),v(t))$.

\definition{ 
Supposons que $u,v$ sont continues.

Si $u$ et $v$ admettent un développement limité à l'ordre $n$ au point $t_0$ : 
\begin{align*}
u(t) = u_0 + u_1(t-t_0) + \ldots + u_n(t-t_0)^{n} + (t-t_0)^{n}\eps(t) \\
v(t) = v_0 + v_1(t-t_0) + \ldots + v_n(t-t_0)^{n} + (t-t_0)^{n}\eps(t)
\end{align*}
alors $f$ admet un développement limité à l'ordre $n$ en $t_0$ : 
\begin{align*}
f(t) &= f_0 + (t-t_0)f_1 + \ldots+ (t-t_0)^{n}f_n + (t-t_0)^{n}\eps(t) \\
\forall i\in \ens{0,\ldots,n}, \; f_i& = (u_i,v_i).
\end{align*}

L'égalité précédente s'appelle le \textit{développement limité de $f$ en $t_0$ à l'ordre $n$}. 
}{}
\paragraph{Remarque}On a bien \[ \lim_{t\to t_0}\eps(t) = (0,0).\]

\definition{ 
Une fonction $f : I \vers \R^{2}$ s'appelle \textit{courbe paramétrée de $\R^{2}$}.
}{}


Supposons que $f$ est dérivable en $t_0\in I$. $f$ admet le développement limité en $t_0$ à l'ordre $1$ suivant : \[ f(t) = f(t_0) + (t-t_0)f'(t_0) + (t-t_0)\eps(t).\]


\section{Tangentes}


\definition{ 
Si $f'(t_0)\neq 0$ alors la tangente à la courbe au point $f(t_0)$ est la droite affine passant par $f(t_0)$ et de vecteur directeur $f'(t_0)$. L'équation est \[ \det\matrice{x -u(t_0) & u'(t_0) \\ y - v(t_0) & v'(t_0)} = 0.\]En d'autres termes, c'est l'équation : \[ (y-v(t_0))\cdot u'(t_0) - (x-u(t_0))\cdot v'(t_0) = 0.\]
}{}

On se demande quelles sont les conditions à l'existence de la tangente en un point ainsi que la position de la tangente par rapport à la courbe.

\paragraph{Remarque}On retrouve l'étude des fonctions à valeurs dans $\R$ si on a \[f(t) = (t,v(t)). \]

Supposons que $u,v$ admettent des développements limités en $t_0$ à l'ordre $n\geq 2$. On a \[f(t) = f(t_0) + (t-t_0)f'(t_0) + (t-t_0)^{2}w_2 + \ldots + (t-t_0)^{n}w_n + (t-t_0)^{n}\eps(t)\]où $w_2,\ldots,w_n\in \R^{2}$ et $\lim_{t\to t_0}\eps(t) = 0_{\R^{2}}$.
\begin{enumerate}
\item Supposons que $f'(t_0)\neq 0$ et $f'(t_0)$ est non colinéaire à $w_2$. On tronque le développement limité à l'ordre $2$  : \[ f(t) = f(t_0) + (t-t_0)f'(t_0) + (t-t_0)^{2}w_2 + (t-t_0)^{2}\eps(t).\]Soient $(a(t),b(t))$ les coordonnées de $\eps(t)$ dans la base $(f'(t_0),w_2)$. Ainsi :
\begin{align*}
f(t) - f(t_0) &= \left(t-t_0 + (t-t_0)^{2}a(t)\right)f'(t_0) + (t-t_0)^{2}(b(t) +1)w_2\\
\lim_{t\to t_0} a(t) &= \lim_{t\to t_0} b(t) = 0.
\end{align*}
Selon la coordonnée de $f'(t_0)$ on a que $(t-t_0)^{2}a(t)$ tend vers $0$ et alors $t-t_0$ détermine le signe. Selon la coordonnée $w_2$, dans un voisinage suffisamment petit de $t_0$ on a que la coordonnée est de signe positif.
\item Supposons que $f'(t_0)\neq 0$, $w_2 = \lambda f'(t_0)$ et enfin $w_3$ et $f'(t_0)$ non colinéaires. On a alors dans la base $(f'(t_0),w_3)$ : \[f(t) - f(t_0) = \left(t-t_0 + \lambda(t-t_0)^{2}\right)f'(t_0) + (t-t_0)^{3}w_3 + (t-t_0)^{3}\eps(t). \]
On décompose $\eps(t)$ dans cette base : \[\eps(t) = a(t) f'(t_0) + b(t) w_3.\]On sait que \[ \lim_{t\to t_0}a(t) = \lim_{t\to t_0}b(t) = 0.\]
Dans cette base, on a : \[ f(t) - f(t_0) = \matrice{t-t_0 + \lambda(t-t_0)^{2} + (t-t_0)^{3}a(t) \\ (t-t_0)^{3} + (t-t_0)^{3}b(t) }\]
Sur chaque coordonnée, le signe est celui de $t-t_0$.


\paragraph{Remarque}Supposons $f'(t_0)\neq 0, n\geq 3$ et il existe un entier $p\in\ens{3,\ldots,n}$ tel que les vecteurs $w_2,w_3,\ldots,w_{p-1}$ sont colinéaires à $f'(t_0)$ et tel que $w_p$ n'est pas colinéaire à $f'(t_0)$. Ainsi, $(f'(t_0),w_p)$ est une base de $\R^{2}$.

On écrit le développement limité de $f(t)-f(t_0)$ dans cette base. On étudie le signe des coordonnées de $f(t)-f(t_0)$ quand $t\to t_0$. Si $p$ est pair alors la courbe est comme dans le cas $p=2$ (la courbe est du côté de $w_p$ par rapport à la tangente), sinon comme dans le cas $p=3 $(elle traverse la tangente).

\item Supposons que $f'(t_0) = 0$ et que $w_2,w_3$ forme une base de $\R^{2}$. On a \[f(t) - f(t_0) = (t-t_0)^{2}w_2 + (t-t_0)^{3}w_3 + (t-t_0)^{3}\eps(t).\]
On décompose $\eps(t)$ dans la base $(w_2,w_3)$ : $\eps(t)= a(t)w_2 + b(t)w_3$ avec $\lim_{t\to t_0}a(t) = \lim_{t\to t_0}b(t) = 0$. Les coordonnées dans cette base de $f(t)-f(t_0)$ sont alors : \[ f(t)-f(t_0) = \matrice{ (t-t_0)^{2} + (t-t_0)^{3}a(t) \\ (t-t_0)^{3} + (t-t_0)^{3}b(t) }. \]
Ainsi, la première coordonnée est positive et la seconde est du signe de $t-t_0$.
Une telle situation est un \textit{point de rebroussement}.

\item Supposons que $f'(t_0) = 0$, $w_3=\lambda w_2$ et $w_2,w_4$ forme une base. On pose $\eps(t) = a(t)w_2 + b(t)w_4$. Dans ces coordonnées : \[ f(t)-f(t_0) = \matrice{ (t-t_0)^{2} + \lambda (t-t_0)^{3} + (t-t_0)^{4}a(t) \\ (t-t_0)^{4}(1+b(t)) }.\]
Les deux coordonnées sont positives quand $t\to t_0$. C'est aussi un point de rebroussement
\end{enumerate}

\section{Branches infinies}

\definition{ 
Soit $f:I\vers \R^{2}$ une courbe paramétrée avec $f=(u,v)$. Soit $t_0\in \barre{I}\union\ens{\infty}\union\ens{-\infty}$.
\begin{itemize}
\item On a une \textit{branche infinie} quand $t\to t_0$ si soit $u$ ou soit $v$ n'est pas bornée.
\item Si $\lim_{t\to t_0}u(t) = a\in \R$ et si $\lim_{t\to t_0}v(t) = \pm\infty$ alors la droite $x=a$ est une \textit{asymptote verticale}.
\item Si $\lim_{t\to t_0}u(t) = \pm\infty$ et $\lim_{t\to t_0}v(t) =a\in \R$ alors la droite $y=a$ est \textit{asymptote horizontale}.
\item Si $u$ et $v$ tendent vers $\pm\infty$ en $t_0$ :
\begin{itemize}
\item Si $\lim_{t\to t_0}u(t)/v(t) =a\in \R$ alors la droite $y=ax$ est direction asymptotique.
\item Si de plus $\lim_{t\to t_0}(v(t)-au(t)) = b$ alors la droite $y=ax+b$ est asymptote.
\end{itemize}
\end{itemize}
}{}

\paragraph{Exemple}Soit $f$ : \[ \fonc{f}{\left]-\frac{\pi}{2},\frac{\pi}{2}\right[}{\R^{2}}{t}{(\tan(t),2t-1/\cos(t))}.\]On étudie l'asymptote en $t_0 = \pi/2$. \[\lim_{t\to t_0} u(t) = +\infty, \; \lim_{t\to t_0}v(t) = -\infty.\]On étudie le rapport $v/u$ en $t\to t_0$. On pose $t=\pi/2+h$.
\begin{align*}
u(t) &= \tan(\pi/2-h) = \frac{\sin(\pi/2-h)}{\cos(\pi/2-h)}\\
u(t) &= \frac{\cos (h)}{\sin (h)} = \frac{1 -h^{2}/2 + h^{2}\eps(h)}{h-h^{3}/6 + h^{3}\eps(h)}\\
u(t) &= \frac{1}{h}\left(1-\frac{h^{2}}{2}+h^{2}\eps(h)\right)\frac{1}{1-\frac{h^{2}}{6} + h^{2}\eps(h)}\\
u(t) &= \frac{1}{h}\left(1 -\frac{h^{2}}{2} + h^{2}\eps(h) \right)\left( 1+ \frac{h^{2}}{6} + h^{2}\eps(h) \right)\\
u(t) &= \frac{1}{h}\left( 1 - \frac{h^{2}}{3} + h^{2}\eps(h)\right)\\
\end{align*}
\begin{align*}
v(t) &= \pi - 2h - \frac{1}{\cos(\pi/2-h)} = \pi - 2h - \frac{1}{\sin(h)}\\
v(t) &= \pi - 2h - \frac{1}{h-\frac{h^{3}}{6} + h^{3}\eps(h)}\\
v(t) &= \pi - 2h- \frac{1}{h}\left(\frac{1}{1-\frac{h^{2}}{6}+h^{2}\eps(h)}\right)\\
v(t) &=-\frac{1}{h} + \pi-\frac{13}{6}h + h\eps(h) \\
\frac{v(t)}{u(t)} &= \frac{-\frac{1}{h}+\pi-\frac{13}{6}h + h\eps(h)}{\frac{1}{h}-\frac{1}{3}h + h\eps(h)}\\
\frac{v(t)}{u(t)} &= \frac{-1+\pi h -\frac{13}{6}h^{2} + h^{2}\eps(h)}{1-\frac{1}{3}h^{2}+h^{2}\eps(h)}\\
\lim_{t\to \frac{\pi}{2}^{-}} v(t)/u(t) &= -1.
\end{align*}
Et donc $y=-x$ est direction asymptotique.
\begin{align*}
v(t) + u(t) &= -\frac{1}{h} + \pi -\frac{13}{6}h - \frac{1}{h} + \frac{1}{3}h + h\eps(h) \\
\lim_{t\to \frac{\pi}{2}^{-}}v(t)+u(t) = \pi.
\end{align*}
Et donc la droite d'équation $y=-x+\pi$ est asymptote en $t_0$.

\section{\'Etude de courbes paramétrées}

Soit : \[ \fonc{f}{\R\prive{-1,+1}}{\R^{2}}{t}{\left(\frac{t^{2}+1}{t^{2}-1},\frac{t^{2}}{t-1}\right)}.\]
On a :
\begin{align*}
u'(t) &= \frac{-4t}{(t^{2}-1)^{2}}\\
v'(t) &= \frac{t(t-2)}{(t-1)^{2}}.
\end{align*}
%\[ \begin{matrix}
%t & \vline & -\infty& & -1&  0 &1& 2&+\infty\\
%\hline
%u'(t) & \vline & &+& \vline\vline& + &\vline & - &\vline\vline & - && - \\
%\hline
%u(t) & \vline & _1\nearrow^{+\infty} & \vline\vline&_{-\infty}\nearrow^{-1}\searrow_{-\infty} & \vline\vline& ^{+\infty}\searrow 5/3\searrow 1\\
%\hline 
%v'(t) & \vline & + && + & \vline & - & \vline \vline & - & \vline & +\\
%\hline
%v(t) & -\infty\nearrow&-1/2&\nearrow &0&\searrow&_{-\infty}&\vline\vline & ^{+\infty}& \searrow 4&\nearrow&+\infty
%\end{matrix}\]

Au voisinage $t=0$ :
\begin{align*}
u(t) &= -1-2t^{2}+t^{3}\eps(t) \\
v(t) &= -t^{2}-t^{3}+t^{3}\eps(t)\\
f(t) &= (-1,0) + t^{2}(-2,-1) + t^{3}(0,-1) + t^{3}\eps(t)\\
f(0) &= (-1,0), \; f'(0)= (0,0).
\end{align*}
Les vecteurs $(-2,-1)$ et $(0,-1)$ sont linéairement indépendants.


$t=0$ est un point singulier car $f'(0) = (0,0)$.
\paragraph{Branches infinies}On a :
\begin{itemize}
\item Une asymptote horizontale $y = -1/2$ quand $t\to -1$.
\item Quand $t\to 1$ : \[ \frac{v(t)}{u(t)} \frac{t^{2}(t+1)}{t^{2}+1} \underset{t\to 1}{\longrightarrow} 1\]donc une direction asymptotique $y=x$. \[ v(t) - 1\times u(t) = \frac{t^{2}+t+1}{t+1}\to \frac{3}{2}\]et donc l'asymptote est $y=x+3/2$.
\item Quand $t\to-\infty$ alors $u\to 1$ et $v\to-\infty$. On a une branche infinie et $x=1$ est asymptote verticale.
\item Quand $t\to+\infty$ alors $u\to 1$ et $v\to \infty$. $x=1$ est asymptote verticale.
\end{itemize}

\begin{sagesilent}
p = plot(x+3/2,(x,-5,5))
p += parametric_plot((1,x),(x,-5,5))
p += plot(-1/2,(x,-5,5))
t= var('t')
p += parametric_plot(((t**2+1)/(t**2-1),t**2/(t-1)),(t,1.1,5),rgbcolor = 'red')
p += parametric_plot(((t**2+1)/(t**2-1),t**2/(t-1)),(t,-1,0.8),rgbcolor = 'red')
p += parametric_plot(((t**2+1)/(t**2-1),t**2/(t-1)),(t,-5,-1.2),rgbcolor = 'red')
p.set_axes_range(-5,5,-5,5)
\end{sagesilent}
\begin{center}
\sageplot[width = 14cm]{p}
\end{center}

\newpage
\chapter{Séries numériques}
\section{Définitions}
On considère des séries  numériques, c'est-à-dire à valeurs dans $\R$.

\definition{ 
Soit $(u_n)_{n\in \N}$ une suite numérique.

On dit que la série $\Sigma u_n$ de terme général $u_n$ converge si la suite de terme général \[ s_n = \sum_{k=0}^{n}u_k\]converge.

Si la suite $s_n$ diverge, alors on dit que la série $\Sigma u_n$ de terme général $u_n$ diverge.

Les $s_n$ s'appellent les \textit{sommes partielles}.
}{}
\definition{ 
On note \[ \sum_{n=0}^{+\infty}u_n= \lim_{n\to+\infty}s_n\](quand elle est définie).

On l'appelle la \textit{somme} de la série $(\Sigma u_n)$.
}{}

\paragraph{Remarque}La suite de terme général \[ s_n = \sum_{k=0}^{n}u_k\] converge si, et seulement si, la suite de terme général (pour $n_0$ fixé) \[ S_n = \sum_{k=n_0}^{n}u_k\]converge.


\proposition{ 
Si la série $\sum u_n$ converge alors la suite $(u_n)_{n\in\N}$ converge vers $0$.
}{}
\demonstration{ 
Avec \[ s_n = \sum_{k=0}^{n}u_k\]et $l$ la limite de $s_n$.
Soit $\eps>0$. Par convergence de $s_n$, il existe $n_0$ tel que pour tout $n\geq n$, $\abs{l-s_n}<\eps$. Et donc \[ \abs{s_{n+1} - s_n} = \abs{s_n+1 -l + l-s_n} \leq \abs{s_{n+1}-l} + \abs{s_n-l} < 2\eps. \] Or \[ \abs{s_{n+1}-s_n} = \abs{u_{n+1}}< 2\eps.\]
}{}

\paragraph{Exemple -- Séries géométriques}Soit $x\in \R$. On pose \[ u_n = a\cdot x^{n}.\]On a 
\begin{align*}
s_n &= \sum_{k=0}^{n}u_k\\
s_n &= a\sum_{k=0}^{n}x^{n}\\
s_n &= a\frac{1-x^{n+1}}{1-x} = \frac{a}{1-x}(1-x^{n+1}).
\end{align*}
\begin{itemize}
\item Si $\abs{x}<1$ alors $(s_n)_{n\in\N}$ converge vers $a/(1-x)$.
\item Si $\abs{x}\geq 1$ alors la série $\sum ax^{n}$ diverge.
\end{itemize}
\paragraph{Exemple -- Série exponentielle}Soit $x\in \R$. On regarde la série de terme général $x^{n}/n!$. Alors cette série a pour somme partielle : \[s_n = \sum_{k=0}^{n}\frac{x^{k}}{k!}\]et la formule de \textsc{Taylor} nous assure que $s_n$ tend vers $\exp(x)$. La série est convergente pour tout $x$ et de somme $\exp(x)$.

\paragraph{Exemple}Soit $x\in \R$. On considère la série \[ \sum_{n\geq 1}\frac{x^{n}}{n}.\]
\begin{itemize}
\item Si $\abs{x}>1$ alors la suite de terme général $x^{n}/n$ ne converge pas et donc la série ne converge pas.
\item Si $x=1$ alors les sommes partielles sont \[ s_n = \sum_{k=1}^{n}\frac{1}{k}.\]Cependant \[ s_{2n} - s_n \geq \frac{n}{2n} = \frac{1}{2}.\]Ainsi, la série $\sum 1/n$ diverge.
\item Si $-1\leq x<1$ alors pour tout $n\geq 1$, on pose \[ \fonc{f_n}{\R}{\R}{t}{1+t^{2}+\ldots+t^{n-1}}\]et pour tout $t\neq 1$ : \[f_{n}(t) = \frac{1-t^{n}}{1-t} \]et alors \[\frac{1}{1-t}= f_n(t) + \frac{t^{n}}{1-t}. \]On peut intégrer, pour tout $x\in[-1,1[$ : 
\begin{align*}
\int_{0}^{x}\frac{\dt}{1-t} &= \int_{0}^{x}f_n(t)\dt + \int_{0}^{x}\frac{t^{n}}{1-t}\dt\\
-\log(1-x) &= x + \frac{x^{2}}{2} +\ldots + \frac{x^{n}}{n} + \int_{0}^{x}\frac{t^{n}}{1+t}\dt\\
-\log(1-x) &= s_n+ \int_{0}^{x}\frac{t^{n}}{1+t}\dt.
\end{align*}
Il s'agit donc d'examiner la convergence du dernier terme. 
\begin{enumerate}
\item Pour $0\leq x < 1$, on a $0\leq t \leq x < 1$ : 
\begin{align*}
\frac{t^{n}}{1-t} &\leq \frac{t^{n}}{1-x} \\
\int_{0}^{x}\frac{t^{n}}{1-t}\dt &\leq \frac{1}{1-x}\int_{0}^{x}t^{n}\dt \\
&\leq \frac{1}{1-x}\frac{1}{n+1}x^{n+1}\leq \frac{1}{1-x}\frac{1}{n+1} \to 0
\end{align*}
et donc \[ \lim_{n\to+\infty} \int_{0}^{x}\frac{t^{n}}{1+t}\dt = 0. \]
\item Pour $1\leq x < 0$, on a $1\leq x \leq t \leq 0$ :
\begin{align*}
\abs{\int_{0}^{x}\frac{t^{n}}{1-t}\dt} &\leq \int_{x}^{0}\frac{\abs{t}^{n}}{1-t}\dt \\
&\leq \int_{x}^{0}\abs{t}^{n}\dt 
\int_{x}^{0}\abs{t}^{n}\dt &= (-1)^{n}\int_{x}^{0}t^{n}\dt \\
&= \frac{(-1)^{n}}{n+1}[ 0 -x^{n+1}] \\
&= \frac{(-1)^{n+1}x^{n+1}}{n+1}\\
&= \frac{\abs{x}^{n}}{n+1} \leq \frac{1}{n+1}\to 0.
\end{align*}
Et donc on a aussi une limite nulle.
\end{enumerate}
Finalement, on peut conclure que \[\lim_{n\to +\infty} \int_{0}^{x}\frac{t^{n}}{1+t}\dt. \]Ainsi, les sommes partielles $\sum_{k=1}^{n}x^{k}/k$ ont pour limite $-\log(1-x)$. La série converge donc \[ \sum_{n=1}^{+\infty}\frac{x^{n}}{n} = -\log(1-x).\]
\end{itemize}

\paragraph{Remarque}Posons une suite $(a_n)_{n\in \N}$. On considère la série $\sum u_n$ de terme général $u_n = a_n - a_{n+1}$. On a \[ \sum_{k=0}^{n} u_k = \sum_{k=0}^{n}(a_k - a_{k+1}) = a_0 - a_{n+1}. \]Ainsi $\sum u_n$ converge si, et seulement si, $\lim_{n\to +\infty} \sum_{k=0}^{n}u_k$ existe, c'est-à-dire si, et seulement si, $\lim_{n\to +\infty} a_n$ existe.

\paragraph{Exemple}On regarde la série \[\sum_{n\geq 1}\frac{1}{n(n+1)}.\]On a \[ s_n = \sum_{k=1}^{n} \frac{1}{k(k+1)} = \sum_{k=1}^{n}\frac{1}{k}-\frac{1}{k+1} = 1 - \frac{1}{n+1}.\]Ainsi, \[ \sum_{n=1}^{+\infty}\frac{1}{n(n+1)} = 1.\]

\paragraph{Exemple -- nombres décimaux}On peut écrire un nombre réel comme $\sum_{n=n_0}^{+\infty}a_n\cdot 10^{-n}$ où $n_0\in \Z$ et $a_n \in \ens{0,1,\ldots,0}$.

\section{Opérations sur les séries}

\definition{ 
Soient $\sum u_n$, $\sum v_n$ deux séries. 
\begin{itemize}
\item La somme des séries est la série $\sum (u_n + v_n)$ de terme général $u_n + v_n$.
\item Soit $\lambda\in \R$. Le produit de $\sum u_n$ par $\lambda$ est la série $\sum \lambda u_n$ de terme général $\lambda u_n$.
\end{itemize}
}{}
\proposition{ 
On a :
\begin{enumerate}
\item Si les séries $\sum u_n$ et $\sum v_n$ convergent alors leur somme converge aussi \[ \sum_{n=0}^{+\infty} u_n + v_n = \sum_{n=0}^{+\infty} u_n + \sum_{n=0}^{+\infty} v_n.\]
\item Si la série $\sum u_n$ converge alors $\sum \lambda u_n$ aussi et \[ \sum_{n=0}^{+\infty}\lambda u_n = \lambda \sum_{n=0}^{+\infty} u_n.\]
\end{enumerate}
}{}
\demonstration{ 
Dans l'ordre :
\begin{enumerate}
\item Notons \[ U_n = \sum_{k=0}^{n}u_k \; ; \; V_n = \sum_{k=0}^{n}v_k.\]Alors \[ \lim_{n\to +\infty} U_n = \sum_{n=0}^{+\infty}u_n \; ; \; \lim_{n\to+\infty} V_n = \sum_{n=0}^{+\infty}v_n.\]Donc \[ \lim_{n\to+\infty}(U_n + V_n) = \sum_{n=0}^{+\infty} u_n + \sum_{n=0}^{+\infty}v_n.\]Par définition, $\sum_{n\geq 0}(u_n+v_n)$ converge si, et seulement si, $\sum_{k=0}^{n}(u_k+v_k) = U_n+V_n$ converge. Donc on a bien, si $U_n+V_n$ converge : \[\sum_{n=0}^{+\infty}u_n + v_n = \sum_{n=0}^{+\infty} u_n + \sum_{n=0}^{+\infty}v_n. \]
\item De même, en remarquant que $\sum \lambda u_n = \lambda \sum u_n$.
\end{enumerate}
}{}

\paragraph{Exemple}Soit $x\in \R$ et soit $P(X) = aX^{2}+ bX + c$ un polynôme. Il s'agit de montrer que la série \[ \sum_{n\geq 0}\frac{P(n)}{n!}x^{n}\]et convergente et de donner sa somme. On se ramène à une combinaison linéaire de séries exponentielles : \[ \frac{P(n)}{n!}x^{n} = au_n + (a+b) v_n + cw_n\]où 
\begin{align*}
u_n &= \frac{n(n-1)}{n!}x^{n} \\
v_n &= \frac{n}{n!}x^{n}\\
w_n &= \frac{x^{n}}{n!}.
\end{align*}
On a 
\begin{align*}
\sum_{k=0}^{n} \frac{P(k)}{k!}x^{k} &= P(0) + P(1) a + \sum_{k=2}^{n}\frac{P(k)}{k!}x^{k}\\
\sum_{k=0}^{n} \frac{P(k)}{k!}x^{k} &=c + (a+b+c)x + \sum_{k=2}^{n}\left( (ax^{2})\frac{x^{k-2}}{(k-2)!} + (a+b)x \frac{x^{k-1}}{(k-1)!} + c\frac{x^{k}}{k!}\right)\\
\sum_{n=0}^{+\infty}\frac{P(n)}{n!}x^{n} &= c + (a+b+c)x + ax^{2}e^{x} + (a+b)x(e^{x}-1) + c(e^{x} -2)\\
\sum_{n=0}^{+\infty}\frac{P(n)}{n!}x^{n} &= ax^{2}e^{x} + (a+b)xe^{x} + ce^{x}\\
\sum_{n=0}^{+\infty}\frac{P(n)}{n!}x^{n} &= (ax^{2}+(a+b)x+c)e^{x}.
\end{align*}

\paragraph{Remarque}On a vu qu'une somme de deux séries convergentes est convergente. On a aussi qu'une somme d'une série convergente et d'une série divergente est divergente. En effet, supposons que $\sum u_n$ converge et que $\sum v_n$ diverge. Considérons la série $\sum w_n = \sum (u_n + v_n).$ Supposons que $\sum_{k=0}^{n}w_k$ converge alors $\sum u_n$ et $\sum w_k$ convergent. Or, $\sum v_n = \sum (u_n - w_n)$ et ne peut converger.
D'où :
\proposition{ 
Si $\sum u_n$ converge et $\sum v_n$ diverge alors $\sum u_n + v_n$ diverge.
}{}
\paragraph{Remarque}Une somme de deux séries divergentes peut converger ou diverger. En effet, considérons $\sum 1/n = \sum u_n$, c'est une série divergente. Cependant, $\sum u_n + \sum u_n$ diverge aussi, mais $\sum u_n - \sum u_n$ converge.

\definition{Soient $\sum a_n$ et $\sum b_n$ des séries numériques. Considérons la série $\sum u_n$ de terme général $u_n = a_n + ib_n \in \C$. On définit la convergence de $\sum u_n$ en disant qu'elle converge si, et seulement si, $\left(\sum_{k=0}^{n}u_k\right)$ converge, i.e. si, et seulement si, $\sum a_n$ et $\sum b_n$ convergent. 
}{}

\section{Critères de convergence}

\subsection{Convergence des séries à terme positif}

Soit $\sum u_n$ telle que $u_n\in \R_+$ pour tout $n$. On se demande à quelle condition la série $\sum u_n$ converge. Posons \[ s_n =\sum_{k=0}^{n}u_n.\] La suite $(s_n)_{n\in \N}$ ainsi définie est croissante. Ainsi, $(s_n)_{n\in \N}$ est convergente à l'unique condition qu'elle soit majorée. On a ainsi : 

\proposition{ 
Une série de terme général positif converge si, et seulement si, la suite des sommes partielles est majorée.
}{}


\theoreme{ 
Soient $\sum u_n$, $\sum v_n$ des séries telles  que \[ \forall n\in \N,\; 0\leq u_n \leq v_n.\]
\begin{itemize}
\item Si la série $\sum u_n$ diverge, alors $\sum v_n$ aussi.
\item Si la série $\sum v_n$ converge, alors $\sum u_n$ aussi et sa somme est majorée par celle de $\sum v_n$.
\end{itemize}
}{De comparaison}
\demonstration{ 
Notons 
\begin{align*}
U_n &= \sum_{k=0}^{n}u_k \\
V_n &= \sum_{k=0}^{n} v_k.
\end{align*}
La proposition nous dit que $\sum u_n$ converge si, et seulement si, $(U_n)_{n\in \N}$ est majorée et de même pour $\sum v_n$. Ainsi, si $\sum u_n$ diverge alors $(U_n)_{n\in \N}$ est non bornée et donc $(V_n)_{n\in \N}$ non plus et donc $\sum v_n$ diverge.

Si $\sum v_n$ converge alors $(U_n)_{n\in \N}$ étant majorée par $(V_n)_{n\in \N}$ qui est majorée par un réel, donc $\sum u_n$ converge.

D'autre part, dans le second cas, on a :
\begin{align*}
\sum_{n=0}^{+\infty}u_n &= \lim_{n\to+\infty} U_n \\
&\leq \lim_{n\to +\infty} V_n\\
&\leq \sum_{n=0}^{+\infty} v_n.
\end{align*}
}{}

\corollaire{ 
Soient $\sum u_n$, $\sum v_n$ des séries de termes généraux $u_n$ et $v_n$ strictement positifs. Alors si la suite $(u_n/v_n)_{n\in \N}$ a une limite finie non nulle alors on a $\sum u_n$ converge si, et seulement si, $\sum v_n$ converge.
}{}
\demonstration{ 
Soit $l = \lim_{n\to +\infty} u_n/v_n$ avec $l\in \R^{*}$. Comme pour tout $n$, $u_n/v_n >0$, on sait que $l>0$. Fixons $a,b$ tels que $0<a<l<b$. Par convergence, il existe $n_0$ tel que pour tout $n\geq n_0$ on ait $a<u_n/v_n<b$, c'est-à-dire $av_n < u_n < bv_n$. Par le théorème de comparaison des séries, $u_n$ converge si, et seulement si, $v_n$ converge.
}{}

\paragraph{Exemple 1}On considère la série de terme général \[\forall n >0, \; u_n = \frac{1}{n^{2}}.\]
On a vu que la série de terme général $v_n = 1/[n(n+1)]$ pour tout $n>0$, converge. En effet $v_n = \frac{1}{n} - \frac{1}{n+1}$ et $\lim a_n= 0$ donc par le théorème de comparaison, $\sum v_n$ converge. On sait que \[\forall n > 1, \; v_{n-1} > u_n \]et \[ \frac{u_n}{v_n} = \frac{n+1}{n} \underset{n\to +\infty}{\longrightarrow} 1.\]Ainsi par le corollaire, $\sum u_n$ converge.

\paragraph{Exemple 2}Considérons la série de terme général \[u_n = \sin\left(\frac{\pi}{2^{n}}\right). \]
\begin{align*}
&\forall n \geq 1, \; 0 < \frac{\pi}{2^{n}}\leq \frac{\pi}{2}\\
&\forall n \geq 1, \; u_n \geq 0.
\end{align*}
De plus \[\forall x \in \R, \; \abs{\sin x}\leq \abs{x}\]et donc \[u_n = \abs{u_n}\leq \frac{\pi}{2^{n}}. \]La série de terme général $v_n$ est une série géométrique de raison $1/2$. Donc elle converge et donc $\sum u_n$ converge.

\paragraph{Exemple 3}Soit $(x_{n})_{n\in \N}$ une suite d'entiers tels que $0\leq x_n\leq 0$ pour tout $n\geq 0$. Considérons la série $\sum_{n\geq 0}x_n10^{-n}$. Le terme général est positif et \[ 0\leq \frac{x_n}{10^{n}} \leq 10^{1-n}\]et $10^{1-n}$ est le terme général d'une série géométrique de raison $1/10$. Cette série converge donc et donc $(x_n)_{n\in \N}$ aussi.


\subsection{Séries de \textsc{Riemann}}
Soit $\alpha\in \R_+^{*}$. On considère la série de terme général $1/n^{\alpha}$.

\proposition{ 
La série de terme général $1/n^{\alpha}$ avec $\alpha>1$ converge. Elle diverge si $\alpha\leq 1$.
}{}
\demonstration{ 
Si $\alpha\leq 1$ alors pour tout $n\geq 1$, $n^{\alpha}\leq n$ et donc \[ \frac{1}{n^{\alpha}}\geq \frac{1}{n}\]or la série de terme général $1/n$ diverge.

Si $\alpha > 1$, on considère l'application \[f : x \donne -\frac{1}{x^{\alpha-1}}. \]De plus, $\alpha-1>0$. On a : \[f(n+1) - f(n) = \frac{1}{n^{\alpha-1}} - \frac{1}{(n+1)^{\alpha-1}}. \]Par le théorème des accroissements finis sur l'intervalle $[n,n+1]$ : \[f(n+1) - f(n) = f'(c) \]avec $c\in]n,n+1[$. Comme \[ f'(x) = \frac{\alpha-1}{x^{\alpha}}\]on a \[ f'(c) \geq \frac{\alpha-1}{(n+1)^{\alpha}} \]et donc \[ \frac{1}{n^{\alpha-1}} - \frac{1}{(n+1)^{\alpha-1}} \geq \frac{\alpha-1}{(n+1)^{\alpha}}.\]On pose \[ v_n = \frac{1}{n^{\alpha-1}} - \frac{1}{(n+1)^{\alpha-1}}.\]
Les sommes partielles de $\sum v_n$ sont \[ \sum_{n=1}^{k}v_k = \frac{1}{1	^{\alpha-1}} - \frac{1}{(k+1)^{\alpha-1}} \underset{k\to+\infty}{\longrightarrow} 1.\]Donc la série $\sum_{n\geq 1} v_n$ converge, de somme $1$.

On applique le théorème de comparaison, la série $\sum_{n\geq 1}1/n^{\alpha}$ converge.
}{}

\subsection{Convergence absolue}

\definition{ 
Soit $\sum u_n$ une série de terme général $u_n$. Si la série de terme général $\abs{u_n}$ est convergente, on dit que $\sum u_n$ est \textit{absolument convergente}.
}{}

\paragraph{Remarque}Dans la définition, on peut prendre $u_n\in \R$ avec la valeur absolue ou $u_n\in \C$ avec le module.

\theoreme{ 
Toute série absolument convergente est convergente.
}{}
%\demonstration{ 
%On a par l'inégalité triangulaire pour tout $k\geq 0$ : \[\abs{\sum_{n=0}^{k} u_n}\leq \sum_{n=0}^{k}\abs{u_n} \]et donc si le terme de droite converge, $s_n$ aussi.
%}{}
\demonstration{ 
Soit $u_n\in \R$. On considère $v_n = \abs{u_n} -u_n$. Par l'inégalité triangulaire : \[ 0\leq v_n \leq 2\abs{u_n}.\]Par hypothèse $\sum 2\abs{u_n} = 2\sum\abs{u_n}$ converge. Donc $\sum v_n$ converge par le théorème de comparaison. Comme $u_n = \abs{u_n} - v_n$ on a que $\sum u_n$ converge.

Si $u_n\in \C$ alors en posant $u_n = a_n +ib_n$ avec $a_n,b_n\in \R$ on a\[ 0\leq \abs{a_n},\abs{b_n}\leq \abs{u_n}.\]Comme $\sum\abs{u_n}$ converge, $\sum \abs{a_n}$ et $\sum \abs{b_n}$ aussi. Donc $\sum a_n$ et $\sum b_n$ converge et donc $\sum u_n$ aussi.
}{}

\proposition{ 
Soit $\sum u_n$ une série absolument convergente. Alors \[ \abs{\sum_{n=0}^{+\infty}u_n} \leq \sum_{n=0}^{+\infty}\abs{u_n}.\]
}{}
\demonstration{ 
Pour tout $k\geq 0$ : \[ \abs{\sum_{n=0}^{k}u_n}\leq \sum_{n=0}^{k}\abs{u_n}\]or $\abs{\sum u_n}$ et $\sum \abs{u_n}$ convergent et donc l'égalité tient pour $k=+\infty$.
}{}

\paragraph{Remarque}Si $\sum u_n$ et $\sum v_n$ sont absolument convergentes alors elles sont convergentes et donc leur $\sum u_n+v_n$ aussi. Mieux, $\sum u_n+v_n$ est absolument convergente. En effet $\abs{u_n + v_n}\leq \abs{u_n} + \abs{v_n}$ et comme $\sum \abs{u_n} + \abs{v_n}$ est convergente, $\sum \abs{u_n + v_n}$ est convergente.

\paragraph{Exemple}Considérons la série de terme général \[ \frac{\cos(nx)}{n^{\alpha}}\]avec $\alpha\in \R$ et $x\in \R$.
\begin{align*}
\forall n\geq 1, \; \abs{\frac{\cos nx}{n^{\alpha}}}\leq \frac{1}{n^{\alpha}}
\end{align*}
\begin{itemize}
\item si $\alpha>1$ alors le théorème de comparaison conclut ;
\item si $\alpha=1$ et $x=0$ alors le terme général est $1/n$ et la série diverge ;
\item si $\alpha=1$ et $x = \pi$ alors le terme général est $(-1)^{n}/n$ et alors la série converge (mais pas en valeur absolue).
\end{itemize}

\paragraph{Exemple}Soit \[ u_n = (-1)^{n}\frac{\sqrt{n+2} - \sqrt{n}}{n}, \; \abs{u_n} = \frac{\sqrt{n+2}-\sqrt{n}}{n} = \frac{2}{n\sqrt{n+2} + n\sqrt{n}}.\]On a donc \[\abs{u_n} \leq \frac{1}{n\sqrt{n}}. \]C'est une série de \textsc{Riemann} avec $\alpha = 3/2$ et donc la série $\sum u_n$ converge absolument.

\subsection{Comparaison avec des séries géométriques}

\theoreme{ 
Soit $\sum u_n$ une série de terme général $u_n>0$. S'il existe $K\in \R$ tel que $K<1$ et pour tout $n$, $u_{n+1}/u_n \leq K$ alors $\sum u_n$ converge.
}{Règle de \textsc{d'Alembert}}
\demonstration{ 
On a : \[ u_n \leq K^{n}u_0\]et donc comme $\sum K^{n}u_0$ converge, par le théorème de comparaison, $\sum u_n$ converge.
}{}

\corollaire{ 
Soit $\sum u_n$ la série dont le terme général, $u_n$, est strictement positif.

Supposons que $\lim_{n\to+\infty}u_{n+1}/u_n = l$.
\begin{itemize}
\item Si $l>1$ alors $\sum u_n$ diverge ;
\item Si $l < 1$ alors $\sum u_n$ converge.
\end{itemize}
}{}
\demonstration{ 
Supposons $l<1$. Par définition, il existe $n_0$ et $K$ entiers tels que $l<K<1$ et pour tout $n\geq n_0$, $u_{n+1}/u_n \leq K$. Donc $\sum u_{n+n_0}$ converge et donc $\sum u_n$ aussi.

Supposons $l>1$. Il existe $n_0$ et $K>1$ tels que  pour tout $n\geq n_0$ on a $u_{n+1}/u_n\geq K$. Donc $u_{n+1}\geq Ku_n >u_n$ et donc $\sum u_n$ est grossièrement divergente.
}{}

\paragraph{Exemple}Soit $x\in \R$. On pose \[u_n = n^{2}x^{n}. \]Si $x=0$ alors $\sum u_n$ converge. Pour $x\neq 0$ on a \[ \frac{u_{n+1}}{u_n} = \frac{(n+1)^{2}}{n^{2}}x.\] On a \[ \lim_{n\to+\infty}\abs{\frac{u_{n+1}}{u_n}} = \abs{x} >0.\]
\begin{itemize}
\item Si $\abs{x}<1$ alors $\sum u_n$ est absolument convergente ;
\item si $\abs{x}>1$ alors $\sum u_n$ n'est pas convergente ;
\item si $\abs{x} = 1$ alors la règle de \textsc{d'Alembert} ne permet pas de conclure.
\end{itemize}

\subsection{Régèle de \textsc{Cauchy}}

\theoreme{ 
Soit $\sum u_n$ une série numérique à termes positifs. S'il existe $K<1$ tel que \[\forall n\in \N^{*},\; \sqrt[n]{u_n}\leq K\]alors la série $\sum u_n$ converge.
}{Règle de \textsc{Cauchy}}
\demonstration{ 
On a que : \[\forall n\geq 1, \; u_n\leq K^{n}. \]D'autre part, $0<K<1$ donc la série de terme général $K^{n}$ converge et donc $\sum u_n$ converge.
}{}

\corollaire{ 
Soit $\sum u_n$ de terme général positif. Supposons $\lim \sqrt[n]{u_n} = l$.
\begin{enumerate}
\item Si $l<1$ alors $\sum u_n$ converge.
\item Si $l>1$ alors $\sum u_n$ diverge.
\end{enumerate}
}{}
\demonstration{ 
Dans l'ordre :
\begin{enumerate}
\item Supposons $l<1$. Fixons $0< l <K<1$. Il existe $n_0$ tel que pour tout $n\geq n_0$, $\sqrt[n]{u_n}\leq K$. On a alors $0\leq u_n\leq K^{n}$ et donc on conclut.
\item Supposons $l>1$. Fixons $0<1<K<l$. Il existe $n_0$ tel que pour tout $n\geq n_0$, $\sqrt[n]{u_n}\geq K$. Ainsi, $u_n\geq K^{n}$ et donc comme $\sum K^{n}$ diverge, $\sum u_n$ aussi.
\end{enumerate}
}{}

\paragraph{Exemple}Si $u_n = x^{n}/n^{n}$ pour tout $n\geq 1$. Ainsi, \[\sqrt[n]{u_n} = \frac{x}{n} \underset{n\to+\infty}{\longrightarrow}0. \]Ainsi, $\sum x^{n}/n^{n}$ converge.

\subparagraph{Règle de Riemann}

On a que $\sum 1/n^{\alpha}$ converge pour $\alpha>1$. Ainsi :
\theoreme{ 
Soit $\sum u_n$ de terme général positif et soit $\alpha$ un réel strictement positif.
\begin{enumerate}
\item Si $\lim n^{\alpha}u_n$ existe et est non nulle alors la série de terme général $u_n$ converge si, et seulement si, $\alpha>1$.
\item Si $\lim n^{\alpha}u_n = 0$ et si $\alpha>1$ alors la série de terme général $u_n$ converge.
\item Si $\lim nu_n = +\infty$ alors la série $\sum u_n$ diverge.
\end{enumerate}
}{Règle de \textsc{Riemann}}
\demonstration{ 
Posons \[v_n = \frac{1}{n^{\alpha}} .\] On a \[ \frac{u_n}{v_n} = n^{\alpha}u_n.\]
\begin{enumerate}
\item Le théorème de comparaison entraine que si $\sum u_n$ et si $\sum v_n$ sont à termes généraux strictement positifs telles que la suite $u_n/v_n$ a une limite non nulle, alors $\sum u_n$ converge si, et seulement si, $\sum v_n$ converge.
\item On a alors pour $n$ assez grand $u_n \leq 1/n^{\alpha}$ et donc $\sum u_n$ converge si $\alpha >1$.
\item Pour $n$ assez grand, $nu_n \geq 1$ et donc $u_n\geq 1/n$ et donc $\sum u_n$ diverge puisque $\sum 1/n$ diverge.
\end{enumerate}
}{}

\paragraph{Exemple}Avec \[u_n = \frac{\log n}{n^{2}} \]alors, comme \[ \lim n^{\frac{3}{2}}u_n = 0\] et comme $3/2>1$, $u_n \geq 0$ on a bien que $\sum u_n$ converge.

\paragraph{Remarque}On a \[ \frac{u_{n+1}}{u_n} = \frac{\log n+1}{\log n}\frac{n^{2}}{(n+1)^{2}}\]et cela tend vers $1$ quand $n$ tend vers l'infini. Le critère de \textsc{d'Alembert} ne permettait pas de conclure.

\paragraph{Remarque}Les critères de \textsc{d'Alembert}, \textsc{Cauchy} ou \textsc{Riemann} ne sont pas valables pour les séries à termes négatifs. Si $u_n$ est à valeurs négatives, on peut appliquer ces critères à $\abs{u_n}$.

\subsection{Comparaison avec des intégrales}

Soit $f : [a,+\infty[\vers \R$ une fonction numérique. Supposons que $f(x) \geq x$ pour tout $x$ et supposons que $f$ est décroissante. Pour tous entiers $p<q$ supérieurs à $a$ on a \[ \sum_{n=p}^{q}f(n) \leq \int_{p}^{q}f(t)\dt.\]

\lemme{ 
Pour tous $a\leq p<q$ entiers. On a \[ f(q) + \int_{p}^{q}f(t)\dt \leq \sum_{n=p}^{q}f(n) \leq f(p) + \int_{p}^{q}f(t)\dt.\]
}{}
\demonstration{ 
Soit $n\in \N$ tel que $p<n<q$. Comme $f$ est décroissante, $f(n+1)\leq f(t)\leq f(n)$ pour tout $t\in [p,q]$.
\begin{align*}
f(n+1) &= \int_{n}^{n+1}f(n+1)\dt \leq \int_{n}^{n+1}f(t)\dt \leq \int_{n}^{n+1}f(n)\dt = f(n)\\
f(n+1) &\leq \int_{n}^{n+1}f(t)\dt \leq f(n).
\end{align*}
On somme alors sur $n\in [p,q[$ :
\begin{align*}
\sum_{n=p}^{q-1}f(n+1) &\leq \int_{p}^{q}f(t)\dt \leq \sum_{n=p}^{q-1}f(n)\\
f(q) + \int_{p}^{q}f(t)\dt  &\leq \sum_{n=p}^{q}f(n)\leq f(n) + \int_{p}^{q}f(t)\dt.
\end{align*}
}{}
\paragraph{Exemple}Soit $\alpha>0$. On pose \[ u_n = \frac{1}{n(\log n)^{\alpha}}.\]On a $\lim u_{n+1}/u_n = 1$. $n^{\beta}u_n$ ne converge pas pour $\beta>1$. Si $\beta\leq 1$ alors $n^{\beta}u_n$ converge vers $0$. Cependant : \[f : x \donne \frac{1}{x(\log x)^{\alpha}} \]est décroissante sur $[2,+\infty[$. De plus, $f(x)>0$ pour tout $x$ dans cet intervalle. Ainsi, considérons l'intégrale :
\begin{align*}
\int f(t)\dt &= \int \frac{1}{t(\log t)^{\alpha}}\dt .\\
\int f(t)\dt &= \int (\log t)^{-\alpha} \frac{\dt}{t}\\
\int f(t)\dt &= \systeme{& \frac{1}{1-\alpha}(\log t)^{1-\alpha}  &\text{ si }\alpha\neq 1\\ &\log(\log t) &\text{ si } \alpha = 1}.\\
\int_{2}^{x}f(t)\dt &= \systeme{& \frac{1}{1-\alpha}((\log x)^{1-\alpha} -(\log 2)^{1-\alpha})  &\text{ si }\alpha\neq 1\\ &\log(\log x) - \log(\log 2) &\text{ si } \alpha = 1}
\end{align*}
\begin{itemize}
\item Si $\alpha = 1$ : 
\begin{align*}
\lim_{x\to +\infty} \int_{2}^{x}f(t)\dt &= +\infty.
\end{align*}
Le lemme précédent nous dit que :
\[ \sum_{2\leq n \leq x}^{} f(n) \geq f(x) + \int_{2}^{x}f(t)\dt\]et donc $\sum u_n$ est divergente.
\item Si $\alpha<1$ alors $1-\alpha>0$ et donc \[ \lim_{x\to +\infty} \int_{2}^{x}f(t)\dt = +\infty. \]De même, la série $\sum u_n$ diverge.
\item Si $\alpha>1$ alors $1-\alpha<0$ et donc 
\[ \lim_{x\to+\infty}\int_{2}^{x} f(t)\dt = \frac{1}{\alpha-1}(\log 2)^{1-\alpha}.\]Le lemme dit que \[\sum_{n=2}^{x}u_n \leq f(2) + \int_{2}^{x}f(t)\dt \]et donc $\sum u_n$ converge.
\end{itemize}
Ainsi, $\sum u_n$ converge si, et seulement si, $\alpha>1$.

\subsection{Séries alternées}
\definition{ 
Une \textit{série alternée} est une série $\sum u_n$ telle que $u_n$ et $-u_{n+1}$ ont le même signe pour $n$ assez grand.
}{}

\theoreme{ 
Soit $(a_n)_{n\in \N}$ une suite de termes réels strictement positifs. Si la suite $(a_{n})_{n\in \N}$ est décroissante et a pour limite $0$ alors la série de terme général $(-1)^{n}a_n$ converge.

D'autre part, si $\sum (-1)^{n}a_n = S$ alors pour tout $n$ on a \[s_n \leq S \leq s_{n+1} \]et \[\abs{S-s_n} \leq a_{n+1}. \]
}{Critère des séries alternées}
\demonstration{ 
On considère les sous-suites :
\begin{align*}
s_{2p} &= a_0 -a_1 +\ldots + a_{2p} \\
s_{2p+1} &= a_0 - a_1+\ldots + a_{2p}-a_{2p+1}.
\end{align*}
\begin{align*}
s_{2p+2} - s_{2p} &= a_{2p+2}-a_{2p+1}\leq 0\\
s_{2p+3}-s_{2p+1} &= a_{2p+2} - a_{2p+3}\geq 0.
\end{align*}
Ainsi, les sous-suites $(s_{2p})_{p\in \N}$ et $(s_{2p+1})_{p\in \N}$ sont respectivement décroissante et croissante. D'autre part, \[ \lim s_{2p+1} - s_{2p} = \lim -a_{2p+1}= 0.\]Ces suites sont adjacentes et leurs différences tendent vers $0$. Ainsi, $(s_p)$ est convergente et donc $\sum u_n$ converge.
}{}

\paragraph{Exemple}Prenons \[ u_n = \frac{(-1)^{n}}{n^{\alpha}}\]avec $\alpha\in \R$. Elle n'est pas absolument convergente pour tout $\alpha$.
\begin{itemize}
\item $\alpha =1$ : on a vu que $\sum u_n = -\log(2)$ ;
\item $\alpha >1$ : $\sum u_n$ est absolument convergente ;
\item pour $0<\alpha<1$ : on a $a_n = 1/n^{\alpha} \geq 0$ est décroissante et de limite nulle, donc $\sum u_n$ est convergente (mais pas absolument).
\end{itemize}


\subsection{Application des développements limités à la convergence}

\paragraph{Exemple 1}Soit \[ u_n = \frac{1}{\sqrt{n}} - \sqrt{n}\sin\left(\frac{1}{n}\right).\]
On considère la fonction \[f(x) = \sqrt{x} - \frac{1}{\sqrt{x}}\sin(x). \]On a $u_n = f(1/n)$ et on regarde le développement limité de $f$ au voisinage de $0$ : 
\begin{align*}
\sin(x) &= x - \frac{x^{3}}{6} + x^{3}\eps(x) \\
f(x) &= \sqrt{x} - \frac{1}{\sqrt{x}}\left(x-\frac{x^{3}}{6} + x^{3}\eps(x)\right)\\
&= \frac{x^{5/2}}{6} + x^{5/2}\eps(x) \\
u_n = \frac{n^{-5/2}}{6} + n^{-5/2}\eps(1/n),
\end{align*}
et donc il existe $n_0$ tel que $u_n$ est du signe de $n^{-5/2}/6$ pour $n\geq n_0$. Donc on peut considérer $u_n$ comme une série de terme général positif telle que $n^{5/2}u_n$ tend vers $1/6$. Par le critère de \textsc{Riemann}, cette série converge et donc $\sum u_n$ aussi.

\paragraph{Exemple 2}Soit $a\in \R$ et soit \[ u_n = (n^{2}+1)^{a} - (n^{2}-1)^{a}.\]
\begin{itemize}
\item Si $a = 0$ alors $u_n=0$ et $\sum u_n$ converge.
\item Si $a\neq 0$, on calcule :
\begin{align*}
u_n &= n^{2a}\left( \left( 1 + \frac{1}{n^{2}}\right)^{a} - \left(1-\frac{1}{n^{2}}\right)^{a}\right) \\
u_n &= f(1/n),\\
f(x) &= \frac{1}{x^{2a}}\left( (1+ x^{2})^{a} -(1-x^{2})^{a}\right).
\end{align*}
On fait un développement limité de $f$ en $0$ :
\begin{align*}
f(x) &= x^{-2a}(1+ax^{2}-(1-ax^{2}) + x^{3}\eps(x)) \\
f(x) &= x^{-2a}(2ax^{2}+x^{3}\eps(x))\\
f(x) &= 2ax^{2(1-a)} + x^{3-2a}\eps(x).
\end{align*}
Ainsi 
\begin{align*}
u_n &= \frac{2a}{n^{2(1-a)}} + \frac{\eps(1/n)}{n^{2(1-a)+1}}
\end{align*}
Ainsi \[ \lim_{n\to\infty} n^{2(1-a)}u_n = 2a \]et $u_n$ est une suite de terme général du signe de $a$ au moins à partir d'un certain rang.
Par le critère de \textsc{Riemann} :
\begin{itemize}
\item Si $a>0$, alors $u_n$ est à terme général positif ;
\item si $a\geq 1/2$ alors $2(1-a)\leq 1$ et donc la série $\sum u_n$ est divergente ;
\item si $a<1/2$ alors $2(1-a)>1$ donc la série $\sum u_n$ est convergente.
\item Si $a<0$ alors $u_n$ est négatif. On applique le critère à $\sum -u_n$. On a $2(1-a)>2$ et donc $\sum -u_n$ est convergente et donc $\sum u_n$ est convergente.
\end{itemize}
\end{itemize}

\section{Transformation d'\textsc{Abel}}

Soient $(a_n)_{n\in \N}$ et $(b_n)_{n\in \N}$ deux suites à valeurs complexes. Soit $p\in \N$. On pose \[ \forall n\geq p, \; B_n = \sum_{k=p}^{n} b_p.\] Dans la somme (avec $q\geq p$) : \[ \sum_{n=p}^{q}a_nb_n\]on remplace les $b_n$ par $B_n - B_{n-1}$, c'est-à-dire :
\begin{align*}
\sum_{n=p}^{q}a_nb_n &= \sum_{n=p}^{q}a_n(B_n-B_{n-1})\\
&= \sum_{n=p}^{q}[a_nB_n]  - \sum_{n=p}^{q}[a_nB_{n-1}].
\end{align*}

\lemme{ 
Supposons que les $a_n$ sont des réels positifs et que $(a_n)_{n\in \N}$ est décroissante. Supposons également que les $B_n$ avec $n\geq p$ sont majorés en valeur absolue par $B$. Alors on a : \[\forall q >p, \; \abs{\sum_{n=p}^{q} a_nb_n} \leq a_p B. \]
}{}
\demonstration{ 
On a $a_n-a_{n+1}\geq 0$ puisque $(a_n)_{n\in \N}$ est décroissante. Ainsi $\abs{(a_n-a_{n+1})B_n} = (a_n-a_{n+1})B_n$. Ainsi, \[ \abs{\sum_{n=p}^{q} a_nb_n} = \abs{ \sum_{n=p}^{q-1} [(a_k - a_{k+1})B_k] + a_qB_q} \]et donc 
\begin{align*}
 \abs{\sum_{n=p}^{q} a_nb_n}  &\leq  \sum_{k=p}^{q-1}(a_k-a_{k+1})\abs{B_k} + a_q\abs{B_q} \\
 \abs{\sum_{n=p}^{q} a_nb_n}  &\leq ( \sum_{n=p}^{q-1}(a_k-a_{k+1}) + a_q)B\\
 \abs{\sum_{n=p}^{q} a_nb_n}  &\leq a_p B.
\end{align*}
}{}
\corollaire{ 
Soit $(a_n)_{n\in \N}$ une suite à termes réels positifs décroissante. Soit $x\in]0,2\pi[$, posons $z = \cos x + i\sin x$. Pour tous $p,q\in \R$ tels que $p<q$ : \[ \abs{\sum_{n=p}^{q} a_nz^{n}}\leq \frac{a_p}{\sin(x/2)} .\]
}{}
\proposition{ 
Soit $(a_n)_{n\in \N}$ une suite décroissante de réels tels que $\lim a_n = 0$. Alors la série de terme général $a_n\cos(n_x)$ est convergente pour tout $x\in \R-2\pi\Z$. De plus, la série de terme général $a_n\sin(nx)$ converge pour tout $x \in \R$.
}{}

\demonstration{ 
On considère la somme partielle : \[ \sum_{n=p}^{q}a_nz^{n}.\]On applique le lemme avec $b_n = z_n$. Il faut vérifier que pour tout $n$, $\abs{B_n} \leq B$ avec $B = 1/\sin(x/2)$ et $B_n = \sum z^{n}$ pour $p\leq n\leq q$.
\begin{align*}
B_n &= z^{p} \frac{1-z^{n-p+1}}{1-z},\\
1-z &= 1-\cos x - i\sin x \\
1-z &= 2\sin^{2}(x/2) - 2i\sin(x/2)\cos(x/2)\\
1-z &= 2\sin(x/2)(\sin(x/2)-i\cos(x/2))\\
\abs{1-z} &= 2\sin(x/2),\\
\abs{z^{p}\frac{1-z^{n-p+1}}{1-z}} &= \frac{1}{2\sin(x/2)}\abs{z^{p}}\abs{(1-z^{n-p+1})}\\
\abs{z^{p}\frac{1-z^{n-p+1}}{1-z}} &\leq \frac{1}{2\sin(x/2)}\abs{z^{p}}(1+\abs{z}^{n-p+1})\\
\abs{z^{p}\frac{1-z^{n-p+1}}{1-z}} &\leq \frac{2}{\sin(x/2)}.
\end{align*}
}{Corollaire}


%%%
\newpage
\chapter{Intégrales}
\section{Fonctions étagées}
Soit $I = [a,b]\dans \R$ un intervalle.
\definition{ 
Une fonction $f : I \vers \R$ est \textit{étagée} s'il existe une subdivision $a = x_0<x_1<\ldots< x_n = b$ de $[a,b]$ telle que $f$ est constante sur $]x_{i-1},x_{i}[$. Un telle subdivision est dite \textit{adaptée} à $f$.
}{}

\lemme{ 
Soit $f:I\vers \R$ une fonction étagée. Soit $x_0,x_1,\ldots,x_n$ une subdivision de $I$ adaptée à $f$. Posons $m_i=f(x)$ pour tout $x\in]x_{i-1},x_i[$ avec $i\in \ens{1,\ldots,n}$.

Alors le nombre $(x_1-x_0)m_1 + (x_2-x_1)m_2+\ldots+(x_n-x_{n-1})m_n$ ne dépend pas de la subdivision adaptée choisie.
}{}

\definition{ 
Soit $f : I\vers \R$ étagée et $x_0,x_1,\ldots,x_n$ une subdivision adaptée à $f$. Posons $m_i$ la valeur de $f$ sur chaque intervalle $]x_{i-1},x_i[$. La somme \[ \sum_{i=1}^{n}(x_i-x_{i-1})m_i\]s'appelle \textit{l'intégrale} de $f$ sur $I$ et se note \[\int_a^{b}f(t)\dt. \]
}{}

\paragraph{Remarque}Si $f$ est à valeurs positives et étagée, alors son intégrale est l'aire délimitée par le graphe de $f$, l'axe des abscisses et les droites d'équation $x=a$ et $x=b$. Si $f$ est constante alors son intégrale est égale à $(b-a)f(x)$ pour n'importe quel $x\in I$.

\proposition{ 
Soient $f,g$ deux fonctions étagées sur $I$.
\begin{enumerate}
\item si $\lambda\in \R$, $\lambda f+g$ est étagée et $$\int_{a}^{b}(\lambda f+g)(t)\dt = \lambda\int_{a}^{b}f(t)\dt + \int_{a}^{b}g(t)\dt \ ;$$
\item si $f(x)\geq g(x)$ pour tout $x\in I$ alors \[\int_{a}^{b}f(t)\dt \geq \int_{a}^{b}g(t)\dt \ ; \]
\item si $f$ et $g$ diffèrent en un nombre fini de points de $I$ alors leurs intégrales sont identiques ;
\item pour tout $c\in I$ : \[ \int_{a}^{b}f(t)\dt = \int_{a}^{c}f(t)\dt + \int_{c}^{b}f(t)\dt.\]
\end{enumerate}
}{}
\demonstration{ 
Soient $\ens{x_i}$ et $\ens{y_j}$ des subdivisions adaptées respectives à $f$ et $g$. Soit $\ens{z_k} = \ens{x_i}\union \ens{y_j}$, c'est une subdivision adaptée à $\lambda f + g$. Ainsi :
\begin{align*}
\int_{a}^{b}(\lambda f + g)(t)\dt &= \sum_{p=1}^{k}(z_{p-1}-z_p)\cdot(\lambda f + g)_p\\
\int_{a}^{b}(\lambda f + g)(t)\dt &= \sum_{p=1}^{k}(z_{p-1}-z_p)\cdot (\lambda f_p + g_p) \\
\int_{a}^{b}(\lambda f + g)(t)\dt &= \sum_{p=1}^{k}\lambda (z_{p-1}-z_p)f_p + \sum_{p=1}^{k}(z_{p-1}-z_p)g_p\\
\int_{a}^{b}(\lambda f + g)(t)\dt &= \lambda\int_{a}^{b}f(t)\dt + \int_{a}^{b}g(t)\dt.
\end{align*}

Remarquons enfin qu'une application nulle sauf en un nombre fini de points est d'intégrale nulle.
}{}

\section{Fonctions intégrables}


Soit $I = [a,b]\dans \R$ un intervalle.

\subsection{Critère d'intégrabilité}

\definition{ 
Soit $f : I\vers \R$. On dit que $f$ est \textit{intégrable} si pour tout $\eps>0$, il existe des fonctions étagées $u,U : I \vers \R$ telles que  \[\forall x \in I, \; u(x)\leq f(x)\leq U(x) \]et \[ \int_{a}^{b}(U-u)(t)\dt < \eps.\]
}{}

\paragraph{Remarque}Une fonction étagée est intégrable avec $u=U=f$.
\bigbreak
On considère $E_-$ l'ensemble des fonctions étagées inférieures à $f$ en tout point. On considère de même $E_+$ celles qui sont supérieures à $f$ en tout point. On pose \[ A =\enstq{\int_{a}^{b}u(t)\dt}{u\in E_-} \dans \R\; ; \; B = \enstq{\int_{a}^{b}U(t)\dt}{U\in E_+}\dans \R. \]On remarque que pour tout $\alpha\in A$ et tout $\beta\in B$, on a $\alpha\leq\beta$. D'autre part, comme $f$ est intégrable : \[ \forall\eps>0,\exists\alpha\in A,\exists\beta\in B,\forall x\in I,\; 0\leq \beta(x)-\alpha(x)<\eps.\]

$A$ est majorée donc $\sup A\in\R$ et de même, $\inf B\in \R$. Par la propriété ci-dessus : \[ \sup A = \inf B .\]

\definition{ 
On appelle le réel $\sup A = \inf B$ l'intégrale de $f$ sur $[a,b]$ et on le note $$\int_{a}^{b}f(t)\dt.$$
}{}

\subsection{Propriétés de l'intégrale}

\proposition{ 
Soient $f,g:I\vers \R$ intégrables.
\begin{enumerate}
\item si $\lambda\in \R$, $\lambda f+g$ intégrable et $$\int_{a}^{b}(\lambda f+g)(t)\dt = \lambda \int_{a}^{b}f(t)\dt +\int_{a}^{b}g(t)\dt \ ;$$
\item  si $f(x)\geq g(x)$ pour tout $x\in I$ \[ \int_{a}^{b}f(t)\dt \geq \int_{a}^{b}g(t)\dt\ ;\]
\item si $f,g$ diffèrent en un nombre fini de points, leurs intégrales sont identiques ;
\item pour tout $c\in I$ : \[\int_{a}^{b}f(t)\dt = \int_{a}^{c}f(t)\dt + \int_{c}^{b}f(t)\dt. \]
\end{enumerate}
}{}
\demonstration{ 
Soient $u,U,v,V$ étagées telles que pour tout $x\in I$ : \[u(x)\leq f(x)\leq U(x) \]et\[v(x)\leq f(x)\leq V(x).\]Par définition, si \[ 0\leq \int_{a}^{b}U(t)-u(t)\dt < \eps \]et de même pour $v,V$ alors $$\lambda u + v \leq \lambda f + g \leq \lambda U + v$$ et \[0 \leq\abs{ \int_{a}^{b}(\lambda U + v)(t)\dt - \int_{a}^{b}(\lambda u +v)(t)\dt }<\abs{ \lambda +1}\eps. \]Comme $\eps$ peut être choisi arbitrairement petit, on en déduit que $\lambda f+g$ est intégrable et on a l'égalité voulue.
}{}

\corollaire{ 
Soit $f:I\vers \R$ intégrable. Si $M,m\in\R$ tels que\[\forall x\in I,\; m\leq f(x)\leq M \]alors \[ (b-a)m\leq \int_{a}^{b}f(t)\dt \leq (b-a)M.\]
}{}

\theoreme{ 
Si $f:I\vers \R$ est continue, alors elle est intégrable.
}{}

\proposition{ 
Si $f:I\vers \R$ est continue alors il existe $c\in I=[a,b]$ tel que \[ \int_{a}^{b}f(t)\dt = (b-a)f(c).\]
}{}
\demonstration{ 
Puisque $f$ est continue sur $[a,b]$, il existe $m\leq M$ tel que $f(I) = [m,M]$. Par le corollaire, \[ m \leq \frac{1}{b-a}\int_{a}^{b}f(t)\dt \leq M. \]Donc il existe $c\in [a,b]$ tel que \[  \frac{1}{b-a}\int_{a}^{b}f(t)\dt = c.\]
}{}

\theoreme{ 
Si $f$ est monotone alors elle est intégrable.
}{}
\demonstration{ 
On suppose $f$ croissante. Fixons $n\in \N^{*}$ et posons la subdivision : \[\enstq{x_i = a + i\frac{b-a}{n}}{i\in\ens{0,\ldots,n}}. \] On définit les fonctions étagées $u,U$ telles que \[ u(x) = \systeme{&f(x_i) \text{ si } x\in[x_i,x_{i+1}[ \et i\in\ens{0,\ldots,n-1} \\ &f(b) \text{ si } x=b }\]\[ V(x) = \systeme{
&f(a) \text{ si } x =a \\
&f(x_{i+1}) \text{ si }x\in ]x_i,x_{i+1}]\et  i\in\ens{0,\ldots,n-1} 
  }\]
Ainsi, \[\int_{a}^{b}U(t)\dt - \int_{a}^{b}u(t)\dt = -\frac{b-a}{n}f(x_0) + \frac{b-a}{n}f(x_n) = \frac{b-a}{n}(f(b)-f(a)). \]Donc pour $n$ assez grand, on a bien une majoration par $\eps$ et donc $f$ est intégrable.
}{}

\definition{ 
Soit $f:I\vers \R$ une fonction. On appelle \textit{primitive} de $f$ une fonction continue $F : I\vers \R$ telle que $F$ est dérivable de dérivée $F' = f$ sur $\ouv{I}$\footnotemark
}{}
\footnotetext{$\ouv{I}$ est le plus grand intervalle ouvert contenu dans $I$.}

\paragraph{Remarque}Si $F$ et $G$ sont deux primitives alors $F-G$ est constante.

\theoreme{ 
Soit $f : I\vers \R$ continue. Alors pour tout $a\in I$, la fonction $$\fonc{F}{I}{\R}{x}{\int_{a}^{x}f(t)\dt}$$ est une primitive de $F$.
}{}
\demonstration{ 
Soient $a,x\in I$. Le segment d'extrémités $a$ et $x$ est contenu dans $I$. Donc $F$ est bien définie puisque $f$ intégrable sur $I$. Soit $x_0\in \ouv{I}$.
\begin{align*}
F(x) - F(x_0) &= \int_{x_0}^{x}f(t)\dt, \\
(x-x_0)f(x_0) &= \int_{x_0}^{x}f(x_0)\dt, \\
F(x)-F(x_0) -(x-x_0)f(x_0) &= \int_{x_0}^{x}(f(t)-f(x_0))\dt \\
\abs{F(x)-F(x_0) -(x-x_0)f(x_0)}&\leq \int_{x_0}^{x}\abs{f(t)-f(x_0)}\dt
\end{align*}
Soit $\eps>0$, comme $f$ est continue, il existe $\alpha>0$ tel que pour tout $x$, $\abs{x-x_0}< \alpha\implique \abs{f(x)-f(x_0)}<\eps$.

Pour tout $t$ dans le segment d'extrémités $x$ et $x_0$ on a aussi $\abs{t-x_0}<\alpha$. Ainsi
\begin{align*}
\abs{F(x)-F(x_0) -(x-x_0)f(x_0)}&\leq \abs{x-x_0}\eps\\
\abs{\frac{F(x)-F(x_0)}{x-x_0}-f(x_0)}&\leq \eps \\
\lim_{x\to x_0}\frac{F(x)-F(x_0)}{x-x_0}&= f(x_0).
\end{align*}
}{}

\corollaire{ 
Toute fonction $f$ continue sur $I$ admet une primitive $F$ et on a \[ \forall x,y\in I , \; F(x)-F(y) = \int_{x}^{y}f(t)\dt.\]
}{}

\subsection{Primitives de fonctions usuelles}

Soit $a\in \R$.

\propt{} Si $b\neq -1$ : \[\int (t+a)^{b}\dt = \left(\frac{x+a}{b+1}\right)^{b+1} \]et si $b=-1$ : \[ \int \frac{1}{t+a}\dt = \log\abs{x+a}.\]

Avec $a\neq 0$ :

\propt{} $\int \cos(at)\dt = \sin(ax)/a$ 

\propt{} $\int \sin(at)\dt = -\cos(ax)/a$ 

\propt{} $\int \ch(at)\dt = \sh(ax)/a$ 

\propt{} $\int \sh(at)\dt = \ch(ax)/a$

\propt{} $\int \exp(at)\dt = \exp(ax)/a$ 

\propt{} $\int \dt / \cos^{2}t = \tan x$

\propt{} $\int \dt / \sin^{2}t = -\coth x$

\propt{} $\int \dt / \ch^{2}t = \tan x$ 

\propt{} $\int \dt / \sh^{2}t = -1/\tan x$

\propt{} $\int \dt /(1+t^{2}) = \arctan x$

\propt{} $\int \dt / \sqrt{t^{2}-1} = \log \abs{x + \sqrt{x^{2}-1}}$

 \propt{} $\int \dt / \sqrt{t^{2}+1} = \log \abs{x + \sqrt{x^{2}+1}}$

\propt{} $\int \dt / \sqrt{1-t^{2}} = \arcsin x$
\newpropt

\subsection{Techniques d'intégration}
\theoreme{ 
Soient $u,v$ des fonctions définies et dérivables sur un même intervalle. Alors on a : \[ \int u'(t)v(t)\dt = -\int u(t)v'(t) \dt + uv\]
}{Intégration par parties}

\theoreme{ 
Soit $u$ une fonction dérivable à valeurs dans $J= u(I)$, un intervalle de $\R$, qui ne s'annule pas. Soit $v : J\vers I$ l'inverse de $u$ (qui définit une bijection). Soit $f:I\vers \R$ continue. Alors \[ \int_a^{b}f(t)\dt = \int_{u(a)}^{u(b)}(f\rond v)(t)v'(t)\dt. \]
}{Changement de variable}

\section{Intégrales impropres}

\definition{ 
Soit $f : [a,b[ \vers \R$. Si la fonction \[x\donne \int_{a}^{x}f(t)\dt \]
est définie sur $[a,b[$ et admet une limite quand $x\to b$ avec $x<b$ alors on note cette limite \[\int_{a}^{b}f(t)\dt. \]

Si cette limite existe et appartient à $\R$, on dit que l'intégrale impropre $\int_{a}^{b}f(t)\dt$ est convergente. Sinon, elle est divergente.
}{}

\subsection{Exemples fondamentaux}
\proposition{On a : 
\begin{enumerate}
\item Soit $a\in \R^{*}_+$.
\begin{enumerate}
\item L'intégrale impropre \[ \int_{a}^{+\infty}\frac{\dt}{t^{\alpha}}\]converge si $\alpha>1$ et diverge si $\alpha\leq 1$.
\item \[ \int_{0}^{a}\frac{\dt}{t^{\alpha}}\]converge si $\alpha<1$ et diverge si $\alpha \geq 1$.
\end{enumerate}
\item Soient $a,b\in \R$ tel que $a<b$. Même résultats que ci-dessus pour l'intégrale impropre : \[ \int_{a}^{b}\frac{\dt}{(b-t)^{\alpha}}.\]
\end{enumerate}
}{}
\demonstration{ Dans l'ordre
\begin{enumerate}
\item Si $\alpha \neq 1$ alors \[\int \frac{1}{t^{\alpha}}\dt = \frac{1}{1-\alpha}t^{1-\alpha}. \] Et \[ \int_{a}^{x}\frac{1}{t^{\alpha}}\dt = \frac{1}{1-\alpha}(x^{1-\alpha} -t^{1-\alpha}). \]Si $\alpha >1$ : \[\lim_{x\to\infty}\int_{a}^{x} \frac{1}{t^{\alpha}\dt = -\frac{a^{1-\alpha}}{1-\alpha}}.\]Si $\alpha<1$ alors cette même limite vaut $+\infty$.

Pour \[ \int_{x}^{a}\frac{1}{t^{\alpha}}\dt = \frac{1}{1-\alpha}(a^{1-\alpha}-x^{1\alpha}),\]si $\alpha>1$ alors \[ \lim_{x\to 0, x>0}\int_{x}^{a}\frac{1}{t^{\alpha}}\dt = +\infty.\]Si $\alpha<1$ alors la même limite vaut $a^{1-\alpha}/(1-\alpha)$.

Si $\alpha=1$ alors \[ \int \frac{1}{t}\dt = \log \abs{t}.\]On a \[\lim_{x\to+\infty} \int_{a}^{x}\frac{\dt}{t} = +\infty \]et \[\lim_{x\to 0}\int_{x}^{a}\frac{\dt}{t} = +\infty. \]

\item Avec le changement de variable $u(t) = b-t$ on a \[ \int_{a}^{x}\frac{\dt}{(b-t)^{\alpha}} = \int_{b-x}^{b-a}\frac{-\dd u}{u^{\alpha}}\]et donc on se ramène au premier point.
\end{enumerate}
}{}

\paragraph{Remarque}Si $f$ admet un prolongement par continuité, alors l'intégrale est celle d'une fonction continue sur le segment et donc ce n'est pas une intégrale impropre.

\subsection{Méthodes}

\proposition{ 
Soit $f$ une fonction positive sur $[a,b[$. L'intégrale impropre \[\int_{a}^{b}f(t)\dt \]converge si, et seulement si, la fonction \[x\donne \int_{a}^{x}f(t)\dt\]est majorée.
}{}

\theoreme{ 
Soient $f,g : [a,b[\vers \R$ intégrables sur $[a,x]$ pour tout $x\in[a,b[$. Supposons que pour tout $x$, $0\leq f(x)\leq g(x)$. Alors :
\begin{enumerate}
\item si l'intégrale impropre $\int_{a}^{b}g(t)\dt$ converge, celle de $f$ aussi et on a \[\int_{a}^{b}f(t)\dt \leq \int_{a}^{b}g(t)\dt \ ; \]
\item si l'intégrale impropre de $f$ diverge, celle de $g$ aussi.
\end{enumerate}
}{}

\theoreme{ 
Soit $f : [a,b[\vers \R$. Si \[ \int_{a}^{b}\abs{f(t)}\dt\]converge, alors \[ \int_{a}^{b}f(t)\dt\]converge et : \[\abs{\int_{a}^{b} f(t)\dt} \leq \int_{a}^{b}\abs{f(t)}\dt.\]L'intégrale est alors absolument convergente.
}{}

\theoreme{ 
On a :
\begin{enumerate}
\item Soit $f:[a,b[\vers \R$ intégrable sur $[a,x]$ pour tout $x\in [a,b[$. Alors \[\int_{a}^{b}f(t)\dt \]est convergente si, et seulement si, \[\forall \eps >0, \exists \alpha >0, \forall (x,y)\in[a,b[^{2}, \; (\abs{b-x}< \alpha \et \abs{b-y} < \alpha) \implique \abs{\int_{x}^{y}f(t)\dt}<\eps. \]
\item Soit $f:[a,+\infty[\vers \R$ intégrable sur $[a,x]$ pour tout $x\in [a,+\infty[$. Alors \[\int_{a}^{+\infty}f(t)\dt \]est convergente si, et seulement si, \[\forall \eps >0, \exists A >0, \forall (x,y)\in]a,+\infty[^{2}, \; (x>A \et y>A) \implique \abs{\int_{x}^{y}f(t)\dt}<\eps. \]
\end{enumerate}
}{Critère de \textsc{Cauchy}}

\end{document}
























