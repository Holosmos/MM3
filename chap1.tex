\documentclass{mybourbaki}

\newcommand{\card}{\sharp}
\renewcommand{\div}{\mid}
\titre{Groupes et groupes symétriques}

\begin{document}
\begin{center}
\href{mailto:eric.vasserot@imj-prg.fr}{eric.vasserot@imj-prg.fr}
\end{center}


\section{Introduction}

\subsection{Groupe abstrait}

\definition{ 
Un groupe est la donnée d'un couple $(G,\cdot)$ où $G$ est un ensemble et $\cdot : G\fois G \vers G$ une loi de composition interne, telle que :
\begin{enumerate}
\item associativité : \[ \forall a,b,c\in G, \; (a\cdot b) \cdot c = a \cdot (b\cdot c) ;\]
\item existence de l'élément neutre $e\in G$ : \[ \forall g \in G, \; g\cdot e = e\cdot g = g ;\]
\item existence de l'inverse : \[\forall x\in G, \exists y \in G, \; x\cdot y  = y\cdot x = e. \]
\end{enumerate}
}{}

\paragraph{Notations}Pour un groupe multiplicatif on note $ab$ l'élément $a\cdot b$, l'élément neutre est noté $1$ et l'inverse de $a$ est noté de $a^{-1}$.

\demonstration{ 
Soient $e,e'$ deux éléments neutres. Alors \[ e' =e\cdot e' = e.\]

Soient $b,c$ inverses de $a$. Alors : \[ b = b\cdot a \cdot c = c.\]
}{Unicité de l'élément neutre et de l'inverse}

\subsection{Groupe commutatif}

\definition{ 
Un groupe $G$ est commutatif si la loi de composition l'est : \[\forall x,y\in G, \; xy = yx. \]
}{Groupe commutatif (ou Abélien)}
\paragraph{Notations}
En général la loi de composition d'un tel groupe est notée comme un groupe additif $(G,+)$. Le neutre est alors $0$ et l'inverse de $x$ est $-x$.

\subsection{Exemples}

\begin{itemize}
\item Le couple $(\Z,+)$ est un groupe abélien où $+$ est l'addition usuelle des entiers.
\item $(\R,+)$ et $(\Q,+)$ sont également des groupes abéliens.
\item $(\R\prive{0},\fois)$ et $(\Q\prive{0},\fois)$ sont des groupes abéliens.
\item $\GL(n,\R)$ est un groupe pour la composition de matrices en tant que loi de composition. Ce n'est pas un groupe commutatif.
\end{itemize}


\newpage
\section{Sous-groupe}

\subsection{Sous-groupe}
\definition{ 
Soit $G$ un groupe (multiplicatif) et $H\dans G$ un sous-ensemble de $G$. $H$ est un sous-groupe de $G$ si c'est un groupe avec la loi de composition et d'inverse astreintes à $H$\footnotemark.
}{Sous-groupe}
\footnotetext{C'est-à-dire si $H$ est stable par l'application $(x,y) \donne xy^{-1}$.}

\proposition{ 
Soit $G$ un groupe.

 Si $(H_i)_{i\in I}$ est une famille de sous-groupes de $G$ alors $\Inter_{i\in I}^{}H_i$ est un sous-groupe de $G$.
}{}
\definition{ 
Pour tout $i\in I$, $H_i$ vérifie la propriété de sous-groupe et donc l'intersection aussi.
}{}

\paragraph{Remarque}Généralement la réunion de sous-groupes n'est pas un sous-groupe. En effet si $x\in H_1$ et $y\in H_2$ alors il n'y a aucune raison que $xy\in \Union H_i$.

Pour une équivalence il faut rajouter une hypothèse. Si $H,K$ sont deux sous-groupes de $G$ alors $H\union K$ est un sous-groupe si, et seulement si, $H\dans K$ ou $K \dans H$.

En effet supposons $H\not\dans K$ et que $H\union K$ est un sous-groupe. Si $K\not\dans H$ alors on peut choisir $x\in K-K\inter H$ et $y\in H-K\inter H$. On a $x,y\in K\union H$ et donc par hypothèse $xy\in H \union K$ et donc il existe des inverses respectifs $x^{-1},y^{-1}$. Supposons $xy\in H$ : $H\ni (xy)y^{-1}=  xe=x\in H$ absurde.


\definition{ 
Si $G$ est un groupe et $X$ une partie de $G$ alors on appelle sous-groupe de $G$ engendré par $X$ le plus petit sous-groupe de $G$ contenant $X$. On le notera ici $\langle X \rangle$.

On a de plus si on note $\mathbb{G}$ l'ensemble des sous-groupes de $G$ : \[ \langle X \rangle  = \Inter_{H \in \mathbb{G} \et H \contient X} H.\]
}{Groupe engendré}

\paragraph{Exemple}Soit $G$ un groupe et $x \in G$. Alors : \[ \langle x \rangle = \enstq{x^{k}}{k\in \Z}.\]
En effet c'est un sous-groupe de $\langle x \rangle$ et le plus petit.

\subsection{Ordre d'un groupe et d'un élément}

\definition{ 
Si $G$ est un groupe fini, on appelle \textit{ordre de} $G$ son cardinal, on le note généralement $\abs{G}$ ou $\sharp G$.

Si $G$ est un groupe et $x\in G$ alors on appelle \textit{ordre de} $x$ le cardinal de son sous-groupe engendré (s'il est fini).

Dans le cas où le groupe en question ne serait pas fini, on dit que l'ordre est infini.
}{Ordre d'un groupe}

\paragraph{Exemples}
\begin{itemize}
\item Dans $\Z$, tous les éléments non nuls sont d'ordre infini.
\item Dans $\Z/n\Z$ pour $n\in \N^{*}$, $\Z/n\Z$ est d'ordre $n$ puisque toute classe admet un représentant dans $\ens{0,\ldots, n-1}$.
\item Ordre des éléments de $\Z/4\Z$ : 
\[ \begin{matrix}
x & \vline & \barre{0} & \vline & \barre{1} & \vline & \barre{2} & \vline & \barre{3} \\
\abs{x} & \vline & 1 & \vline & 4 & \vline & 2 & \vline & 4
\end{matrix}\]
\end{itemize}


\theoreme{ 
Pour tout groupe $G$ et tout sous-groupe $H$ de $G$, l'ordre (i.e. le cardinal) de $H$ divise l'ordre de $G$ : \[\card H \div \card G. \]
}{Théorème de \textsc{Lagrange}}
\demonstration{
Le cardinal de l'ensemble $G/H$ est appelé \textit{indice} de $H$ dans $G$ et est noté $[G:H]$. 
De plus, ses classes forment une partition de $G$ et chacune d'entre elles a le même cardinal que $H$. On a alors : \[\card G = \card H \times [G:H]. \]
}{Théorème de \textsc{Lagrange}}


\newpage
\section{Morphisme de groupes}

\subsection{Morphisme de groupes}

\definition{ 
Soient $G,H$ deux groupes. Une application $f : G \vers H$ est un morphisme de groupes si : \[\forall x,y\in G, \; f(x\cdot y) = f(x)\cdot f(y). \]
}{}

\proposition{ 
Soient $f:G \vers H$ un morphisme de groupes. Alors : 
\begin{enumerate}
\item $f(e_G) = e_H$ ;
\item $\forall x\in G, \; f(x^{-1}) = f(x)^{-1}$
\end{enumerate}
}{}

\subsection{Image et noyau}
\definition{ 
Soit $f : G \vers H$ un morphisme de groupes. On définit :
\begin{enumerate}
\item $\Ker(f) = \enstq{x\in G}{f(x) = e}$ ;
\item $\im(f) = \enstq{f(x)}{x\in G}$.
\end{enumerate}
}{}

\proposition{ 
Soit $f: G\vers H$ un morphisme de groupes.
\begin{enumerate}
\item $\Ker(f)$ et $\im(f)$ sont des sous-groupes de $G$ et $H$ respectivement ;
\item $f$ est injective si, et seulement si, $\Ker(f) = \ens{e}$ ;
\item $f$ est surjective si, et seulement si, $\im(f) = H$.
\end{enumerate}
}{}
\demonstration{ Point par point :

\begin{enumerate}
\item On a bien entendu $f(e) = e$ et $f(x)^{-1} = f(x^{-1})$ pour tout $x\in G$. Ainsi $\im(f) = f(G)$ est un sous-groupe de $H$.

Soient $x,y \in G$, alors $f(xy^{-1}) = f(x)f(y^{-1}) = ee^{-1} = e$ donc $xy^{-1}\in G$. De plus $f(e) = e$ donc $\Ker(f)$ est un sous-groupe de $G$.
\item Soient $x,y\in G$ : \[\left(f(x) = f(y) \ssi x = y \right)\ssi \left( f(xy^{-1}) = e \ssi xy^{-1} = e \right). \]
\item Par définition, si $\im(f) = H$ alors $f$ est surjective et réciproquement.
\end{enumerate}
}{}


\newpage
\section{Groupe symétrique}
\subsection{Groupe de permutations}
\definition{ 
Soit $E$ un ensemble. On définit : \[S_E = \ens{\text{bijections}\ E\vers E}. \]
La loi étant la composition des applications. Elle est associative, admet un élément neutre (application identité) et toute application admet une application inverse par définition.
}{}

\proposition{ 
Si $\card E = n$ alors $S_E$ est isomorphe (au sens de groupes) à $S_{\ens{1,2,\ldots,n}} := S_n$.
}{}
\demonstration{ 
Puisque $\card E = n$ il existe une bijection $\phi = E \vers \ens{1,2,\ldots,n}$. On considère alors l'application de $\theta : S_E \vers S_n$ définie par : $\omega \donne \phi\rond\omega\rond \phi^{-1}$. Comme $\omega,\phi$ sont des bijections, l'application $\phi\rond\omega\rond \phi^{-1}$ est une bijection. L'application  $\theta$ est bien définie.

On a :
\begin{align*}
\theta(\omega' \rond \omega) &= \phi\rond(\omega'\rond \omega)\rond \phi^{-1} \\
\theta(\omega' \rond \omega) &= \phi \rond \omega' \rond \id \rond \ \omega \rond \phi^{-1} \\
 \theta(\omega' \rond \omega) &= \theta(\omega')\rond \theta(\omega).
\end{align*}
$\theta$ est bien un morphisme de groupes.
On a $\theta^{-1}(\omega) = \phi^{-1} \rond \omega \rond \phi$ qui fait de $\theta$ une bijection.
}{}

\definition{ 
On appelle $S_n$ le \textit{groupe symétrique}.
}{Groupe symétrique}
\paragraph{Remarque}On omet la notation $\rond$. Si $\omega \in S_n$ on décrit son action sur $\ens{1,2,\ldots,n}$ par : \[ \matrice{1 & 2 & \ldots & n \\ \omega(1) & \omega(2) & \ldots & \omega(n)}.\]

\paragraph{Exemple de composition}Dans $S_4$ : \[ \matrice{1 & 2 & 3 & 4 \\ 1 & 4 & 3 & 2} \matrice{1 & 2 & 3 & 4 \\ 2 & 1 & 4 & 3} = \matrice{1 & 2 & 3 & 4 \\ 4 & 1 & 2 & 3}.\]
\subsection{Transpositions et cycles}
\definition{ 
Une \textit{transposition} de $S_n$ est une permutation qui échange deux éléments et laisse invariants les $n-2$ autres.
}{Transposition}
\paragraph{Notation}Pour tous $i,j \in \ens{1,2,\ldots,n}$ avec $i\neq j$ on note $(ij)$ la transposition : \[ (ij) : \systeme{i &\donne j \\ j &\donne i \\ k &\donne k, \; \forall k \neq i,j }. \]
\paragraph{Remarque}Une transposition est une involution. C'est à dire que l'ordre d'une transposition est $2$.

\proposition{ 
$\card S_n = n!$.
}{}
\definition{ 
On appelle \textit{cycle} de longueur $r>1$ (noté $r$-cycle) (dans $S_n$) une permutation $\omega$ telle qu'il existe $x_1,x_2,\ldots,x_r \in \ens{1,2,\ldots,n}$ vérifiant : 
\begin{enumerate}
\item $\omega(x_1) = x_2, \omega^{n}(x_1) = x_{1+n}$ avec $n < r$ ;
\item $\omega(x_r) = x_1$ ;
\item $\omega(x) = x$ si $x\not\in\ens{x_1,x_2,\ldots,x_r}$.
\end{enumerate}
}{Cycle}
\paragraph{Notation}On  note un tel cycle : $\matrice{x_1 & x_2 & \ldots & x_r}$.
\paragraph{Remarque}Les $2$-cycles sont exactement les transpositions.
\paragraph{Exemple}Dans $S_3$ : \[ \matrice{1 & 2 & 3 \\ 2 & 3 & 1} = \matrice{1 & 2 & 3} = \systeme{1 \donne 2 \\ 2 \donne 3 \\ 3 \donne 1}.\]

\subsection{Décomposition des cycles}

\definition{ 
On appelle \textit{support} du cycle $\omega$ le sous-ensemble : \[ \ens{x_1,x_2,\ldots,x_r} \dans \ens{1, 2 ,\ldots, n}. \]
}{Support}

\lemme{ 
Deux cycles de supports disjoints commutent.
}{}
\demonstration{ 
Soient : \[ \systeme{v = (x_1,x_2,\ldots,x_r) \\  w = (y_1,y_2,\ldots,y_s)}\] avec $\ens{x_1,x_2,\ldots,x_r}\inter \ens{y_1,y_2,\ldots,y_s} = \vide$.

Sur un élément extérieur du support la permutation agit comme l'identité donc deux supports disjoints impliquent que les permutations associées permutent (puisque que l'identité permute).
}{}

\lemme{ 
Un $r$-cycle est d'ordre $r$.
}{}
\demonstration{ 
Soit $w = \matrice{x_1 & x_2 & \ldots & x_r}$ un $r$-cycle. Il est clair qu'un élément du support est d'ordre $r$. Les autres restent fixés par $w$ et donc $w$ est d'ordre $r$.
}{}

\proposition{ 
Toute permutation de $S_n$ est décomposable en produit de cycles de supports disjoints. Cette décomposition est unique à l'ordre des facteurs près.
}{}

\paragraph{Exemples} Soit : \[S_5 \ni \matrice{1 & 2 & 3 & 4 & 5 \\ 3&  2 & 5 & 4 & 1} = w. \] On peut décomposer $w$ : \[ \matrice{1 & 3 & 5}\matrice{2}\matrice{4} = \matrice{1 & 3 & 5}.\] \[ S_8 \ni w = \matrice{1 & 2 & 3 & 4 & 5 & 6 & 7 & 8 \\ 5 & 6& 1 & 7 & 3 & 8 & 4 & 2} = \matrice{1 & 5 & 3} \matrice{2 & 6 & 8} \matrice{4 & 7}.\]

\theoreme{ 
Le groupe symétrique est engendré par les transpositions.
}{}
\demonstration{
On procède par récurrence sur $n$.
\begin{enumerate}
\item $S_2 = \ens{1, \matrice{1 & 2}}$ est engendré par $\matrice{1 & 2}$.
\item Soit $n>2$, supposons que $S_{n-1}$ est engendré par les transpositions de $S_{n-1}$. Soit $w \in S_n$ :
\begin{enumerate}
\item Soit $w(n) = n$ et alors on décompose $w$ en cycles de tailles inférieures ou égales à $S_{n-1}$ et c'est démontré.
\item Soit $w(n) \neq n$. On pose $m = w(n)$ et soit $t = \matrice{n & m}$. On pose $v = tw$ et alors $v(n) = n$ et on lui applique le cas précédent. On a alors par unicité de la décomposition que $w$ est elle-même engendrée par des transpositions et c'est démontré.
\end{enumerate}
\end{enumerate}
}{}

\theoreme{ On a les propositions suivantes :
\begin{enumerate}
\item Si $w \in S_n$ est une permutation qui s'écrit de deux façons différentes comme produit de transpositions  : \[ w = \tau_1 \tau_2 \ldots \tau_r = \tau_1' \tau_2' \ldots \tau_{r'}',\] alors $(-1)^{r} = (-1)^{r'}$.

On appelle $(-1)^{r}$ la \textit{signature} de $w$.
\item La signature est un morphisme de groupes de $S_n \vers \ens{1,-1} \cong \Z/2\Z$.
\end{enumerate}
}{}
\demonstration{ 
Soit $w\in S_n$. On pose : 
\begin{align*}
\eps(w)  &= \prod_{1 \leq i < j \leq n} \frac{w(i) - w(j)}{i-j} \\
\eps(w) &= \frac{\prod_{1 \leq i < j \leq n} (w(i) - w(j))}{\prod_{1 \leq i < j \leq n}(i-j)}\\
\eps(w) &= \frac{N}{D}.
\end{align*}
Avec
\begin{align*}
N = \prod_{1 \leq i,j \leq n \; ; \; w^{-1}(i) < w^{-1}(j)} (i-j) = \pm D.
\end{align*}
D'où : \[ \eps(w) = \pm 1.\]
}{}
\paragraph{Exemple}$w = \matrice{1 & 2 & 3}$. On a : \[\eps(w) = \frac{ (w(1) - w(2))(w(1) - w(3))(w(2)-w(3))}{(1-2)(1-3)(2-3)} =  \frac{(2-3)(2-1)(3-1)}{(1-2)(1-3)(2-3)} = 1.\]

\lemme{ 
On a :
\begin{enumerate}
\item $\eps : S_n \vers \ens{\pm 1}$ est un morphisme de groupes ;
\item $\eps(ij) = -1$ pour tout $i\neq j$.
\end{enumerate}
}{}
\demonstration{ 
Si \[ w = \tau_1 \tau_2 \ldots \tau_r = \tau_1' \tau_2' \ldots \tau_{r'}'\] alors par le lemme : \[\eps(w) = (-1)^{r} = (-1)^{r'}. \]
}{Théorème}

\demonstration{ 
Soit $E = \enstq{(ij)}{1 \leq i < j \leq n}$. On pose : \[ f_w : \systeme{
E &\vers E \\ 
\matrice{i & j} &\donne \matrice{w(i) & w(j)} \text{ si } w(i) <w(j) \\
\matrice{i & j}& \donne \matrice{w(j) & w(i)} \text{ si } w(i) > w(j)
}.\]
$f$ est une bijection car elle est injective et l'ensemble de départ et d'arrivée ont le même cardinal qui est fini.

Donc on a : 
\begin{align*}
\eps(w) &= \frac{\prod_{1 \leq i < j \leq n}(w(i) - w(j))}{\prod_{(i,j)\in E} (w(i) - w(j))} \\
\eps(w) &= \pm 1.
\end{align*}
Pour vérifier que $\eps$ est un morphisme, on calcul $\eps(wv)$ : 
\begin{align*}
\eps(wv) &=  \prod_{(i,j) \in E} \frac{wv(i) - wv(j)}{i-j} \\
\eps(wv) &= \prod_{(i,j) \in E} \frac{wv(i) - wv(j)}{v(i) - v(j)} \prod_{(i,j) \in E} \frac{v(i) - v(j) }{i - j} \\
\eps(wv) &=  \prod_{(i,j) \in E} \frac{wv(i) - wv(j)}{v(i) - v(j)} \eps(v).
\end{align*}
On calcule :
\begin{align*}
\eps(w) &\overset{?}{=} \prod_{(i,j) \in E} \frac{wv(i) - wv(j)}{v(i) - v(j)} \\
\eps(w) &= \prod_{(i,j) \in E_1} \frac{wv(i) - wv(j)}{v(i) - v(j)}\prod_{(i,j) \in E_2} \frac{wv(i) - wv(j)}{v(i) - v(j)}
\end{align*}
Où $E_1 = \enstq{(i,j) \in E}{v(i) < v(j)}$ et $E_2 = \enstq{(i,j) \in E}{v(j) < v(i)}$ ; $E = E_1 \coprod E_2$.
\begin{align*}
\eps(w) &= \prod_{(i,j) \in E_2} \frac{wv(j) - wv(i)}{v(j) - v(i)}\prod_{(i,j) \in E_1} \frac{wv(i) - wv(j)}{v(i) - v(j)} \\
\eps(w) &= \prod_{i < j  \; ; \; v^{-1}(j) < v^{-1}(i)} \frac{w(i) - w(j)}{i - j} \prod_{i < j \; ; \; v^{-1}(i) < v^{-1}(j)}\frac{w(i)- w(j)}{i-j} \\
\eps(w) &= \prod_{i < j} \frac{w(i) - w(j)}{i - j}
\end{align*}
}{Lemme}

\end{document}














