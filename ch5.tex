\documentclass{mybourbaki}
\titre{Séries numériques}

\begin{document}
\tableofcontents
\section{Définitions}
On considère des séries  numériques, c'est-à-dire à valeurs dans $\R$.

\definition{ 
Soit $(u_n)_{n\in \N}$ une suite numérique.

On dit que la série $\Sigma u_n$ de terme général $u_n$ converge si la suite de terme général \[ s_n = \sum_{k=0}^{n}u_k\]converge.

Si la suite $s_n$ diverge, alors on dit que la série $\Sigma u_n$ de terme général $u_n$ diverge.

Les $s_n$ s'appellent les \textit{sommes partielles}.
}{}
\definition{ 
On note \[ \sum_{n=0}^{+\infty}u_n= \lim_{n\to\infty}s_n\](quand elle est définie).

On l'appelle la \textit{somme} de la série $(\Sigma u_n)$.
}{}

\paragraph{Remarque}La suite de terme général \[ s_n = \sum_{k=0}^{n}u_k\] converge si, et seulement si, la suite de terme général (pour $n_0$ fixé) \[ S_n = \sum_{k=n_0}^{n}u_k\]converge.


\proposition{ 
Si la série $\sum u_n$ converge alors la suite $(u_n)_{n\in\N}$ converge vers $0$.
}{}
\demonstration{ 
Avec \[ s_n = \sum_{k=0}^{n}u_k\]et $l$ la limite de $s_n$.
Soit $\eps>0$. Par convergence de $s_n$, il existe $n_0$ tel que pour tout $n\geq n$, $\abs{l-s_n}<\eps$. Et donc \[ \abs{s_{n+1} - s_n} = \abs{s_n+1 -l + l-s_n} \leq \abs{s_{n+1}-l} + \abs{s_n-l} < 2\eps. \] Or \[ \abs{s_{n+1}-s_n} = \abs{u_{n+1}}< 2\eps.\]
}{}

\paragraph{Exemple -- Séries géométriques}Soit $x\in \R$. On pose \[ u_n = a\cdot x^{n}.\]On a 
\begin{align*}
s_n &= \sum_{k=0}^{n}u_k\\
s_n &= a\sum_{k=0}^{n}x^{n}\\
s_n &= a\frac{1-x^{n+1}}{1-x} = \frac{a}{1-x}(1-x^{n+1}).
\end{align*}
\begin{itemize}
\item Si $\abs{x}<1$ alors $(s_n)_{n\in\N}$ converge vers $a/(1-x)$.
\item Si $\abs{x}\geq 1$ alors la série $\sum ax^{n}$ diverge.
\end{itemize}
\paragraph{Exemple -- Série exponentielle}Soit $x\in \R$. On regarde la série de terme général $x^{n}/n!$. Alors cette série a pour somme partielle : \[s_n = \sum_{k=0}^{n}\frac{x^{k}}{k!}\]et la formule de \textsc{Taylor} nous assure que $s_n$ tend vers $\exp(x)$. La série est convergente pour tout $x$ et de somme $\exp(x)$.

\paragraph{Exemple}Soit $x\in \R$. On considère la série \[ \sum_{n\geq 1}\frac{x^{n}}{n}.\]
\begin{itemize}
\item Si $\abs{x}>1$ alors la suite de terme général $x^{n}/n$ ne converge pas et donc la série ne converge pas.
\item Si $x=1$ alors les sommes partielles sont \[ s_n = \sum_{k=1}^{n}\frac{1}{k}.\]Cependant \[ s_{2n} - s_n \geq \frac{n}{2n} = \frac{1}{2}.\]Ainsi, la série $\sum 1/n$ diverge.
\item Si $-1\leq x<1$ alors pour tout $n\geq 1$, on pose \[ \fonc{f_n}{\R}{\R}{t}{1+t^{2}+\ldots+t^{n-1}}\]et pour tout $t\neq 1$ : \[f_{n}(t) = \frac{1-t^{n}}{1-t} \]et alors \[\frac{1}{1-t}= f_n(t) + \frac{t^{n}}{1-t}. \]On peut intégrer, pour tout $x\in[-1,1[$ : 
\begin{align*}
\int_{0}^{x}\frac{\dt}{1-t} &= \int_{0}^{x}f_n(t)\dt + \int_{0}^{x}\frac{t^{n}}{1-t}\dt\\
-\log(1-x) &= x + \frac{x^{2}}{2} +\ldots + \frac{x^{n}}{n} + \int_{0}^{x}\frac{t^{n}}{1+t}\dt\\
-\log(1-x) &= s_n+ \int_{0}^{x}\frac{t^{n}}{1+t}\dt.
\end{align*}
Il s'agit donc d'examiner la convergence du dernier terme. 
\begin{enumerate}
\item Pour $0\leq x < 1$, on a $0\leq t \leq x < 1$ : 
\begin{align*}
\frac{t^{n}}{1-t} &\leq \frac{t^{n}}{1-x} \\
\int_{0}^{x}\frac{t^{n}}{1-t}\dt &\leq \frac{1}{1-x}\int_{0}^{x}t^{n}\dt \\
&\leq \frac{1}{1-x}\frac{1}{n+1}x^{n+1}\leq \frac{1}{1-x}\frac{1}{n+1} \to 0
\end{align*}
et donc \[ \lim_{n\to\infty} \int_{0}^{x}\frac{t^{n}}{1+t}\dt = 0. \]
\item Pour $1\leq x < 0$, on a $1\leq x \leq t \leq 0$ :
\begin{align*}
\abs{\int_{0}^{x}\frac{t^{n}}{1-t}\dt} &\leq \int_{x}^{0}\frac{\abs{t}^{n}}{1-t}\dt \\
&\leq \int_{x}^{0}\abs{t}^{n}\dt 
\int_{x}^{0}\abs{t}^{n}\dt &= (-1)^{n}\int_{x}^{0}t^{n}\dt \\
&= \frac{(-1)^{n}}{n+1}[ 0 -x^{n+1}] \\
&= \frac{(-1)^{n+1}x^{n+1}}{n+1}\\
&= \frac{\abs{x}^{n}}{n+1} \leq \frac{1}{n+1}\to 0.
\end{align*}
Et donc on a aussi une limite nulle.
\end{enumerate}
Finalement, on peut conclure que \[\lim_{n\to \infty} \int_{0}^{x}\frac{t^{n}}{1+t}\dt. \]Ainsi, les sommes partielles $\sum_{k=1}^{n}x^{k}/k$ ont pour limite $-\log(1-x)$. La série converge donc \[ \sum_{n=1}^{\infty}\frac{x^{n}}{n} = -\log(1-x).\]
\end{itemize}

\paragraph{Remarque}Posons une suite $(a_n)_{n\in \N}$. On considère la série $\sum u_n$ de terme général $u_n = a_n - a_{n+1}$. On a \[ \sum_{k=0}^{n} u_k = \sum_{k=0}^{n}(a_k - a_{k+1}) = a_0 - a_{n+1}. \]Ainsi $\sum u_n$ converge si, et seulement si, $\lim_{n\to \infty} \sum_{k=0}^{n}u_k$ existe, c'est-à-dire si, et seulement si, $\lim_{n\to \infty} a_n$ existe.

\paragraph{Exemple}On regarde la série \[\sum_{n\geq 1}\frac{1}{n(n+1)}.\]On a \[ s_n = \sum_{k=1}^{n} \frac{1}{k(k+1)} = \sum_{k=1}^{n}\frac{1}{k}-\frac{1}{k+1} = 1 - \frac{1}{n+1}.\]Ainsi, \[ \sum_{n=1}^{\infty}\frac{1}{n(n+1)} = 1.\]

\paragraph{Exemple -- nombres décimaux}On peut écrire un nombre réel comme $\sum_{n=n_0}^{\infty}a_n\cdot 10^{-n}$ où $n_0\in \Z$ et $a_n \in \ens{0,1,\ldots,0}$.

\section{Opérations sur les séries}

\definition{ 
Soient $\sum u_n$, $\sum v_n$ deux séries. 
\begin{itemize}
\item La somme des séries est la série $\sum (u_n + v_n)$ de terme général $u_n + v_n$.
\item Soit $\lambda\in \R$. Le produit de $\sum u_n$ par $\lambda$ est la série $\sum \lambda u_n$ de terme général $\lambda u_n$.
\end{itemize}
}{}
\proposition{ 
On a :
\begin{enumerate}
\item Si les séries $\sum u_n$ et $\sum v_n$ convergent alors leur somme converge aussi \[ \sum_{n=0}^{\infty} u_n + v_n = \sum_{n=0}^{\infty} u_n + \sum_{n=0}^{\infty} v_n.\]
\item Si la série $\sum u_n$ converge alors $\sum \lambda u_n$ aussi et \[ \sum_{n=0}^{\infty}\lambda u_n = \lambda \sum_{n=0}^{\infty} u_n.\]
\end{enumerate}
}{}

\end{document}























