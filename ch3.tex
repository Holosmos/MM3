\documentclass{mybourbaki}
\titre{Développements limités}
\newcommand{\iR}{\barre{\R}}
\begin{document}

\section{Fonctions négligeables et équivalentes}

On considère des fonctions $f,g$ de $V$ dans $\R$ où $V$ est un voisinage épointé dans $\barre{\R} = \R \union \ens{\infty}$. C'est-à-dire que $V$ est de la forme $U - \ens{a}$ où $U$ est un voisinage de $a$ dans $\barre{\R}$ et $a\in \iR$.
\begin{itemize}
\item si $a = \infty$ alors $V \contient \ens{k,\infty}$ ;
\item si $a\in \R$ alors $V \contient ]k,a[ \union ]a,l[$ avec $k < a < l$ et $k,l\in \R$.
\end{itemize}

$f,g$ sont définies au voisinage de $a\in \iR$.

\subsection{Négligeable}

\definition{ 
On dit que $f$ est \textit{négligeable} devant $g$ au voisinage de $a$ s'il existe un voisinage $V$ tel qu'il existe une fonction $\eps : V \vers \R$ telle que : 
\begin{itemize}
\item $f = \eps \cdot g$  ;
\item $\lim_a \eps = 0$.
\end{itemize}
On note $f \underset{(a)}{=}\oo(g)$.
}{}
\paragraph{Remarque}On note :\[\fonc{\eps f}{V}{\R}{t}{\eps(t)f(t)}. \]
\paragraph{Exemples}Par exemple :
\begin{enumerate}
\item Si $g=1$ alors $f = \oo(1)$ si, et seulement si, $\lim_a f =0$.
\item Si $f=0$ au voisinage de $a$ alors pour toute fonction $g$ : $f = \oo(g)$.
\item Si $f$ est bornée et $\lim_a(g) = \infty$ alors $f = \oo(g)$ (on prend alors $\eps = f/g$).
\item On a $x^m \underset{(\infty)}{=}\oo(x^n)$ si, et seulement si, $m < n$.
\item Pour tous $\alpha,\beta >0$ : \[\systeme{ x^{\alpha} &\underset{\infty}{=} \oo(e^{\beta x}) \\ (\ln x)^{\alpha} &\underset{(\infty)}{=} \oo(x^{\beta})},\]car $\lim_\infty x^{\alpha}e^{-\beta x} = 0$.
\end{enumerate}

\proposition{ 
Si $f/g$ est définie dans un voisinage de $a$, alors : \[ f\underset{(a)}{=}\oo(g) \ssi \lim_a (f/g) = 0.\]
}{}
\demonstration{ 
On prend $\eps = f/g$.
}{}
\paragraph{Remarque}Il peut arriver que $f/g$ n'est pas défini dans aucun voisinage de $a$.
\paragraph{Exemples}Contre-exemples :
\begin{enumerate}
\item Avec $g(t) = \sin(1/[t-a])$, pour tout voisinage de $V$ de $a$, $g(t)$ s'annule en un point de $V$.
\item Même si le quotient n'est pas définit : $t \underset{(0)}{=} \oo(\sin(1/t))$.
\end{enumerate}

\proposition{ 
On a au voisinage de $a$ :
\begin{enumerate}
\item la propriété $\oo$ est transitive ;
\item la propriété $\oo$ est compatible avec la multiplication, i.e. : si $f =\oo(g)$ alors $fh = \oo(gh)$ ;
\item si $f = \oo(g)$ et si $h = \oo(k)$ alors $fh = \oo(gk)$.
\end{enumerate}
}{}
\demonstration{ 
Dans l'ordre :
\begin{enumerate}
\item Pour $f = \eps_1 g$ et $g = \eps_2 h$ avec $\lim_a \eps_i = 0$ alors : $f = \eps_1\eps_2 h$ et $\lim_a \eps_1\eps_2 = 0$.
\item Si $f = \eps g$, $\lim_a \eps =0$, alors $fh = \eps gh$.
\item De même.
\end{enumerate}
}{}
\paragraph{Contre-exemple}$\oo$ n'est pas compatible avec l'addition. Par exemple : $x \underset{(\infty)}{=} \oo(x^{3})$ et $x^2 \underset{(\infty)}{=} \oo(-x^3)$ n'entraine pas $x+x^2 \underset{(\infty)}{=} \oo(0)$.

\subsection{\'Equivalence}

\definition{ 
On dit que $f$ est \textit{équivalence} à $g$ au voisinage de $a$ si : $f-g \underset{(a)}{=} \oo(g)$. On note $f\underset{(a)}{\sim} g$.
}{}

\proposition{ 
Si $f/g$ est définie dans un voisinage de $a$ alors : \[f \underset{(a)}{\sim} g \ssi \lim_a f/g = 1.\]
}{}
\proposition{ 
$\underset{(a)}{\sim}$ est une relation d'équivalence.
}{}
\demonstration{ 
Par définition :
\begin{enumerate}
\item elle est réflexive : $f \underset{(a)}{\sim} f$ puisque $0 \underset{(a)}{=}\oo(f)$ ;
\item elle est symétrique si $f\underset{(a)}{\sim}g$ alors il existe $\eps$ telle que $\lim_a \eps = 0$ et $f = (1+\eps)g$, or $1/(1+\eps)$ est aussi définie au voisinage de $a$ et puisque $g = (1/[1+\eps])f$ on a \[g = (1+(1/[1+\eps] -1))f \] or en posant $\eps' = [1+\eps] -1$ on a $\lim_a \eps' = 0$ ;
\item elle est transitive : $f \underset{(a)}{\sim} g \et g \underset{(a)}{\sim} h $ implique qu'il existe $\eps_1,\eps_2$ telles que $f = (1+\eps_1)g$, $g = (1+\eps_2)h$ et donc $f = (1+\eps)h$ avec $\eps = \eps_1 + \eps_2 + \eps_1\eps_2$ et $\lim_a \eps = 0$.
\end{enumerate}
}{}

\proposition{
Si $f \underset{(a)}{\sim} g$ et si $\lim_a f$ existe alors $\lim_a g$ existe et $\lim_a g = \lim_a f$.
}{}
\demonstration{ 
Soit $\eps$ telle que $\lim_a \eps = 0$ alors puisque $f = (1+\eps)g$ on a \[\lim_a f = \lim_a(1+\eps)g = \lim_a g. \]
}{}

\proposition{ 
Le produit et le quotient (quand il est défini) d'équivalences est une équivalence.

Une puissance entière d'équivalences est une équivalence.
}{}
\demonstration{ 
Si $f =(1+\eps_1)g $et $h = (1+\eps_2)k$ alors $fh = (1+\eps)gk$ avec $\eps = \eps_1+\eps_2 +\eps_1\eps_2$.
}{}

\proposition{ 
Si $f\underset{(a)}{\sim}g$ et si $\varphi : I \vers \R$ telle que $\lim_b \varphi = a$, $b\in I$. Alors \[ f\rond \varphi \underset{(a)}{\sim} g \rond \varphi.\]
}{}
\demonstration{ 
Si $f = (1+\eps)g$ avec $\lim_a \eps = 0$. Alors \[ f\rond \varphi =(1+\eps') \cdot g\rond \varphi\]avec $\eps' = \eps\rond\varphi$ et $\lim_a \eps' = 0$.
}{}

\proposition{ 
On a :
\begin{enumerate}
\item Si $f$ est dérivable en $a$ alors si $f'(a) \neq 0$ on a $f(x) -f(a) \sim f'(a)(x-a)$.
\item Si $g$ est continue dans un voisinage épointé de $a$, alors si $f\underset{(a)}{\sim}g>0$ alors \[\int_{a}^{x}f(t)\dt \underset{(a)}{\sim} \int_{a}^{x}g(t)\dt. \]
\end{enumerate}
}{}
\demonstration{ 
Dans l'ordre :
\begin{enumerate}
\item Si $f$ est dérivable en $a$ alors : \[ \frac{f(x) - f(a)}{x-a} \underset{(a)}{\sim} f'(a)\]puisque si $\lim_a g =b\in \R^{*}$ alors $g \underset{(a)}{\sim}b$.
\item On sait que $f-g \underset{(a)}{=} \oo(g)$ et on veut : \[ \int_{x}^{a}(f-g)(t)\dt \underset{(a)}{=} \oo \left( \int_{x}^{a}g(t)\dt \right).\]

En posant $h = f-g$ on se ramène au problème : \[ h = \oo(g) \implique \int_{a}^{x}h = \oo \int_{a}^{x}g.\]
Si $h= \eps g$ et $\lim_a\eps = 0$ alors 
\begin{align*}
\int_{a}^{x}g &= \int_{a}^{x}\eps g 
\end{align*}
Or \[ \frac{\abs{\int_{x}^{a}\eps g}}{\int_{a}^{x}g} \leq \max_{[a,x]}\abs{\eps}\frac{\int_{a}^{x}g}{\int_{a}^{x}g} \underset{x\to a}{\longrightarrow} 0.\]
Donc \[\frac{\abs{\int_{a}^{x}\eps g =h}}{\abs{\int_{a}^{x}g}} \underset{x\to a}{\longrightarrow} 0.\]
\end{enumerate}
}{}

\section{Dérivées successives et formules de \textsc{Taylor}}

Soit $p\geq 0$ un entier.
\definition{ 
Soit $I$ un intervalle de $\R$ et $f:I\vers \R$.
\begin{enumerate}
\item $f\in C^{0}$ si $f$ est continue ;
\item $f \in C^{p}$ ($p\geq 1$) si $f$ est dérivable et $f' \in C^{p-1}$.
\end{enumerate}
}{}
\paragraph{Remarque}Si $f\in C^{p}$ alors les $p$-ièmes dérivées successives et $f$ sont toutes continues sur $I$. $f\in C^{\infty}$ si $f^{(p)}$ existe et est continue pour tout $p\geq 1$.

\proposition{ 
Si $f,g \in C^{p}$ alors $f+g$, $fg$, $f/g$ et $f\rond g$ (si définie) sont $C^{p}$.
}{}
\demonstration{ 
Dans l'ordre :
\begin{enumerate}
\item $(f+g)^{(p)} = f^{(p)} + g^{(p)}$ par récurrence sur $p$ ;
\item $(fg)^{(p)} = \sum_{k=0}^{p}\kpn{k}{p}f^{(k)}g^{(p-k)}$ ;
\item par récurrence sur $p$ pour $(f\rond g)^{(p)}$ en utilisant : $(f\rond g)' = (f'\rond g)g'$.
\end{enumerate}
}{}

\paragraph{Rappels sur les primitives}Si $f:I\vers \R$ est de classe $C^1$ avec $I\dans \R$ un intervalle ouvert. Alors si $f'$ est continue $f(x) -f(a) = \int_{a}^{x}f'(t)\dt$.

\subsection{Formules de \textsc{Taylor}}

Soit $I\dans \R$ un intervalle ouvert.

\theoreme{ 
Soit $f:I\vers \R$ de classe $C^k$. Alors pour tous $a,b\in I$ on a : \[ f(b) = \sum_{i=0}^{n-1}\frac{(b-a)^{i}}{i!}f^{(i)}(a) + \int_{a}^{b}\frac{(b-t)^{n-1}}{(n-1)!}f^{(n)}(t)\dt.\]
}{Formule de \textsc{Taylor} avec reste intégral}
\demonstration{ 
Par récurrence sur $n$, on note \[ (T_n) : f(b) = \sum_{i=0}^{n-1}\frac{(b-a)^{i}}{i!}f^{(i)}(a) + \int_{a}^{b}\frac{(b-t)^{n-1}}{(n-1)!}f^{(n)}(t)\dt.\]Supposons que $(T_k)$ soit vraie pour tout $k<n$. Alors par intégration par parties : 
\begin{align*}
u(t) &= -\frac{(b-t)^{k}}{k!}, \\
v(t) &= f^{(k)}(t),\\
R_k &= \int_{a}^{b}\frac{(b-s)^{k-1}}{(k-1)!}f^{(k)}(s)\dd s,
\end{align*}
on a :
\begin{align*}
R_k &= \int_{a}^{b}u'(s)v(s)\dd s\\
R_k &= [u(s)v(s)]^{b}_{a} - \int_{a}^{b}u(s)v'(s)\dd s\\
R_k &= u(b)v(b) - u(a)v(a)  + \int_{a}^{b}\frac{(b-s)^{k}}{k!}f^{(k+1)}(s)\dd s\\
R_k &= \frac{(b-a)^{k}}{k!}f^{(k)}(a) + \int_{a}^{b}\frac{(b-s)^{k}}{k!}f^{(k+1)}(s)\dd s\\
\end{align*}
On applique $(T_{n-1})$ : 
\begin{align*} 
f(b) &=f(a) + \sum_{i=0}^{n-2}\frac{(b-a)^{i}}{i!}f^{(i)}(a) + R_{n-1}  \\
f(b) &= f(a) + \sum_{i=1}^{n-2}\frac{(b-a)^{i}}{i!} + \frac{(b-a)^{n-1}}{(n-1)!}f^{(n-1)}(a) + R_n
\end{align*}
donc $(T_n)$ vraie.
}{}

\theoreme{ 
Soit $n>0$, $f:I\vers \R$ de classe $C^{n+1}$. Pour tous $a,b\in I$ avec $a\neq b$, il existe $\theta$ strictement compris en $a$ et $b$ tel que : 
\[ f(b) = \sum_{i=0}^{n}\frac{(b-a)^{i}}{i!}f^{(i)}(a) + \frac{(b-a)^{n+1}}{(n+1)!}f^{(n+1)}(\theta). \]
}{Formule de \textsc{Taylor} avec reste en $f^{(n+1)}(\theta)$}
\demonstration{
On pose $A$ telle que \[ \frac{(b-a)^{n+1}}{(n+1)!}\cdot A = \int_{a}^{b}\frac{(b-s)^{n+1}}{(n+1)!}f^{(n)}(s)\dd s - \frac{(b-a)^{n}}{n!}f^{(n)}(a).\]
Soit $F:I\vers \R$ telle que :
\[ F(x) = \int_{x}^{b}\frac{(b-t)^{n-1}}{(n-1)!}f^{(n)}(t)\dt - \frac{(b-x)^{n}}{n!}f^{(n)}(x) - \frac{(b-x)^{n+1}}{(n+1)!}A. \]
On calcule $F'(x)$ :
\begin{align*}
F'(x) &= -\frac{(b-x)^{n-1}}{(n-1)!}f^{(n)}(x) -\frac{(b-x)^{n}}{n!}f^{(n+1)}(x) +\frac{(b-x)^{n-1}}{(n-1)!}f^{(n)}(x) + \frac{(b-x)^{n}}{n!}A \\
F'(x) &= \frac{(b-x)^{n}}{n!}\left(A-f^{(n+1)}(x)\right).
\end{align*}
$F$ est dérivable donc continue sur $I$ :
\begin{align*}
F(a) &= \int_{a}^{b}\frac{(b-t)^{n-1}}{(n-1)!}f^{(n)}(t)\dt - \frac{(b-a)^{n}}{n!}f^{(n)}(a) - \frac{(b-a)^{n+1}}{(n+1)!}A =0,\\
F(b) &= 0.
\end{align*}
Par le théorème de \textsc{Rolle}, il existe $\theta$ strictement entre $a$ et $b$ tel que $F'(\theta) = 0$. C'est-à-dire :
\begin{align*}
\frac{(b-\theta)^{n}}{n!}\left(A-f^{(n+1)}(\theta)\right) = 0 \\
A &= f^{(n+1)}(\theta).
\end{align*}
On en déduit :
\[ \frac{(b-a)^{n+1}}{(n+1)!}f^{(n+1)}(\theta) = \int_{a}^{b}\frac{(b-s)^{n-1}}{(n-1)!}f^{(n)}(s)\dd s - \frac{(b-a)^{n}}{n!}f^{(n)}(a). \]
On a alors le résultat en remplaçant dans $(T_n)$.
}{}
\paragraph{Remarque}Si $\abs{f^{(n+1)}(s)}\leq M$ pour tout $s\in I$ alors \[\abs{f(b) - \sum_{i=0}^{n}\frac{(b-a)^{i}}{i!}f^{(i)}(a) }\leq M \frac{\abs{b-a}^{n+1}}{(n+1)!}. \]

\subsection{Fonctions usuelles}
\proposition{ 
Soit $n\in \N$, on regarde le développement de \textsc{Taylor} en $0$ à l'ordre $n+1$, $\forall i, \; \exp^{(i)}(0) = 1$. On prend $b=x,a=0$ :
\begin{align*}
\exp(x) &= \sum_{i=0}^{n}\frac{x^{n}}{n!}+\frac{x^{n+1}}{(n+1)!}\exp(\theta)\\
\theta &\in ]0,x[.
\end{align*}
}{Exponentielle}
\proposition{ 
La dérivée $n$-ième de $\cos(t)$ est $\cos(t+n\pi/2)$.
\[\abs{\cos(x) -\sum_{i=0}^{n}(-1)^{i+1}\frac{x^{2i}}{(2i)!}}\leq \frac{x^{2n+2}}{(2n+2)!}\]
car $\abs{\cos\theta}\leq 1$.
}{Cosinus, sinus}

\section{Développements limités}
\definition{ 
Soit $I\dans \R$ un intervalle ouvert tel que $0\in I, n\in \N$. On dit qu'une fonction $f : I\vers \R$ admet un \textit{développement limité} à l'ordre $n$ en $0$ si, et seulement s'il existe un polynôme $P$ de degré $n$ à coefficients réels tel que 
\[\lim_{x\to 0} \frac{f(x)-P(x)}{x^{n}} = 0. \]
Notons $$\eps(x) = \frac{f(x) - P(x)}{x^{n}}$$ alors \[\systeme{ f(x) &= P(x) + x^{n}\eps(x)\footnotemark, \\ \lim_{x\to 0}\eps(x) &= 0. } \]
}{}
\footnotetext{C'est-à-dire, $f(x) - P(x) = \oo(x^{n})$.}

\definition{ 
Soit $I\dans \R$ un intervalle ouvert et soit $n\in \N$. On dit qu'une fonction $f : I\vers \R$ admet un \textit{développement limité} à l'ordre $n$ en $a$ si, et seulement si, la fonction $t\donne f(t+a)$ admet un développement limité à l'ordre $n$ en $0$. C'est-à-dire si, et seulement s'il existe un polynôme de degré $n$, $P$ à coefficients réels tel que :
\[f(x) = P(x-a)+ \oo((x-a)^{n}) \]au voisinage de $a$.
}{}

\theoreme{ 
Si $f$ admet  un développement limité à l'ordre $n$ en un point $a$, alors ce développement limité est unique.
}{}
\demonstration{ On peut supposer $a=0$.
Supposons que $$f(x) = P_1(x) + x^{n}\eps_1(x) = P_2(x) + x^{n}\eps_2(x)$$ où $\lim_0 \eps_i = 0$ pour $i\in \ens{1,2}$. On a que \[ (P_1-P_2)(x) = x^{n}(\eps_1 - \eps_2)(x)\]et $(P_1-P_2)(x)$ est de la forme $r_0 + r_1x+\ldots + r_{n}x^{n}$ avec $r_0,r_1,\ldots,r_n\in \R$.

On montre par récurrence que les $r_k$ sont tous nuls.
Quand $x\to 0$ on trouve : \[r_0= 0\] et donc \[r_1x + \ldots + r_nx^{n} = x^{n}(\eps_1-\eps_2)(x). \]

Supposons que $r_0 =r_1 = r_{k-1}=0$, $k>0$. Alors 
\begin{align*}
r_k x^{k} + \ldots + r_nx^{n} &=x^{n}(\eps_1-\eps_2)(x),\\
r_k + r_{k+1}x + \ldots + r_nx^{n-k} &= x^{n-k}(\eps_1 - \eps_2)(x),
\end{align*}
$n-k\geq 0$ et donc $r_k = 0$ en passant à la limite.
}{}
\corollaire{ 
Soit $f(x) = P(x) + x^{n}\eps(x)$ le développement limité d'une fonction $f$ à l'ordre $n$ en $0$. Alors :
\begin{enumerate}
\item si $f$ est paire alors $P$ est pair ;
\item si $f$ est impaire alors $P$ est impaire.
\end{enumerate}
}{}
\demonstration{ 
\begin{align*}
f(x) &= P(x) + x^{n}\eps(x), \\
f(-x) &= P(-x) + x^{n}(-1)^{n}\eps(-x) = P(-x) + x^{n}\eps_1(x),
\end{align*}
Or comme $\eps(x) \to 0$ quand $x\to 0$ alors $\eps_1\to 0$ aussi.
\begin{enumerate}
\item si $f$ est impaire alors on a : \[f(x) = -P(-x) - x^{n}\eps_1(x) \] et comme la première et cette expression sont des développements limits de $f$ à l'ordre $n$ en $0$, par unicité on a $-P(-x) = P(x)$, c'est-à-dire $P$ impaire ;
\item si $f$ est paire, on a : \[ f(x) = P(-x) + x^{n}\eps_1(x)\] alors de même, l'unicité nous dit que $P$ est alors paire.
\end{enumerate}
}{}

\proposition{ 
Soit $f:I\vers \R$ une fonction continue en $a\in I$.
\begin{enumerate}
\item le développement limité de $f$ en $a$ à l'ordre $0$ est \[ f(x) = f(a) + \eps(x), \; \lim_{x\to a}\eps(x) = 0 \ ; \]
\item la fonction $f$ est dérivable en $a$ si, et seulement si, elle possède un développement limité à l'ordre $1$ en $a$, alors dans ce cas le développement limité est donné par : \[ f(x) = f(a) + f'(a)(x-a) + \eps(x)(x-a), \; \lim_{x\to a}\eps(x) = 0.\]
\end{enumerate}
}{}
\demonstration{ 
Dans l'ordre :
\begin{enumerate}
\item On pose $\eps(x) = f(x) - f(a)$. Comme $f$ est continue en $0$, $\eps(x)$ aussi et $\lim_{x\to a}\eps(x) = 0$.
\item Supposons que $f$ soit dérivable en $a$, c'est-à-dire : \[ \lim_{x\to a}\frac{f(x) - f(a)}{x-a} = f'(a).\]On pose \[\eps(x) = \frac{f(x)-f(a)}{x-a}-f'(a).\]On a bien $\lim_{x\to a}\eps(x) = 0$ et \[ f(x)  = f(a) + (x-a)f'(a) + (x-a)\eps(x).\]

Réciproquement, supposons que $f$ admette un développement limité : \[ f(x) = a_0 + (x-a)a_1 + (x-a)\eps(x),\]avec $\lim_{x\to a}\eps(x) = 0$. Alors, par continuité $a_0 = f(a)$ et \[\lim_{x\to a}\frac{f(x) - f(a)}{x-a} = \lim_{x\to a}a_1 + \eps(x) = a_1 =  f'(a). \]
\end{enumerate}
}{}

\section{Développement limité à l'ordre $n$ d'une fonction de classe $C^n$}
\subsection{Développements limités et primitives}
\theoreme{ 
Soit $f: I \vers \R$ une application continue. Soit $F$ une primitive de $f$. Soit $a\in I$ et supposons que $f$ admette un développement limité en $a$ à l'ordre $n$ : \[ f(x) = a_0 + a_1(x-a) + \frac{a_2}{2}(x-a)^{2} + \ldots + \frac{a_n}{n!}(x-a)^{n} + (x-a)^{n}\eps(x), \; \lim_{x\to a}\eps(x) = 0. \]
Alors $F$ admet le développement limité suivant à l'ordre $n+1$ en $a$ : \[ F(x) = F(a) + a_0(x-a) + \frac{a_1}{2}(x-a)^{2} + \ldots + \frac{a_n}{(n+1)!}x^{n+1} + (x-a)^{n+1}\eps_1(x), \; \lim_{x\to a}\eps_1(x) = 0.\]
}{}
\demonstration{ 
Soit \[ P(t) = \sum_{k=0}^{n}\frac{a_k}{k!}(t-a)^{k}.\]
Pour tout $x\neq a$ : \[ \eps(x) = \frac{f(x) -P(x)}{(x-a)^{n}}.\]Par hypothèse, $\lim_{x\to a}\eps(x)= 0$. En posant $\eps(a) = 0$, on obtient que $\eps$ est continue sur $I$. Donc $\eps$ admet une primitive et dans l'identité \[ f(x) = a_0 + a_1(x-a) + \frac{a_2}{2}(x-a)^{2} + \ldots + \frac{a_n}{n!}(x-a)^{n} + (x-a)^{n}\eps(x), \; \lim_{x\to a}\eps(x) = 0 \] tous les termes admettent des primitives. Donc
\begin{align*}
F(x) - F(a) &= \int_{a}^{x}f(t)\dt \\
 F(x) - F(a) &= \int_{a}^{x}\left( \sum_{k=0}^{n}\frac{a_k}{k!}(t-a)^{k} + (t-a)^{n}\eps(t)\right)  \dt \\
 F(x) - F(a)  &= \sum_{k=0}^{n}\frac{a_k}{(k+1)!}(x-a)^{k+1} + u(x),\\
 u(x) &= \int_{a}^{x} (t-a)^{n}\eps(t)\dt.
\end{align*}
Par le théorème de \textsc{Rolle} : \[ u(x) = (x-a)(\theta - a)^{n}\eps(\theta)\]pour un $\theta$ compris entre $a$ et $x$. Donc \[ \abs{u(x)} = \abs{x-a}\abs{\theta-a}^{n}\abs{\eps(\theta)} \leq \abs{x-a}^{n+1}\abs{\eps(\theta)}\]
et $\eps(\theta)$ tend vers $0$ quand $x$ tend vers $a$ puisque $\theta$ est compris entre $a$ et $x$.
Donc : \[F(x) = \sum_{k=0}^{n}\frac{a_k}{(k+1)!}(x-a)^{k+1}+(x-a)^{n+1}\eps_1(x) \]où \[\eps_1(x) = \frac{u(x)}{(x-a)^{n+1}} \to 0. \]
}{}
\end{document}


























