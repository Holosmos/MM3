\section{Fonctions négligeables et équivalentes}

On considère des fonctions $f,g$ de $V$ dans $\R$ où $V$ est un voisinage épointé dans $\barre{\R} = \R \union \ens{\infty}$. C'est-à-dire que $V$ est de la forme $U - \ens{a}$ où $U$ est un voisinage de $a$ dans $\barre{\R}$ et $a\in \iR$.
\begin{itemize}
\item si $a = \infty$ alors $V \contient \ens{k,\infty}$ ;
\item si $a\in \R$ alors $V \contient ]k,a[ \union ]a,l[$ avec $k < a < l$ et $k,l\in \R$.
\end{itemize}

$f,g$ sont définies au voisinage de $a\in \iR$.

\subsection{Négligeable}

\definition{ 
On dit que $f$ est \textit{négligeable} devant $g$ au voisinage de $a$ s'il existe un voisinage $V$ tel qu'il existe une fonction $\eps : V \vers \R$ telle que : 
\begin{itemize}
\item $f = \eps \cdot g$  ;
\item $\lim_a \eps = 0$.
\end{itemize}
On note $f \underset{(a)}{=}\oo(g)$.
}{}
\paragraph{Remarque}On note :\[\fonc{\eps f}{V}{\R}{t}{\eps(t)f(t)}. \]
\paragraph{Exemples}Par exemple :
\begin{enumerate}
\item Si $g=1$ alors $f = \oo(1)$ si, et seulement si, $\lim_a f =0$.
\item Si $f=0$ au voisinage de $a$ alors pour toute fonction $g$ : $f = \oo(g)$.
\item Si $f$ est bornée et $\lim_a(g) = \infty$ alors $f = \oo(g)$ (on prend alors $\eps = f/g$).
\item On a $x^m \underset{(\infty)}{=}\oo(x^n)$ si, et seulement si, $m < n$.
\item Pour tous $\alpha,\beta >0$ : \[\systeme{ x^{\alpha} &\underset{\infty}{=} \oo(e^{\beta x}) \\ (\ln x)^{\alpha} &\underset{(\infty)}{=} \oo(x^{\beta})},\]car $\lim_\infty x^{\alpha}e^{-\beta x} = 0$.
\end{enumerate}

\proposition{ 
Si $f/g$ est définie dans un voisinage de $a$, alors : \[ f\underset{(a)}{=}\oo(g) \ssi \lim_a (f/g) = 0.\]
}{}
\demonstration{ 
On prend $\eps = f/g$.
}{}
\paragraph{Remarque}Il peut arriver que $f/g$ n'est pas défini dans aucun voisinage de $a$.
\paragraph{Exemples}Contre-exemples :
\begin{enumerate}
\item Avec $g(t) = \sin(1/[t-a])$, pour tout voisinage de $V$ de $a$, $g(t)$ s'annule en un point de $V$.
\item Même si le quotient n'est pas définit : $t \underset{(0)}{=} \oo(\sin(1/t))$.
\end{enumerate}

\proposition{ 
On a au voisinage de $a$ :
\begin{enumerate}
\item la propriété $\oo$ est transitive ;
\item la propriété $\oo$ est compatible avec la multiplication, i.e. : si $f =\oo(g)$ alors $fh = \oo(gh)$ ;
\item si $f = \oo(g)$ et si $h = \oo(k)$ alors $fh = \oo(gk)$.
\end{enumerate}
}{}
\demonstration{ 
Dans l'ordre :
\begin{enumerate}
\item Pour $f = \eps_1 g$ et $g = \eps_2 h$ avec $\lim_a \eps_i = 0$ alors : $f = \eps_1\eps_2 h$ et $\lim_a \eps_1\eps_2 = 0$.
\item Si $f = \eps g$, $\lim_a \eps =0$, alors $fh = \eps gh$.
\item De même.
\end{enumerate}
}{}
\paragraph{Contre-exemple}$\oo$ n'est pas compatible avec l'addition. Par exemple : $x \underset{(\infty)}{=} \oo(x^{3})$ et $x^2 \underset{(\infty)}{=} \oo(-x^3)$ n'entraine pas $x+x^2 \underset{(\infty)}{=} \oo(0)$.

\subsection{\'Equivalence}

\definition{ 
On dit que $f$ est \textit{équivalence} à $g$ au voisinage de $a$ si : $f-g \underset{(a)}{=} \oo(g)$. On note $f\underset{(a)}{\sim} g$.
}{}

\proposition{ 
Si $f/g$ est définie dans un voisinage de $a$ alors : \[f \underset{(a)}{\sim} g \ssi \lim_a f/g = 1.\]
}{}
\proposition{ 
$\underset{(a)}{\sim}$ est une relation d'équivalence.
}{}
\demonstration{ 
Par définition :
\begin{enumerate}
\item elle est réflexive : $f \underset{(a)}{\sim} f$ puisque $0 \underset{(a)}{=}\oo(f)$ ;
\item elle est symétrique si $f\underset{(a)}{\sim}g$ alors il existe $\eps$ telle que $\lim_a \eps = 0$ et $f = (1+\eps)g$, or $1/(1+\eps)$ est aussi définie au voisinage de $a$ et puisque $g = (1/[1+\eps])f$ on a \[g = (1+(1/[1+\eps] -1))f \] or en posant $\eps' = [1+\eps] -1$ on a $\lim_a \eps' = 0$ ;
\item elle est transitive : $f \underset{(a)}{\sim} g \et g \underset{(a)}{\sim} h $ implique qu'il existe $\eps_1,\eps_2$ telles que $f = (1+\eps_1)g$, $g = (1+\eps_2)h$ et donc $f = (1+\eps)h$ avec $\eps = \eps_1 + \eps_2 + \eps_1\eps_2$ et $\lim_a \eps = 0$.
\end{enumerate}
}{}

\proposition{
Si $f \underset{(a)}{\sim} g$ et si $\lim_a f$ existe alors $\lim_a g$ existe et $\lim_a g = \lim_a f$.
}{}
\demonstration{ 
Soit $\eps$ telle que $\lim_a \eps = 0$ alors puisque $f = (1+\eps)g$ on a \[\lim_a f = \lim_a(1+\eps)g = \lim_a g. \]
}{}

\proposition{ 
Le produit et le quotient (quand il est défini) d'équivalences est une équivalence.

Une puissance entière d'équivalences est une équivalence.
}{}
\demonstration{ 
Si $f =(1+\eps_1)g $et $h = (1+\eps_2)k$ alors $fh = (1+\eps)gk$ avec $\eps = \eps_1+\eps_2 +\eps_1\eps_2$.
}{}

\proposition{ 
Si $f\underset{(a)}{\sim}g$ et si $\varphi : I \vers \R$ telle que $\lim_b \varphi = a$, $b\in I$. Alors \[ f\rond \varphi \underset{(a)}{\sim} g \rond \varphi.\]
}{}
\demonstration{ 
Si $f = (1+\eps)g$ avec $\lim_a \eps = 0$. Alors \[ f\rond \varphi =(1+\eps') \cdot g\rond \varphi\]avec $\eps' = \eps\rond\varphi$ et $\lim_a \eps' = 0$.
}{}

\proposition{ 
On a :
\begin{enumerate}
\item Si $f$ est dérivable en $a$ alors si $f'(a) \neq 0$ on a $f(x) -f(a) \sim f'(a)(x-a)$.
\item Si $g$ est continue dans un voisinage épointé de $a$, alors si $f\underset{(a)}{\sim}g>0$ alors \[\int_{a}^{x}f(t)\dt \underset{(a)}{\sim} \int_{a}^{x}g(t)\dt. \]
\end{enumerate}
}{}
\demonstration{ 
Dans l'ordre :
\begin{enumerate}
\item Si $f$ est dérivable en $a$ alors : \[ \frac{f(x) - f(a)}{x-a} \underset{(a)}{\sim} f'(a)\]puisque si $\lim_a g =b\in \R^{*}$ alors $g \underset{(a)}{\sim}b$.
\item On sait que $f-g \underset{(a)}{=} \oo(g)$ et on veut : \[ \int_{x}^{a}(f-g)(t)\dt \underset{(a)}{=} \oo \left( \int_{x}^{a}g(t)\dt \right).\]

En posant $h = f-g$ on se ramène au problème : \[ h = \oo(g) \implique \int_{a}^{x}h = \oo \int_{a}^{x}g.\]
Si $h= \eps g$ et $\lim_a\eps = 0$ alors 
\begin{align*}
\int_{a}^{x}g &= \int_{a}^{x}\eps g 
\end{align*}
Or \[ \frac{\abs{\int_{x}^{a}\eps g}}{\int_{a}^{x}g} \leq \max_{[a,x]}\abs{\eps}\frac{\int_{a}^{x}g}{\int_{a}^{x}g} \underset{x\to a}{\longrightarrow} 0.\]
Donc \[\frac{\abs{\int_{a}^{x}\eps g =h}}{\abs{\int_{a}^{x}g}} \underset{x\to a}{\longrightarrow} 0.\]
\end{enumerate}
}{}

\section{Dérivées successives et formules de \textsc{Taylor}}

Soit $p\geq 0$ un entier.
\definition{ 
Soit $I$ un intervalle de $\R$ et $f:I\vers \R$.
\begin{enumerate}
\item $f\in C^{0}$ si $f$ est continue ;
\item $f \in C^{p}$ ($p\geq 1$) si $f$ est dérivable et $f' \in C^{p-1}$.
\end{enumerate}
}{}
\paragraph{Remarque}Si $f\in C^{p}$ alors les $p$-ièmes dérivées successives et $f$ sont toutes continues sur $I$. $f\in C^{\infty}$ si $f^{(p)}$ existe et est continue pour tout $p\geq 1$.

\proposition{ 
Si $f,g \in C^{p}$ alors $f+g$, $fg$, $f/g$ et $f\rond g$ (si définie) sont $C^{p}$.
}{}
\demonstration{ 
Dans l'ordre :
\begin{enumerate}
\item $(f+g)^{(p)} = f^{(p)} + g^{(p)}$ par récurrence sur $p$ ;
\item $(fg)^{(p)} = \sum_{k=0}^{p}\kpn{k}{p}f^{(k)}g^{(p-k)}$ ;
\item par récurrence sur $p$ pour $(f\rond g)^{(p)}$ en utilisant : $(f\rond g)' = (f'\rond g)g'$.
\end{enumerate}
}{}

\paragraph{Rappels sur les primitives}Si $f:I\vers \R$ est de classe $C^1$ avec $I\dans \R$ un intervalle ouvert. Alors si $f'$ est continue $f(x) -f(a) = \int_{a}^{x}f'(t)\dt$.

\subsection{Formules de \textsc{Taylor}}

Soit $I\dans \R$ un intervalle ouvert.

\theoreme{ 
Soit $f:I\vers \R$ de classe $C^k$. Alors pour tous $a,b\in I$ on a : \[ f(b) = \sum_{i=0}^{n-1}\frac{(b-a)^{i}}{i!}f^{(i)}(a) + \int_{a}^{b}\frac{(b-t)^{n-1}}{(n-1)!}f^{(n)}(t)\dt.\]
}{Formule de \textsc{Taylor} avec reste intégral}
\demonstration{ 
Par récurrence sur $n$, on note \[ (T_n) : f(b) = \sum_{i=0}^{n-1}\frac{(b-a)^{i}}{i!}f^{(i)}(a) + \int_{a}^{b}\frac{(b-t)^{n-1}}{(n-1)!}f^{(n)}(t)\dt.\]Supposons que $(T_k)$ soit vraie pour tout $k<n$. Alors par intégration par parties : 
\begin{align*}
u(t) &= -\frac{(b-t)^{k}}{k!}, \\
v(t) &= f^{(k)}(t),\\
R_k &= \int_{a}^{b}\frac{(b-s)^{k-1}}{(k-1)!}f^{(k)}(s)\dd s,
\end{align*}
on a :
\begin{align*}
R_k &= \int_{a}^{b}u'(s)v(s)\dd s\\
R_k &= [u(s)v(s)]^{b}_{a} - \int_{a}^{b}u(s)v'(s)\dd s\\
R_k &= u(b)v(b) - u(a)v(a)  + \int_{a}^{b}\frac{(b-s)^{k}}{k!}f^{(k+1)}(s)\dd s\\
R_k &= \frac{(b-a)^{k}}{k!}f^{(k)}(a) + \int_{a}^{b}\frac{(b-s)^{k}}{k!}f^{(k+1)}(s)\dd s\\
\end{align*}
On applique $(T_{n-1})$ : 
\begin{align*} 
f(b) &=f(a) + \sum_{i=0}^{n-2}\frac{(b-a)^{i}}{i!}f^{(i)}(a) + R_{n-1}  \\
f(b) &= f(a) + \sum_{i=1}^{n-2}\frac{(b-a)^{i}}{i!} + \frac{(b-a)^{n-1}}{(n-1)!}f^{(n-1)}(a) + R_n
\end{align*}
donc $(T_n)$ vraie.
}{}

\theoreme{ 
Soit $n>0$, $f:I\vers \R$ de classe $C^{n+1}$. Pour tous $a,b\in I$ avec $a\neq b$, il existe $\theta$ strictement compris en $a$ et $b$ tel que : 
\[ f(b) = \sum_{i=0}^{n}\frac{(b-a)^{i}}{i!}f^{(i)}(a) + \frac{(b-a)^{n+1}}{(n+1)!}f^{(n+1)}(\theta). \]
}{Formule de \textsc{Taylor} avec reste en $f^{(n+1)}(\theta)$}
\demonstration{
On pose $A$ telle que \[ \frac{(b-a)^{n+1}}{(n+1)!}\cdot A = \int_{a}^{b}\frac{(b-s)^{n+1}}{(n+1)!}f^{(n)}(s)\dd s - \frac{(b-a)^{n}}{n!}f^{(n)}(a).\]
Soit $F:I\vers \R$ telle que :
\[ F(x) = \int_{x}^{b}\frac{(b-t)^{n-1}}{(n-1)!}f^{(n)}(t)\dt - \frac{(b-x)^{n}}{n!}f^{(n)}(x) - \frac{(b-x)^{n+1}}{(n+1)!}A. \]
On calcule $F'(x)$ :
\begin{align*}
F'(x) &= -\frac{(b-x)^{n-1}}{(n-1)!}f^{(n)}(x) -\frac{(b-x)^{n}}{n!}f^{(n+1)}(x) +\frac{(b-x)^{n-1}}{(n-1)!}f^{(n)}(x) + \frac{(b-x)^{n}}{n!}A \\
F'(x) &= \frac{(b-x)^{n}}{n!}\left(A-f^{(n+1)}(x)\right).
\end{align*}
$F$ est dérivable donc continue sur $I$ :
\begin{align*}
F(a) &= \int_{a}^{b}\frac{(b-t)^{n-1}}{(n-1)!}f^{(n)}(t)\dt - \frac{(b-a)^{n}}{n!}f^{(n)}(a) - \frac{(b-a)^{n+1}}{(n+1)!}A =0,\\
F(b) &= 0.
\end{align*}
Par le théorème de \textsc{Rolle}, il existe $\theta$ strictement entre $a$ et $b$ tel que $F'(\theta) = 0$. C'est-à-dire :
\begin{align*}
\frac{(b-\theta)^{n}}{n!}\left(A-f^{(n+1)}(\theta)\right) = 0 \\
A &= f^{(n+1)}(\theta).
\end{align*}
On en déduit :
\[ \frac{(b-a)^{n+1}}{(n+1)!}f^{(n+1)}(\theta) = \int_{a}^{b}\frac{(b-s)^{n-1}}{(n-1)!}f^{(n)}(s)\dd s - \frac{(b-a)^{n}}{n!}f^{(n)}(a). \]
On a alors le résultat en remplaçant dans $(T_n)$.
}{}
\paragraph{Remarque}Si $\abs{f^{(n+1)}(s)}\leq M$ pour tout $s\in I$ alors \[\abs{f(b) - \sum_{i=0}^{n}\frac{(b-a)^{i}}{i!}f^{(i)}(a) }\leq M \frac{\abs{b-a}^{n+1}}{(n+1)!}. \]

\subsection{Fonctions usuelles}
\proposition{ 
Soit $n\in \N$, on regarde le développement de \textsc{Taylor} en $0$ à l'ordre $n+1$, $\forall i, \; \exp^{(i)}(0) = 1$. On prend $b=x,a=0$ :
\begin{align*}
\exp(x) &= \sum_{i=0}^{n}\frac{x^{n}}{n!}+\frac{x^{n+1}}{(n+1)!}\exp(\theta)\\
\theta &\in ]0,x[.
\end{align*}
}{Exponentielle}
\proposition{ 
La dérivée $n$-ième de $\cos(t)$ est $\cos(t+n\pi/2)$.
\[\abs{\cos(x) -\sum_{i=0}^{n}(-1)^{i+1}\frac{x^{2i}}{(2i)!}}\leq \frac{x^{2n+2}}{(2n+2)!}\]
car $\abs{\cos\theta}\leq 1$.
}{Cosinus, sinus}

\section{Développement limité à l'ordre $n$ d'une fonction de classe $C^n$}
\subsection{Développements limités}
\definition{ 
Soit $I\dans \R$ un intervalle ouvert tel que $0\in I, n\in \N$. On dit qu'une fonction $f : I\vers \R$ admet un \textit{développement limité} à l'ordre $n$ en $0$ si, et seulement s'il existe un polynôme $P$ de degré $n$ à coefficients réels tel que 
\[\lim_{x\to 0} \frac{f(x)-P(x)}{x^{n}} = 0. \]
Notons $$\eps(x) = \frac{f(x) - P(x)}{x^{n}}$$ alors \[\systeme{ f(x) &= P(x) + x^{n}\eps(x)\footnotemark, \\ \lim_{x\to 0}\eps(x) &= 0. } \]
}{}
\footnotetext{C'est-à-dire, $f(x) - P(x) = \oo(x^{n})$.}

\definition{ 
Soit $I\dans \R$ un intervalle ouvert et soit $n\in \N$. On dit qu'une fonction $f : I\vers \R$ admet un \textit{développement limité} à l'ordre $n$ en $a$ si, et seulement si, la fonction $t\donne f(t+a)$ admet un développement limité à l'ordre $n$ en $0$. C'est-à-dire si, et seulement s'il existe un polynôme de degré $n$, $P$ à coefficients réels tel que :
\[f(x) = P(x-a)+ \oo((x-a)^{n}) \]au voisinage de $a$.
}{}

\theoreme{ 
Si $f$ admet  un développement limité à l'ordre $n$ en un point $a$, alors ce développement limité est unique.
}{}
\demonstration{ On peut supposer $a=0$.
Supposons que $$f(x) = P_1(x) + x^{n}\eps_1(x) = P_2(x) + x^{n}\eps_2(x)$$ où $\lim_0 \eps_i = 0$ pour $i\in \ens{1,2}$. On a que \[ (P_1-P_2)(x) = x^{n}(\eps_1 - \eps_2)(x)\]et $(P_1-P_2)(x)$ est de la forme $r_0 + r_1x+\ldots + r_{n}x^{n}$ avec $r_0,r_1,\ldots,r_n\in \R$.

On montre par récurrence que les $r_k$ sont tous nuls.
Quand $x\to 0$ on trouve : \[r_0= 0\] et donc \[r_1x + \ldots + r_nx^{n} = x^{n}(\eps_1-\eps_2)(x). \]

Supposons que $r_0 =r_1 = r_{k-1}=0$, $k>0$. Alors 
\begin{align*}
r_k x^{k} + \ldots + r_nx^{n} &=x^{n}(\eps_1-\eps_2)(x),\\
r_k + r_{k+1}x + \ldots + r_nx^{n-k} &= x^{n-k}(\eps_1 - \eps_2)(x),
\end{align*}
$n-k\geq 0$ et donc $r_k = 0$ en passant à la limite.
}{}
\corollaire{ 
Soit $f(x) = P(x) + x^{n}\eps(x)$ le développement limité d'une fonction $f$ à l'ordre $n$ en $0$. Alors :
\begin{enumerate}
\item si $f$ est paire alors $P$ est paire ;
\item si $f$ est impaire alors $P$ est impaire.
\end{enumerate}
}{}
\demonstration{ 
\begin{align*}
f(x) &= P(x) + x^{n}\eps(x), \\
f(-x) &= P(-x) + x^{n}(-1)^{n}\eps(-x) = P(-x) + x^{n}\eps_1(x),
\end{align*}
Or comme $\eps(x) \to 0$ quand $x\to 0$ alors $\eps_1\to 0$ aussi.
\begin{enumerate}
\item si $f$ est impaire alors on a : \[f(x) = -P(-x) - x^{n}\eps_1(x) \] et comme la première et cette expression sont des développements limits de $f$ à l'ordre $n$ en $0$, par unicité on a $-P(-x) = P(x)$, c'est-à-dire $P$ impaire ;
\item si $f$ est paire, on a : \[ f(x) = P(-x) + x^{n}\eps_1(x)\] alors de même, l'unicité nous dit que $P$ est alors paire.
\end{enumerate}
}{}

\proposition{ 
Soit $f:I\vers \R$ une fonction continue en $a\in I$.
\begin{enumerate}
\item le développement limité de $f$ en $a$ à l'ordre $0$ est \[ f(x) = f(a) + \eps(x), \; \lim_{x\to a}\eps(x) = 0 \ ; \]
\item la fonction $f$ est dérivable en $a$ si, et seulement si, elle possède un développement limité à l'ordre $1$ en $a$, alors dans ce cas le développement limité est donné par : \[ f(x) = f(a) + f'(a)(x-a) + \eps(x)(x-a), \; \lim_{x\to a}\eps(x) = 0.\]
\end{enumerate}
}{}
\demonstration{ 
Dans l'ordre :
\begin{enumerate}
\item On pose $\eps(x) = f(x) - f(a)$. Comme $f$ est continue en $0$, $\eps(x)$ aussi et $\lim_{x\to a}\eps(x) = 0$.
\item Supposons que $f$ soit dérivable en $a$, c'est-à-dire : \[ \lim_{x\to a}\frac{f(x) - f(a)}{x-a} = f'(a).\]On pose \[\eps(x) = \frac{f(x)-f(a)}{x-a}-f'(a).\]On a bien $\lim_{x\to a}\eps(x) = 0$ et \[ f(x)  = f(a) + (x-a)f'(a) + (x-a)\eps(x).\]

Réciproquement, supposons que $f$ admette un développement limité : \[ f(x) = a_0 + (x-a)a_1 + (x-a)\eps(x),\]avec $\lim_{x\to a}\eps(x) = 0$. Alors, par continuité $a_0 = f(a)$ et \[\lim_{x\to a}\frac{f(x) - f(a)}{x-a} = \lim_{x\to a}a_1 + \eps(x) = a_1 =  f'(a). \]
\end{enumerate}
}{}


\subsection{Développements limités et primitives}
\theoreme{ 
Soit $f: I \vers \R$ une application continue. Soit $F$ une primitive de $f$. Soit $a\in I$ et supposons que $f$ admette un développement limité en $a$ à l'ordre $n$ : \[ f(x) = a_0 + a_1(x-a) + \frac{a_2}{2}(x-a)^{2} + \ldots + \frac{a_n}{n!}(x-a)^{n} + (x-a)^{n}\eps(x), \; \lim_{x\to a}\eps(x) = 0. \]
Alors $F$ admet le développement limité suivant à l'ordre $n+1$ en $a$ : \[ F(x) = F(a) + a_0(x-a) + \frac{a_1}{2}(x-a)^{2} + \ldots + \frac{a_n}{(n+1)!}x^{n+1} + (x-a)^{n+1}\eps_1(x), \; \lim_{x\to a}\eps_1(x) = 0.\]
}{}
\demonstration{ 
Soit \[ P(t) = \sum_{k=0}^{n}\frac{a_k}{k!}(t-a)^{k}.\]
Pour tout $x\neq a$ : \[ \eps(x) = \frac{f(x) -P(x)}{(x-a)^{n}}.\]Par hypothèse, $\lim_{x\to a}\eps(x)= 0$. En posant $\eps(a) = 0$, on obtient que $\eps$ est continue sur $I$. Donc $\eps$ admet une primitive et dans l'identité \[ f(x) = a_0 + a_1(x-a) + \frac{a_2}{2}(x-a)^{2} + \ldots + \frac{a_n}{n!}(x-a)^{n} + (x-a)^{n}\eps(x), \; \lim_{x\to a}\eps(x) = 0 \] tous les termes admettent des primitives. Donc
\begin{align*}
F(x) - F(a) &= \int_{a}^{x}f(t)\dt \\
 F(x) - F(a) &= \int_{a}^{x}\left( \sum_{k=0}^{n}\frac{a_k}{k!}(t-a)^{k} + (t-a)^{n}\eps(t)\right)  \dt \\
 F(x) - F(a)  &= \sum_{k=0}^{n}\frac{a_k}{(k+1)!}(x-a)^{k+1} + u(x),\\
 u(x) &= \int_{a}^{x} (t-a)^{n}\eps(t)\dt.
\end{align*}
Par le théorème de \textsc{Rolle} : \[ u(x) = (x-a)(\theta - a)^{n}\eps(\theta)\]pour un $\theta$ compris entre $a$ et $x$. Donc \[ \abs{u(x)} = \abs{x-a}\abs{\theta-a}^{n}\abs{\eps(\theta)} \leq \abs{x-a}^{n+1}\abs{\eps(\theta)}\]
et $\eps(\theta)$ tend vers $0$ quand $x$ tend vers $a$ puisque $\theta$ est compris entre $a$ et $x$.
Donc : \[F(x) = \sum_{k=0}^{n}\frac{a_k}{(k+1)!}(x-a)^{k+1}+(x-a)^{n+1}\eps_1(x) \]où \[\eps_1(x) = \frac{u(x)}{(x-a)^{n+1}} \to 0. \]
}{}

\theoreme{ 
Soit $f:I\vers \R$ de classe $C^n$, $a\in I$. Alors $f$ admet pour développement limité à  l'ordre $n$ en $a$ : \[ f(x) + \sum_{k=0}^{n}\frac{f^{(k)}(a)}{k!}(x-a)^{k} + (x-a)^{n}\eps(x), \; \lim_{x\to a}\eps(x) = 0.\]
}{}
\demonstration{ 
Pour $n=0,1$ ça a été déjà vu. Supposons alors $n\geq 2$. Soit $f\in C^n$, posons $g=f'$ avec $g\in C^{n-1}(I)$.

Par récurrence : \[ g(x) = \sum_{k=0}^{n-1}\frac{g^{(k)}(a)}{k!} (x-a)^{k} + (x-a)^{n-1}\eps(x), \; \lim_{x\to a}\eps(x) = 0.\]
$f$ est une primitive de $g$ : 
\begin{align*}
f(x) &= f(a) + \sum_{k=0}^{n-1}\frac{g^{(k)}(a)}{(k+1)!}(x-a)^{k+1} + (x-a)^{n}\eps_1(x), \; \lim_{x\to a}\eps_1(x) = 0 \\
f(x) &= f(a) + \sum_{k=0}^{n-1}\frac{f^{(k+1)}(a)}{(k+1)!}(x-a)^{k+1} + (x-a)^{n}\eps_1(x) \\
f(x) &= f(a) + \sum_{k=1}^{n}\frac{f^{(k)}(a)}{k!}(x-a)^{k} + (x-a)^{n}\eps_1(x).
\end{align*}
}{}
\paragraph{Exemple}Soit : \[ f(x) = \systeme{\exp(-1/x^2), &\; \text{si} \ x>0 \\ 0,, &\; \text{si}\ x\leq 0 }\]son développement limité en $0$ d'ordre $n$ est : \[ f(x) =x^{n}\eps(x),\; \lim_{x\to 0}\eps(x) = 0.\]

\subsection{Développement limités usuels}
Développements limités en $0$ :
\begin{align*}
\exp(x) = &\sum_{i=0}^{n}\frac{x^{i}}{i!} + x^{n}\eps(x) \\
\ch(x) = &\sum_{i=0}^{n}\frac{x^{2i}}{(2i)!}+ x^{2n+1}\eps(x) \\
\sh(x) = &\sum_{i=0}^{n}\frac{x^{2i+1}}{(2i+1)!} + x^{2n+2}\eps(x) \\
\cos(x) =& \sum_{i=0}^{n}(-1)^{i}\frac{x^{2i}}{(2i)!} + x^{2n+1}\eps(x) \\
\sin(x) =& \sum_{i=0}^{n}(-1)^{i} \frac{x^{2i+1}}{(2i+1)!} + x^{2n+2}\eps(x) \\
\alpha\in \R : \;(1+x)^{\alpha} =& 1 +\sum_{i=0}^{n}\frac{\alpha(\alpha-1)\ldots(\alpha-i)}{(i+1)!}x^{i+1} + x^{n+1}\eps(x) \\
\frac{1}{1-x} =& \sum_{i=0}^{n}x^{i} + x^{n+1}\eps(x) \\
\frac{1}{1+x} =& \sum_{i=0}^{n}(-1)^{i}x^{i} + x^{n+1}\eps(x) \\
\log(1-x) =& -\sum_{i=1}^{n}\frac{x^{i}}{i!}+x^{n}\eps(x) \\
\log(1+x) =& \sum_{i=1}^{n}(-1)^{i+1}\frac{x^{i}}{i}x^{n}\eps(x)\\
\Arctan(x) =& \sum_{i=1}^{n}(-1)^{i+1}\frac{x^{2i-1}}{2i-1} +x^{2n}\eps(x)
\end{align*}

\demonstration{ 
\begin{align*}
\ch(x) &= \frac{e^{x} + e^{-x}}{2} \\
\ch'(x) &= \frac{e^{x} - e^{-x}}{2} (=\sh(x))\\
\ch''(x) &= \ch(x)\\
\ch^{(2i)}(0) &= 1 \\
\sh^{(2i)}(0) &= 0
\end{align*}
}{$\ch$}

\demonstration{ 
\begin{align*}
\cos^{(k)}(x) &= \cos(x+k\pi/2) \\
\cos^{(k)}(0) &= \cos(k\pi/2) \\
\cos^{(2k)}(0) &= (-1)^{k} \\
\cos^{(2k+1)}(0) &= 0
\end{align*}
}{$\cos$}

\demonstration{ 
\begin{align*}
\sin^{(k)}(x) &= \sin(x+k\pi/2) \\
\sin^{(2k)}(0) &= 0 \\
\sin^{(2k+1)}(0) &= (-1)^{k}
\end{align*}
}{$\sin$}

\demonstration{ Par récurrence :
\begin{align*}
f^{(k)}(x) &= \alpha(\alpha-1)\ldots(\alpha-k+1)(1+x)^{\alpha-k} \\
f^{(k)}(0) &= \alpha(\alpha-1)\ldots(\alpha-k+1)
\end{align*}
}{$(1+x)^{\alpha}= f(x)$}

\demonstration{ 
\begin{align*}
\frac{1-x^{n}}{1-x} &= 1+x+x^{2}+\ldots +x^n \\
\frac{1}{1-x} &= 1+x+\ldots+x^{n} + x^{n}\cdot\frac{x}{1-x}
\end{align*}
}{$1/1-x$}

\demonstration{ 
Utiliser le théorème sur le développement limité d'une primitive avec le développement limité de $1/1-x$.
}{$\log(1-x)$}

\demonstration{ 
\begin{align*}
\Arctan'(x) &= \frac{1}{1+x^{2}} \\
\frac{1}{1+x^{2}} &= \sum_{i=1}^{n}(-1)^{i}x^{2i}+x^{2n}\eps(x)
\end{align*}
et on conclut avec le théorème du développement limité d'une primitive.
}{$\Arctan(x)$}

\paragraph{Remarque}On a vu que si \[ f(x)  = \systeme{\exp(-1/x^{2}),\; &\text{si}\ x >0 \\ 0, \; &\text{si}\ x\leq 0}\] alors le développement limité de $f(x)$ en $0$ à l'ordre $n$ est \[f(x) = x^{n}\eps(x). \]Or le développement limité de $0$ en $0$ à l'ordre $n$ est identique.
\paragraph{Exemple}Soit : \[\fonc{f}{\R}{\R}{x}{\systeme{0 &\ \text{si} \ x = 0 \\ x^{3}\sin(1/x) & \ \text{si} \ x\neq 0}}. \]
La fonction $f$ est continue en $0$.

 On regarde le développement limité à l'ordre $2$ en $0$ : \[ f(x) = x^{2}\eps(x), \; \eps(x) = \systeme{0 &\ \text{si} \ x = 0 \\ x\sin(1/x) &\ \text{sinon}}, \lim_{x\to 0}\eps(x)  0.\]Donc le développement limité de $f(x)$ en $0$ à l'ordre $2$ est : \[ f(x) = x^{2}\eps(x).\]Dérivabilité de $f$ en $0$ (puisqu'elle est lisse sur $\R^*$) : \[  \frac{f(x) - f(0)}{x-0} = x^{2}\sin(1/x) \underset{x\to 0}{\longrightarrow} 0\]donc $f$ est dérivable et $f'(0) = 0$. \[ \frac{f'(x) - f'(0)}{x-0} = \frac{3x^{2}\sin(1/x) -x\cos(1/x)}{x} = 3x\sin(1/x) - \cos(1/x) \]donc $f$ n'est pas dérivable à l'ordre $2$ en $0$ (même si elle a un développement limité à l'ordre $2$).
 \section{Calculs avec les développements limités}
\subsection{Règles de calcul des développements limités}
 
\proposition{ 
Soit $f,g$ ayant des développements limités à l'ordre $n$ en $0$ : 
\[f(x) = P(x) + x^{n}\eps(x), \; g(x) = Q(x) + x^{n}\eps(x)\]avec $P,Q$ des polynômes de degré au plus $n$ et $\lim_{x\to 0}\eps(x) = 0$ (non forcément identiques). Alors
\begin{enumerate}
\item le développement limité à l'ordre $n$ en $0$ de $f+g$ est \[ (f+g)(x) = (P+Q)(x) + x^{n}\eps(x) ;\]
\item pour tout $\lambda\in \R$, le développement $\lambda f$ à l'ordre $n$ en $0$ est : \[ (\lambda f)(x) = \lambda P(x) + x^{n}\eps(x). \]
\end{enumerate}
}{} 
\demonstration{ 
\'Ecrivons $f(x) =P(x) + x^{n}\eps_f(x)$ et $g(x) = Q(x) + x^{n}\eps_g(x)$.
\begin{enumerate}
\item $(f+g)(x) = P(x)+Q(x) + x^{n}(\eps_f + \eps_g)(x)$ et on note $\eps = \eps_f + \eps_g$ qui tend bien en $0$.
\item De même.
\end{enumerate}
}{}
 
\proposition{ 
Soit $f$ qui admet le développement limité en $0$ à l'ordre $n$ : \[ f(x) = P(x) + x^{n}\eps(x), \; \lim_{x\to 0}\eps(x) = 0.\]Alors pour tout $p\in \ens{0,\ldots,n}$, $f$ admet le développement limité en $0$ à l'ordre $p$ : \[ f(x) = T_p(P)(x) + x^{p}\eps(x)\]avec $T_p(P)$ le polynôme tronqué de $P$ : \[ T_p(P) = \sum_{k=0}^{p}a_kx^{k}, \; P = \sum_{k=0}^{n}a_kx^{k}.\]
}{} 
\demonstration{ 
On a \[ f(x) = T_p(P)(x) + x^{p}\left(\sum_{k=p+1}^{n}a_kx^{k-p} + x^{n-p}\eps(x)\right).\]Et on pose \[ \eps_1(x) = \sum_{k=p+1}^{n}a_kx^{k-p}+x^{n-p}\eps(x).\]On a bien$ \eps_1(x)\to 0$ quand $x\to 0$.
}{}

\proposition{ 
Soient $f,g$ admettant les développements limités : \[ f(x) = P(x) + x^{n}\eps_1(x), \; g(x) = Q(x) + x^{n}\eps_2(x).\]Alors $fg$ admet le développement limité à l'ordre $n$ en $0$ suivant : \[ (fg)(x) = T_n(PQ)(x) + x^{n}\eps(x).\]
}{}
\paragraph{Remarque}Si $f,g$ admettent les développements limités à l'ordre $n$ en $a$ : \[ f(x) = P(x-a) + (x-a)^{n}\eps_1(x), \; g(x) = Q(x-a) + (x-a)^{n} \eps_2(x)\]alors le développement limité : \[ (fg)(x)  = T_n(PQ)(x-a)\note{On tronque avant d'évaluer en $x-a$.} + (x-a)^{n}\eps(x). \]
\demonstration{ 
\begin{align*}
(fg)(x) &= (PQ)(x) + x^{n}(Q\eps_1(x) + P\eps_2(x))\\
PQ(x) &= T_n(PQ)(x) + x^{n+1}R(x), \; R\in \R[x]\\
(fg)(x) &= T_n(PQ)(x) + x^{n}(xR(x) + Q\eps_1(x) + P\eps_2(x))
\end{align*}
On pose : 
\begin{align*}
\eps(x) &= xR(x) + Q\eps_1(x) + P\eps_2(x) \\
\lim_{x\to 0}&\ xR(x) = 0 \\
\lim_{x\to 0}&\ Q\eps_1(x) = 0\\
\lim_{x\to 0}&\ P\eps_2(x) =0 \\
\lim_{x\to 0}&\ \eps(x) = 0
\end{align*}
}{}

\paragraph{Exemple}On veut le développement limité de : \[ \Arctan(x-1)\exp(x)\]en $1$ d'ordre $3$.
\begin{align*}
\Arctan(y) &= y - \frac{y^{3}}{3} +y^{3}\eps(y) \\ 
\Arctan(x-1) &= (x-1) - \frac{(x-1)^{3}}{3}+(x-1)^{3}\eps(x) \\
\exp(x) &= \exp(x-1+1) = e\exp(x-1) \\
\exp(x) &= e\left(  1 + (x-1) + \frac{(x-1)^{2}}{2} + \frac{(x-1)^{3}}{6} + (x-1)^{3}\eps(x)\right)
\end{align*}
Et donc 
\begin{align*}
f(x) &= e\left( (x-1) - \frac{(x-1)^{3}}{3} + (x-1)^{3}\eps(x) \right)\fois \left( 1 + (x-1) + \frac{(x-1)^{2}}{2} + \frac{(x-1)^{3}}{6} + (x-1)^{3}\eps(x) \right)\\
f(x) &= e\left( (x-1) + (x-1)^{2} + \frac{(x-1)^{3}}{2} - \frac{(x-1)^{3}}{3} \right) + (x-1)^{3}\eps(x)
\end{align*}

\subsection{Développement limité d'une fonction composée}
Puisque la composition de deux fonctions polynômiales est encore un polynôme :

\proposition{ 
Soient $f,g$ admettant un développement limité en $0$ à l'ordre $n$ : \[f(x) = P(x) + x^{n}\eps(x), \; g(x) = Q(x) + x^{n}\eps(x) \]avec $P,Q$ deux polynômes de degré inférieur à $n$.

Supposons que $g(0) = 0$ alors $f\rond g$ admet le développement limité suivant à l'ordre $n$ en $0$ : \[ (f\rond g)(x) = T_n(P\rond Q)(x) + x^{n}\eps(x).\]
}{}
\demonstration{ 
Supposons $n=0$, alors $P$ et $Q$ sont deux polynômes constants donc $f(x) = P(0) + \eps(x)$ et $g(x) = Q(0) + \eps(x)$. Comme $Q(0) = 0$ on a bien $f(g(x)) = (P\rond Q)(x) + \eps(x)$ par continuité.

Supposons que $n\geq 1$. On note $f(x) = P(x) + x^{n}\eps_1(x)$ et $g(x) = Q(x) + x^{n}\eps_2(x)$. Posons $P(x) = a_0 + a_1 x + \ldots + a_nx^{n}$.
\begin{align*}
(f\rond g)(x) &= P(g(x)) + g(x)^{n}\eps_1(g(x)) \\
P(g(x)) &= \sum_{i=0}^{n}a_ig(x)^{i} \\
P(g(x)) &=\footnotemark T_n\left(\sum_{i=0}^{n}a_iQ(x)^{i}\right) + x^{n}\eps_3(x)
\end{align*}
Puisque $Q(0) = 0$, on a $Q(x) = b_1x+\ldots + b_nx^{n}$ et donc : 
\begin{align*}
g(x) &= b_1x + \ldots+ b_nx^{n} + x^{n}\eps_2(x) \\
g(x) &= x(b_1+\ldots+ b_nx^{n-1} + x^{n-1} \eps_2(x))\\
g(x) &= xh(x)\\
(f\rond g)(x) &= P(xh(x)) + x^{n}h(x)^{n}\eps_1(xh(x))\\
(f\rond g)(x) &= T_n(P\rond Q)(x)  + x^{n}(h(x)^{n}\eps_1(xh(x)) + \eps_3(x))
\end{align*}
On pose $\eps_4(x) = h(x)^{n}\eps_1(xh(x)) + \eps_3(x)$ et : 
\begin{align*}
\lim_{x\to 0} xh(x) &= 0\\
\lim_{x\to 0}\eps_3(x) &= 0 \\
\lim_{x\to 0} h(x)^{n} &= b_1^{n} \\
\lim_{x\to 0} \eps_4(x) &= 0.
\end{align*}
}{}
\footnotetext{D'après les formules de développements limités d'une somme et d'un produit.}

\paragraph{Exemple}Développement limité de $\cos(\sin(x))$ à  l'ordre $5$ en $0$ :
\begin{align*}
\sin(x) &= x - \frac{x^{3}}{3!} + \frac{x^{5}}{5!}+x^{6}\eps(x) \\
\cos(x) &= 1-\frac{x^{2}}{2!} + \frac{x^{4}}{4!} + x^{5}\eps(x)\\
\cos(\sin(x)) &= T_5\left(1 - \frac{\left(x - \frac{x^{3}}{3!} + \frac{x^{5}}{5!} \right)^{2}}{2!} + \frac{\left(x - \frac{x^{3}}{3!} + \frac{x^{5}}{5!}\right)^{4}}{4!} \right) + x^{5}\eps(x)\\
\cos(\sin(x)) &= 1 - \frac{x^{2}}{2} + \frac{x^{4}}{3!}+ \frac{x^{4}}{4!}  + x^{5}\eps(x) \\
\cos(\sin(x)) &= 1 - \frac{x^{2}}{2} + \frac{5x^{4}}{24} + x^{5}\eps(x)
\end{align*}

\proposition{ 
Soient $f,g$ admettant des développements limités à l'ordre $n$ en $0$. Alors si $g(0) \neq 0$ alors la fonction $f/g$ admet un développement limité à l'ordre $n$ en $0$.
}{}
\demonstration{ 
Puisque $g(0)\neq 0$, $f/g$ est définie et continue en $0$.
Comme $f/g = f \times 1/g$, il suffit de vérifier que $1/g$ admet un développement limité en $0$ (puis on applique la règle de produit).

Posons $a = g(0) \neq 0$. On a : \[\frac{1}{g(x)} = \frac{1}{a +(g(x) - a)} = \frac{1}{a} \cdot\frac{1}{1 + \left(\frac{g(x)}{a} - 1 \right)} \]
Il suffit de vérifier que : \[ \frac{1}{1 + \left(\frac{g(x)}{a} - 1 \right)} \]admet un développement limité à l'ordre $n$ en $0$.
Posons \[h(x) = \frac{1}{1+x} \]on a alors \[ \frac{1}{1 + \left(\frac{g(x)}{a} - 1 \right)}  = h\left( \frac{g(x)}{a}-1\right) = (h\rond k)(x)\]où $k(x) = g(x)/a - 1$. Or $k(x)$ admet un développement limité à l'ordre $n$ en $0$ et $h(x)$ admet également un développement limité à l'ordre $\infty$ en $0$. Enfin, $k(0) = 0$ et donc on conclut avec le résultat précédent.
}{}
\paragraph{Exemple}Développement limité de $f : f(x) = 1/(a-x)$ en $0$ à l'ordre $n$.
\begin{align*}
f(x) &= \frac{1}{a}\frac{1}{1-x/a}\\
\frac{1}{1-t} &= 1 + t + t^{2} + \ldots + t^{n} + t^{n}\eps(t) \\
f(x) &= \frac{1}{a}\left( 1 + \frac{x}{a} + \frac{x^{2}}{a^{2}} + \ldots + \frac{x^{n}}{a^{n}} \right) + x^{n}\eps(x) \\
\frac{1}{a-x} &= \frac{1}{a}+\frac{x}{a^{2}} + \ldots + \frac{x^{n}}{a^{n+1}}+ x^{n}\eps(x).
\end{align*}

La méthode précédente ne donne pas de formule générale pour le développement limité de $f/g$.
\paragraph{Rappel}Si $P,Q\in \R[x]$, $n\in \N$ et si $Q(0)\neq 0$. Alors la division de $P$ par $Q$ suivant les puissances croissantes à l'ordre $n$ est l'unique polynôme $A$ tel que :
\begin{itemize}
\item $P-AQ$ est divisible par $X^{n+1}$ ;
\item soit $A=0$, soit $\deg A \leq n$.
\end{itemize}
\proposition{Soient $f,g$ avec les développements limités suivants à l'ordre $n$ en $0$ :
\begin{align*}
f(x) &= A(x) + x^{n}\eps_1(x), \\
g(x) &= B(x) + x^{n}\eps_2(x).
\end{align*}
Supposons que $g(0)=B(0)\neq 0$. Le développement limité à l'ordre $n$ de $f/g$ en $0$ est : 
\[ \frac{f}{g}(x) = Q(x) + x^{n}\eps(x)\]où $Q$ est la division de $A$ par $B$ à l'ordre $n$ suivant les puissances croissantes.
}{}
\demonstration{ 
On a $A(x) = Q(x)B(x) + x^{n+1}R(x)$ où $R$ est un polynôme et $Q=0$ ou $\deg Q \leq n$. Ainsi 
\begin{align*}
 f(x) &= Q(x)B(x) + x^{n+1}R(x) + x^{n}\eps_1(x)\\
f(x) - Q(x)g(x) &= x^{n+1}R(x) + x^{n}\eps_1(x) - Q(x)x^{n}\eps_2(x) \\
f(x) - Q(x)g(x) &= x^{n}(\eps_1(x) - Q(x) \eps_2(x) + xR(x))\\
\frac{f}{g}(x) &= Q(x) + x^{n}\eps_3(x)\\
\eps_3(x) &= \frac{1}{g(x)}(\eps_1(x) - Q(x)\cdot \eps_2(x) + xR(x)) \underset{x\to 0}{\longrightarrow}0
\end{align*}
}{}
\paragraph{Exemple}Développement limité de $\tan(x)$ à l'ordre $5$ en $0$.
\begin{align*}
\tan(x) &= \frac{\sin(x)}{\cos(x)}\\
\sin(x) &= x - \frac{x^{3}}{3!} + \frac{x^{5}}{5!} + x^{5}\eps(x) \\
\cos(x) &= 1 - \frac{x^{2}}{2!} + \frac{x^{4}}{4!} + x^{5}\eps(x) \\
x - \frac{x^{3}}{3!} + \frac{x^{5}}{5!}  &= \left(1 - \frac{x^{2}}{2!} + \frac{x^{4}}{4!} \right)\left( x + \frac{x^{3}}{3} + \frac{2x^{5}}{15}\right) + x^{6}R(x) \\
\frac{f(x)}{g(x)} &= x + \frac{x^{3}}{3} + \frac{2}{15}x^{5} + x^{5}\eps(x)
\end{align*}

\section{Applications}

\paragraph{Applications}Les développements limités peuvent être utiles pour : 
\begin{enumerate}
\item les calculs de limites (pour des \og formes indéterminées \fg{}) ;
\item études de fonctions ou courbes paramétrées.
\end{enumerate}

\subsection{Calculs de limites}
\paragraph{Exemple}On veut calculer : \[ \lim_{x\to 0} \frac{x\log\ch x}{1+x\sqrt{1+x} - \exp(\sin x)}.\]
\begin{align*}
\ch(x) &= 1 +\frac{x^{2}}{2!} + x^{2}\eps(x) \\
\log(1+x) &= x - \frac{x^{2}}{2} +x^{2}\eps(x), \\
\log \ch x &= \log(1+(\ch x - 1)) \\
\log \ch x &= T_2\left( \frac{x^{2}}{2}  - \frac{\left(\frac{x^{2}}{2} \right)^{2}}{2}\right) + x^{2}\eps(x) \\
\log \ch x &= \frac{x^{2}}{2} + x^{2}\eps(x) \\
x\log\ch x &= \frac{x^{3}}{2} + x^{3}\eps(x) \ ;
\end{align*}
\begin{align*}
\sqrt{1+x} &= 1 + \frac{x}{2} + \frac{\frac{1}{2}\left(\frac{1}{2}-1\right)x^{2}}{2!} + x^{2}\eps(x) \\
\sqrt{1+x} &= 1 + \frac{x}{2} - \frac{x^{2}}{8} + x^{2}\eps(x)\\
x\sqrt{1+x} &= x + \frac{x^{2}}{2} - \frac{x^{3}}{8} + x^{3}\eps(x) \\
\sin(x) &= x - \frac{x^{3}}{6} + x^{3}\eps(x) \\
\exp(x) &= 1 + x + \frac{x^{2}}{2} + \frac{x^{3}}{6} +x^{3}\eps(x), \\
\exp(\sin x)) &= T_3\left( \left(x\donne 1 + x + \frac{x^{2}}{2} + \frac{x^{3}}{6} \right) \left( x - \frac{x^{3}}{6}\right)\right) + x^{3}\eps(x) \\
\exp(\sin x)) &= 1 + x - \frac{x^{3}}{6} + \frac{x^{2}}{2} + \frac{x^{3}}{6} + x^{3}\eps(x) \\
\exp(\sin x)) &= 1 + x + \frac{x^{2}}{2} + x^{3}\eps(x) \ ;
\end{align*}
Ainsi 
\begin{align*}
\frac{x\log\ch x}{1+x\sqrt{1+x} - \exp(\sin x)} &= \frac{ \frac{x^{3}}{2} + x^{3}\eps(x)}{1 +x + \frac{x^{2}}{2} - \frac{x^{3}}{8}  -1 - x - \frac{x^{2}}{2} + x^{3}\eps(x)}\\
\frac{x\log\ch x}{1+x\sqrt{1+x} - \exp(\sin x)} &= \frac{\frac{x^{3}}{2} + x^{3}\eps(x)}{-\frac{x^{3}}{8} + x^{3}\eps(x)} \\
\lim_{x\to 0}\frac{x\log\ch x}{1+x\sqrt{1+x} - \exp(\sin x)} &=\lim_{x\to 0} \frac{1/2 + \eps(x)}{-1/8 + \eps(x)} = -4.
\end{align*}

\paragraph{Remarque}Un calcul de dérivée s'obtient par un calcul de limite et donc parfois par développements limités.
\paragraph{Exemple}On prend \[ f(x) = \frac{\cos x}{1+x+x^{2}}\]et on cherche $f^{(i)}(0)$ pour $i \in \ens{0,\ldots,4}$, c'est-à-dire que l'on cherche le développement limité de $f$ en $0$ à l'ordre $4$.
\begin{align*}
\cos x &= 1 - \frac{x^{2}}{2} + \frac{x^{4}}{4!} + x^{4}\eps(x).
\end{align*}
On cherche le développement limité de \[ g(x) = \frac{1}{1+x+x^{2}}\]que l'on peut voir comme \[ g(x) = (a\rond b)(x)\; ; \; a(x) = \frac{1}{1+x} \; ; \; b(x) = x+x^{2}.\]

\begin{align*}
a(x) &= 1 - x + x^{2} - x^{3} + x^{4} + x^{4}\eps(x) \\
g(x) &= T_4((x\donne  1 - x + x^{2} - x^{3} + x^{4} )(x+x^{2})) + x^{4}\eps(x) \\
g(x) &= 1 - x - x^{2} + x^{2} + x^{4} + 2x^{3} +x^{4} - x^{3} - 3x^{4} + x^{4 }+ x^{4}\eps(x)\\
g(x) &= 1 - x + x^{3} - x^{4} + x^{4}\eps(x) \\
f(x) &= T_4\left( (1-x+x^{3}-x^{4})\left(1 - \frac{x^{2}}{2} + \frac{x^{4}}{24}\right)\right) + x^{4}\eps(x) \\
f(x) &= 1 - x -\frac{x^{2}}{2} + \frac{3x^{3}}{2} - \frac{23x^{4}}{24} + x^{4}\eps(x)
\end{align*}
Comme $f$ admet un développement limité à l'ordre $4$ en $0$, elle est dérivable quatre fois. De plus 
\begin{align*}
f(0) &= 1 \\
f'(0) &= -1 \\
f^{(2)}(0) &= -1 \\
f^{(3)}(0) &= 9 \\
f^{(4)}(0) &= -23.
\end{align*}

\subsection{Courbes paramétrées}

\paragraph{Rappels sur les fonctions classiques}Quelques rappels :
\begin{itemize}
\item on définit le logarithme népérien par : \[ \log(x) = \int_{1}^{x}\frac{\dt}{t}.\]Ainsi $\log : \R^{*}_{+} \to \R$ est croissante, $C^{\infty}$, 
\begin{align*}
\frac{\dd}{\dx}\log x &= \frac{1}{x}\\
\lim_{x\to 0, x>0} \log x &= -\infty \\
\lim_{x\to \infty} \log x &= +\infty \\
\log(ab) = \log a + \log b.
\end{align*}
\item on définit l'exponentielle, $\exp : \R \vers \R$, qui est croissante, lisse et stable par dérivation.
\begin{align*}
\lim_{x\to -\infty}\exp(x) &= 0 \\
\lim_{x\to + \infty} \exp(x) &= \infty\\
\exp(a+b) &= \exp(a)\exp(b).
\end{align*}
\item soient $a\in \R_+^{*}, b\in \R$ alors on définit : \[ a^{b} = \exp(b\log a).\]
\begin{align*}
a^{b+b'} &= a^{b}a^{b'} \\
(aa')^{b} &= a^{b} (a')^{b} \\
\left(a^{b}\right)^{c} &= a^{bc}\\
a^{0} &= 1 = 1^{b}\\
\frac{\dd }{\dx}x^{b} &= bx^{b-1} \\
\frac{\dd }{\dx}a^{x} &= \log (a)a^{x} \\
\lim_{x\to 0, x >0} x^{a}(\log x)^{n} &= 0\; \; ,\; a >0 \et n\in \Z \\
\lim_{x\to+\infty} x^{a} e^{x} &= +\infty \\
\lim_{x\to -\infty} x^{a}e^{x} &= 0.
\end{align*}
\item trigonométrie :
\begin{align*}
\lim_{x\to 0}\frac{\sin x}{x} &= 1 \\
\sin(x+t) &= \cos(t)\sin(x) + \cos(x)\sin(t) \\
\cos(x+t) &= \cos(x)\cos(t) - \sin(x)\sin(t)\\
\tan(x+t) &= \frac{\tan(t) + \tan(x)}{1 - \tan(x)\tan(t)}.
\end{align*}
\item $\Arcsin : [-1,1] \vers [-\pi/2,\pi/2]$ est lisse sur $]-1,1[$ et : \[ \Arcsin' (x) = \frac{1}{\sqrt{1-x^{2}}}.\]$\Arccos : [-1,1] \vers [0,\pi]$ est la réciproque de $\cos$ et on a la relation : \[ \Arccos(x) + \Arcsin(x) = \frac{\pi}{2}.\]$\Arctan : \R\vers ]-\pi/2,\pi/2[$ est lisse et : \[ \Arctan'(x) = \frac{1}{x^{2}+1}.\]
\item trigonométrie hyperbolique : 
\begin{align*}
\sh(x) &= \frac{e^{x} - e^{-x}}{2} \\
\ch(x) &= \frac{e^{x} + e^{-x}}{2} \\
\th(x) &= \frac{\sh(x)}{\ch(x)}
\end{align*}leurs réciproques $\Argsh : \R \vers \R$, $\Argch : [-1,\infty]\vers \R_+$ et $\Argth:]-1,+1[ \vers \R$ sont lisses sur l'intérieur de leur domaine de définition.
\begin{align*}
\Argsh'(t) &= \frac{1}{\sqrt{1+t^{2}}} \\
\Argch'(t) &= \frac{1}{\sqrt{t^{2}-1}} \\
\Argsh(t) &= \log(t + \sqrt{t^{2}+1}) \\
\Argch(t) &= \log(t + \sqrt{t^{2}-1}).
\end{align*}
\end{itemize}

\definition{ 
Soit $f : I \vers \R^{2}$ avec $I$ un intervalle ou une union finie d'intervalles dans $\R$. Soient $u,v$ telles que \[ \forall t, \; f(t) = (u(t),v(t)).\]
\begin{enumerate}
\item On dit que $\lim_{t\to t_0}f(t) = l$ où $l = (l_1,l_2)$ si $\lim_{t\to t_0}u(t) = l_1$ et $\lim_{t\to t_0}v(t) = l_2$.
\item On dit que $f$ est continue en $t_0$ si les fonctions $u$ et $v$ sont continues en $0$. $f$ est continue sur $I$ si elle est continue en tout point de $I$.
\item On dit que $f$ est dérivable en $t_0$ si $u$ et $v$ le sont et on note $f'(t_0) = (u'(t_0),v'(t_0))$.
\end{enumerate}
}{}

\proposition{ 
Si $f,g :  I \vers \R^{2}$ et si $t_0 \in I$ alors :
\begin{enumerate}
\item si $\lim_{t\to t_0} f(t) = l$ et $\lim_{t\to t_0}g(t) = m$ alors $\lim_{t\to t_0}(f+g)(t) = l+m$ ;
\item si $f,g$ sont dérivables en $t_0$ alors $f+g$ aussi et on a $(f+\lambda g)'(t_0) = f'(t_0) + \lambda g'(t_0)$.
\end{enumerate}
}{}
\proposition{ 
Soit $(r,s)$ une base de $\R^{2}$ et soit $f : I \vers \R^{2}$ telle que $f(t) = (u(t),v(t))$. Soit $(a(t),b(t))$ les coordonnées de $f(t)$ dans la base $(r,s)$.
\begin{enumerate}
\item On a : \[\lim_{t\to t_0}f(t) = l \ssi \systeme{ \lim_{t\to t_0} a(t) = \alpha \\ \lim_{t\to t_0}b(t) = \beta} \]où $(\alpha,\beta)$ sont les coordonnées de $l$ dans la base $(r,s)$.
\item Idem pour la dérivée.
\end{enumerate}
}{}

\demonstration{ 
Soient $\alpha,\beta\in \R$ et $r,s\in \R^{2}$. On a $l = \alpha \cdot r + \beta\cdot s$, \[f(t) = (u(t),v(t)) = a(t)\cdot r + b(t) \cdot s \]avec $a(t),b(t) \in \R$.
\begin{enumerate}
\item On a que $\lim_{t\to t_0}f(t) = l$ c'est par définition : \[ \systeme{\lim_{t \to t_0}u(t) =l_1 \\ \lim_{t\to t_0}v(t) = l_2}.\]
\begin{align*}
\systeme{\lim_{t\to t_0}a(t) = \alpha \\ \lim_{t\to t_0}b(t) = \beta} &\ssi \systeme{\lim_{t\to t_0}(a(t)r_1 + b(t)s_1) = \alpha r_1 + \beta s_1 \\ \lim_{t\to t_0}(a(t)r_2 + b(t)s_2) = \alpha r_2 + \beta s_2} \\
& \ssi \systeme{\lim_{t\to t_0} a(t) r_1 + b(t) s_1 = l_1 \\ \lim_{t \to t_0}a(t) r_2 + b(t) s_2 = l_2 }
\end{align*}
\item De même ... 
\end{enumerate}
}{}

\definition{ 
On dit que $f : I \vers \R^{2}, f(t) = (u(t),v(t))$ admet un développement limité à l'ordre $n$ en $t_0$ si $u(t)$ et $v(t)$ admettent un développement limité à l'ordre $n$ en $t_0$.

Si $u(t) = u_0 + u_1(t-t_0) + \ldots + u_n(t-t_0)^{n} + (t-t_0)^{n}\eps_1(t)$ et $v(t) = v_0 + v_1(t-t_0) + \ldots + v_n(t-t_0)^{n} + (t-t_0)^{n}\eps_2(t)$ alors on appelle \[ f(t) = (u_0,v_0) + (t-t_0) (u_1,v_1) + \ldots + (t-t_0)^{n}(u_n,v_n) + (t-t_0)^{n}\eps(t)\]le développement limité de $f$ à l'ordre $n$ en $t_0$ avec $\lim_{t\to t_0} \eps(t) = (0,0)$.
}{}
\paragraph{Exemple}Le développement limité de $f : t\donne (2t^{3}-t\sin t, t^{3}+\cos t)$ à l'ordre $4$ en $0$ :
\begin{align*}
2t^{3} -t\sin t &= -t^{2} + 2t^{3} +\frac{t^{4}}{6} + t^{4}\eps(t) \\
t^{3}+\cos t &= 1 - \frac{t^{2}}{2} + t^{3} + \frac{t^{4}}{24} + t^{4}\eps(t)\\
f(t) &= (0,1) - t^{2}(1,1/2) + t^{3}(2,1) + t^{4}(1/6,1/24) + t^{4}\eps(t).
\end{align*}

\definition{ 
On appelle \textit{courbe paramétrée} de $\R^{2}$ une fonction $f:I\vers \R^{2}$.
}{}
\paragraph{Exemple}$f:R\vers \R^{2}, t \donne (\cos t,\sin t)$.
\begin{center}
\sageplot[width = 8cm]{parametric_plot((cos(x),sin(x)),(x,0,2*pi))}
\end{center}

\paragraph{Remarque}
Supposons que $f$ soit dérivable en $t\in I$. Alors $u(t),v(t)$ admettent des développements limités à l'ordre $1$ en $t_0$ et donc $f$ admet aussi un développement limité à l'ordre $1$ en $t_0$. Or si \[ f(t) = f(t_0) + (t-t_0) f'(t_0) + (t-t_0)\eps(t)\] alors \[ \lim_{t\to t_0}\frac{1}{t-t_0}(f(t) - f(t_0)) = f'(t_0).\]

\definition{
On appelle $f'(t_0)$ \textit{vecteur tangent} de $f$ en $t_0$. La droite affine passant par $f(t_0)$ et de vecteur directeur $f'(t_0)$ s'appelle la \textit{tangente} à $f$ en $t_0$.
}{}

\paragraph{Remarque}Le vecteur tangent dépend du paramétrage de la courbe et non seulement de sa représentation.

\paragraph{Exemple}Soit $f: \R \vers \R^{2}, t \donne (\cos(t),\sin(t))$ et soit $g : \R \vers \R^{2},t\donne (\cos(2t),\sin(2t))$. Remarquons que $f$ et $g$ on même représentation graphique. Cependant les vecteurs tangents en $0$ à $f$ et $g$ sont :  
\begin{align*}
f'(0) &= (0,1) \\
g'(0) &= (0,2).
\end{align*}
La tangente à $f$ en $t_0$ est la droite d'équation : \[\det \matrice{y - v(t_0) & v'(t_0) \\ x - u(t_0) & u'(t_0)} = 0 \]c'est-à-dire : \[(y-v(t_0))u'(t_0) - (x-u(t_0))v'(t_0) = 0. \]


\subsection{\'Etude de fonctions}

Soit $f : I \vers \R$, où $I$ est un intervalle de $\R$. On procède à l'étude de $f$ au voisinage de $x_0\in \iR$. En particulier, on s'intéresse notamment au graphe de $f$.

\subsubsection{\'Etude locale}

\proposition{ 
Soit $x_0\in I, f : I \vers \R$. On suppose que $f$ admet un développement limité à l'ordre $n$ en $x_0$ : \[ f(x) = P(x-x_0) + (x-x_0)^{n}\eps(x), \; \lim_{x\to x_0}\eps(x) = 0\]où $P \in \R[x], P(x) = a_px^{p} + \ldots + a_nx^{n}$ avec $0\leq p \leq n$ et $a_p\neq 0$.

Alors il existe $\alpha\in \R_+^{*}$ tel que pour tout $x\in ]x_0 - \alpha, x_0 + \alpha[$ et $x\neq x_0$, $f(x)$ est non nul et a le signe de $a_p(x-x_0)^{p}$.
}{}
\demonstration{ 
Puisque $p\leq n$, le développement limité de $f$ en $x_0$ à l'ordre $p$ est : \[ f(x) = (T_p(P))(x-x_0) + (x-x_0)^{p}\eps(x).\]
C'est-à-dire : \[ f(x) = a_p(x-x_0)^{p} + (x-x_0)^{p}\eps(x).\]

Pour tout $x\neq x_0$, on a : \[ \frac{f(x)}{(x-x_0)^{p}} = a_p + \eps(x)\]et $a_p \neq 0$, $\lim_{x\to x_0}\eps(x) = 0$. Ainsi il existe $\alpha$ tel que pour tout $x\in ]x_0 - \alpha, x_0 + \alpha[$ et $x\neq x_0$, $\abs{\eps(x)} < \frac{1}{2}(a_p)$. C'est-à-dire que pour un tel $x$, $f(x)\neq 0$ et est du même signe que $a_p(x-x_0)$.
}{}

\definition{ 
Si $I\dans \R$ est un intervalle et $f : I \vers \R$ est une fonction numérique et si $x_0 \in \barre{I}$\footnotemark, on dit que $f$ est \textit{positive au voisinage de $x_0$} s'il existe un voisinage $J\dans I$ de $x_0$ tel que pour tout $x\in J$ et $x\neq x_0$, $f(x) > 0$.
}{}
\footnotetext{Dans $I$ ou l'une de ses bornes.}

\paragraph{Exemple}Prenons : \[ f(x) = e\cdot\sqrt{x}-e^{x}.\]On cherche le signe de $f$ quand $x$ tend vers $1$.
\begin{align*}
f(x) &= e\left[(1+(x-1))^{1/2}  - e^{x-1}\right] \\
&\systeme{ (1+(x-1))^{1/2} &= 1 + \frac{1}{2}(x-1) + (x-1)\eps(x)\\ e^{x-1} &= 1 + (x-1) + (x-1)\eps(x)}\\
f(x) &= e\left(-\frac{1}{2}(x-1) + (x-1)\eps(x)\right)\\
f(x) &= \frac{-e}{2}(x-1) + (x-1)\eps(x).
\end{align*}
Ainsi au voisinage de $1$, le signe de $f$ est le même que celui de $1-x$.
\begin{center}
\sageplot[width = 9cm]{plot(exp(1)*sqrt(x) - exp(x),(x,0,1.5))}
\end{center}

\definition{ 
Soit $f : I\vers \R$ une fonction dérivable en $x_0\in I$.
La \textit{tangente} en $(x_0,f(x_0))$ au graphe de $f$ est la droite affine d'équation : \[ y = f'(x_0)\cdot(x-x_0) + f(x_0). \]
}{}

\definition{ 
Soit $f : I \vers \R$ une fonction dérivable en $x_0\in I$.

On dit que $f$ admet une \textit{inflexion} au point $(x_0,f(x_0))$ si la fonction \[ x\donne f(x) - (f'(x_0)\cdot(x-x_0) + f(x_0))\]s'annule en $x_0$ en changeant de signe.
}{}

\proposition{ Soit
$f : I\vers \R$ une fonction dérivable en $x_0\in I$. On a : 
\begin{enumerate}
\item si $f(x) = a_0 + a_1(x-x_0) + (x-x_0)\eps(x)$ est le développement limité de $f$ à l'ordre $1$ en $x_0$, alors la tangente au graphe de $f$ en $(x_0,f(x_0))$ est donnée par : \[ y = a_0 + a_1(x-x_0) \ ;\]
\item si $f(x) = a_0 + a_1(x-x_0) + a_2(x-x_0)^{2} + (x-x_0)^{2}\eps(x)$ est le développement limité de $f$ à l'ordre $2$ en $x_0$, alors 
\begin{itemize}
\item si $a_2 > 0$ alors pour tout $x\neq x_0$ dans un voisinage suffisamment petit de $x_0$, le point $(x,f(x))$ est au-dessus de la tangente ;
\item si $a_2 < 0$ alors pour tout $x\neq x_0$ dans un voisinage suffisamment petit de $x_0$, le point $(x,f(x))$ est en-dessous de la tangente ;
\end{itemize}
\item si $f(x) = a_0 + a_1(x-x_0) + a_3(x-x_0)^{3} +  (x-x_0)^{3}\eps(x)$ est le développement limité de $f$ à l'ordre $3$ en $x_0$, alors si $a_3 \neq 0$, $f$ admet un point d'inflexion en $(x_0,f(x_0))$.
\end{enumerate}
}{}

\demonstration{ 
Dans l'ordre : 
\begin{enumerate}
\item Comme $f$ est dérivable en $x_0$, on a $a_1 = f'(x_0)$ et $a_0 = f(x_0)$, l'équation de la tangente est \[ y = a_1(x-x_0) + a_0.\]
\item Posons \[ u(x) = f(x) - (f'(x_0)(x-x_0) + f(x_0)).\]On a alors : \[u(x) = a_0 + a_1(x-x_0) + a_2(x-x_0)^{2} - a_1(x-x_0) - a_0 + (x-x_0)^{2}\eps(x) = a_2(x-x_0)^{2} + (x-x_0)^{2}\eps(x). \]
Comme $a_2\neq 0$ (par hypothèse) alors la proposition précédente entraine que le signe de $u(x)$ au voisinage de $0$ est celui de $a_2(x-x_0)^{2}$, c'est-à-dire le signe de $a_2$.
\item Posons de même\[u(x) = a_3(x-x_0)^{3} + (x-x_0)^{3}\eps(x). \]D'après la proposition précédente, le signe de $u(x)$ au voisinage de $x_0$ est celui de $a_3(x-x_0)$ puisque $a_3\neq 0$. Comme $(x-x_0)^{3}$ n'est pas de signe constant, c'est un point d'inflexion.
\end{enumerate}
}{}

\paragraph{Remarque}Si $a_2\neq 0$, alors :
\begin{itemize}
\item si $a_2>0$, $f(x)$ admet un minimum local en $x_0$ ;
\item sinon, $f(x)$ admet un maximum local en $x_0$.
\end{itemize}

\paragraph{Remarque, généralisation du résultat}Supposons que le développement limité de $f$ en $x_0$ est de la forme \[ f(x) = a_0 + a_1(x-x_0) + a_p(x-x_0)^{p} + (x-x_0)^{p}\eps(x),\]avec $p\geq 2$. De plus on suppose $a_p\neq 0$. Alors en posant \[u(x) = f(x) - (f'(x_0)\cdot(x-x_0) + f(x_0))\]est du signe de $a_p(x - x_0)^{p}$ au voisinage de $x_0$.
\begin{itemize}
\item Si $p$ est pair alors $a_p>0$ implique que $x_0$ est un minimum local,  $a_p <0$ implique que $x_0$ est un maximum local.
\item Si $p$ est impair alors $x_0$ est un point d'inflexion.
\end{itemize}

\paragraph{Exemple}
Prenons : \[f(x) = \sqrt[3]{x^{3} + 6x^{2} -5} \]définie sur $\R$ et étudions $f$ au voisinage de $x_0 = 2$.
On a :
\begin{align*}
f(x+2) &= \left( (x+2)^{3} + 6x^{2} - 5\right)^{1/3} \\
f(x+2) &= \left( 27 + 36x + 12x^{2} + x^{3}\right)^{1/3}\\
f(x+2) &= 3\cdot \left( 1 + \frac{4}{3}x + \frac{4}{9}x^{2} + \frac{1}{27}x^{3}\right)^{1/3} \\
f(x+2) &= 3\cdot\left( 1 + \frac{1}{3}\left( \frac{4}{3}x + \frac{4}{9}x^{2} \right) + \frac{\frac{1}{3}\left(\frac{-2}{3}\right)}{2}\left(\frac{16}{9}x^{2}\right) \right)+ x^{2}\eps(x)\\
f(x+2) &= 3 + \frac{4}{3}x - \frac{4}{27}x^{2} + x^{2}\eps(x).
\end{align*}
L'équation de la tangente est : \[ y = 3 + \frac{4}{3}(x-2).\]Comme le terme en $x^{2}$ est non nul et négatif, la courbe est en-dessous de la tangente.
\begin{sagesilent}
p = plot((x**3 + 6*x*x -5)**(1/3),(x,1,3))
p += plot(3+4/3*(x-2),(x,1,3),rgbcolor = 'red')
\end{sagesilent}
\begin{center}
\sageplot[width = 9cm]{p}
\end{center}


\subsubsection{Branches infinies}Soit $f : \R \vers \R$ une fonction numérique.

\definition{ 
Si $\lim_{x\to a}f(x) = \pm \infty$, avec $a\in \R$ alors la droite $x = a$ est une \textit{asymptote verticale} de $f$.

Si $\lim_{x\to +\infty}f(x) = +\infty$ ou si $\lim_{x\to +\infty}f(x) = -\infty$ alors $f$ admet une \textit{branche infinie} en $+\infty$.

Si $\lim_{x\to -\infty}f(x) = +\infty$ ou si $\lim_{x\to-\infty}f(x) = -\infty$ alors $f$ admet une \textit{branche infinie} en $-\infty$.

Soit $a,b\in \R$. La droite $y = ax+b$ est \textit{asymptote} à $f$ quand $x$ tend vers $\pm \infty$ si : \[ \lim_{x\to \pm \infty}f(x) -ax - b = 0.\] Si $a=0$ on dit que l'asymptote est \textit{horizontale}.

Soit $a\in \R$, si \[ \lim_{x\to \pm\infty} \frac{f(x)}{x} = a\]alors on dit que $f$ a une \textit{direction asymptotique de pente $a$ en $\pm \infty$}.

Si \[ \lim_{x\to \pm \infty}\frac{f(x)}{x} = \pm \infty\]alors on dit que $f$ a une \textit{direction asymptotique verticale} en $\pm \infty$.
}{}
\proposition{ 
Soient $a,b\in \R$. La droite $y = ax+b$ est asymptote à $f$ quand $x$ tend vers $+\infty$ (resp. en $-\infty$) si, et seulement si :
\begin{itemize}
\item on a \[ \lim_{x\to +\infty}\frac{f(x)}{x} = a \ ;\]
\item et de plus \[ \lim_{x\to \infty} f(x) -ax = b.\]
\end{itemize}
}{}

\paragraph{Exemple}Soit $f : \R \vers \R$ définie par : \[ f(x) = 1 + \frac{\sin x}{x^{2} +1}.\]On a \[ \lim_{x\to +\infty} \abs{f(x) - 1} = 0\]et donc $y=1$ est asymptote à $f$ en $+\infty$. La différence \[f(x) - 1 = \frac{\sin x}{x^{2} +1} \]est du signe de $\sin x$ qui oscille.
\begin{center}
\sageplot[width = 9 cm]{plot(1+sin(x)/(x*x + 1),(x,0,20))}
\end{center}

\paragraph{Exemple}Avec \[ f(x) = \sqrt[3]{x^{3} + 6x^{2} - 5}\]on regarde s'il y a une asymptote quand $x$ tend vers $\pm \infty$ et la position par rapport à la possible asymptote. On écrit $f$ sous la forme : \[ f(x) = x u(1/x)\]avec \[ u(x) = \sqrt[3]{1 + 6x - 5x^{3}}.\]
Le développement limité de $u$ en $0$ à l'ordre $2$ est : 
\begin{align*}
u(x) &= (1+6x - 5x^{3})^{1/3} \\
u(x) &= 1 + \frac{1}{3}(6x) + \frac{1}{3}\left(\frac{-2}{3}\right)\frac{1}{2}(36x^{2}) + x^{2}\eps(x) \\
u(x) &= 1 + 2x - 4x^{2} + x^{2}\eps(x).
\end{align*}
Ainsi pour $x$ au voisinage de $\infty$ en valeur absolue : \[ f(x) = x\left(1+\frac{2}{x}-\frac{4}{x^{2}}+\frac{\eps(1/x)}{x^{2}} \right) = x+2 -\frac{4}{x} + \frac{1}{x}\eps(1/x),\]c'est-à-dire que \[ \lim_{\abs{x}\to\infty}f(x) - (x+2) = 0.\]
On regarde maintenant la position de $f$ par rapport à $y = x+2$. On a \[ f(x) - (x+2) = \frac{-4}{x} + \frac{1}{x}\eps(1/x).\]Ainsi quand $x\to+\infty$, $f$ est en-dessous de l'asymptote, quand $x\to -\infty$, $f$ est au-dessus de l'asymptote.

\subsubsection{\'Etude de fonction}Par exemple avec \[ f(x) = x \log\abs{2 + \frac{1}{x}}\]de domaine de définition $\R\prive{0,-1/2}$.

\paragraph{Dérivée}On calcule la dérivée de $f$ : 
\begin{align*}
f(x) &= x(\log\abs{2x+1} - \log\abs{x}) \\
f'(x) &= \log\abs{2+\frac{1}{x}} + x\left(\frac{2}{2x+1} -\frac{1}{x}\right)\\
f'(x) &= \log\abs{2+\frac{1}{x}} - \frac{1}{2x+1}\\
f''(x) &= \frac{2}{2x+1} - \frac{1}{x} - \frac{2}{(2x+1)^{2}}\\
f''(x) &= \frac{-1}{x(2x+1)^{2}}
\end{align*}

Une étude des signes montre qu'il existe un unique $\alpha$ entre $-1/2$ et $0$ (strictement) tel que $f'(\alpha) = 0$.
En conclusion, $f$ est croissante partout sauf sur $]-1/2,\alpha[$ où elle est décroissante.
\begin{align*}
\lim_{x\to+\infty}f(x) &= +\infty \\
\lim_{x\to -\infty}f(x) &= -\infty
\end{align*}
ce qui nous donne deux branches infinies.
\begin{align*}
\lim_{x\to -1/2}f(x) &=+\infty
\end{align*}
et donc il y a une asymptote verticale en $x=-1/2$.
\begin{align*}
\lim_{x\to 0}f(x) &= ?\\
f(x) &= x\log\abs{2+\frac{1}{x}}\\
f(x) &= x\log\abs{2x+1}-x\log x
\end{align*}
or les deux termes tendent vers $0$ en $0$ et donc \[ \lim_{x\to 0}f(x) = 0\]et donc $f$ admet un prolongement par continuité à $0$ en $0$.

Il reste à regarder les branches infinies :
\begin{itemize}
\item Quand $x\to \infty$, on cherche un développement limité de $f$ en $+\infty$.
\begin{align*}
f(x) &= x\log(2+1/x) \\
f(x) &= x\log2+x\log\left(1+\frac{1}{2x}\right)\\
f(x) &= x\log 2 + x\left(\frac{1}{2x}-\frac{1}{8x^{2}}+\oo(1/x^{2})\right)\\
f(x) &= (\log 2)x + \frac{1}{2} -\frac{1}{8x} + \oo(1/x)
\end{align*}
On en déduit que la droite d'équation \[y = (\log 2)x +\frac{1}{2} \]est asymptote oblique à $f$  en $+\infty$ et la courbe est en-dessous de l'asymptote.
\item En $-\infty$ la droite d'équation \[ y = (\log 2)x + \frac{1}{2}\]est également asymptote oblique à $f$ en $-\infty$ et la courbe est au-dessus de l'asymptote.
\item Pour ce qui est de la tangente en $0$ : \[f'(x) = \log\abs{2+\frac{1}{x}} - \frac{1}{2x+1}\]et alors \[ \lim_{x\to 0}\frac{f(x)}{x} = \lim_{x\to 0}\log\abs{2 + \frac{1}{x}} = +\infty.\]
\end{itemize}
\begin{center}
\begin{sagesilent}
p = plot(x*log(2+1/x),(x,-2,2))
p += plot(log(2)*x + 1/2,(x,-2,2),rgbcolor = 'red')
\end{sagesilent}
\sageplot[width = 10cm]{p}
\end{center}















